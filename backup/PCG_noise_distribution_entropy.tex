\iffalse
\paragraph{Comparing the entropy of regular and nonregular noise.}View the $R^c$-$\LPN$ problem as a restricted case of primal $\LPN_{(c-1)N, N,\mathcal{D}}$ with public matrix being a circular matrix of size $(c-1)N\times N$, and the noise distribution over $R$ (or equivalently written as a vector in $\ZZ_p^N$) being either $\cHW_{R, t}$ or $\cRHW_{R,t}$. Moreover, let's consider using the entropy of the noise distribution as the security parameter (which means that we consider the adversary's attack being randomly guessing the secret). We use $(N_n, c_n, t_n)$ to denote the parameters for $\cHW^{\otimes c_n}_{R_n, t_n}$ and $(N_r, c_r, t_r)$ to denote the parameters for $\cRHW^{\otimes c_r}_{R_r, t_r}$. We'll compute the relation of two sets of parameters when they reach the same amount of entropy, under the assumption that $N\gg t$ (in PCG $N = 2^{20}$ and $t\le 76$). By Stirling formula, the entropy of $\cHW^{\otimes c_n}_{R_n, t_n}$ is 
\[
  \begin{split}
    \ln 2 \cdot E_n &\approx c_n\cdot \left(N_n\ln N_n - (N_n-t_n)\ln(N_n-t_n) - t_n\ln t_n\right) \\&\approx c_n t_n\left(\ln N_n - \ln t_n + 1\right) \quad(\text{when }N_n\gg t_n)
  \end{split}\]
and the entropy of $\cRHW^{\otimes c_r}_{R_r, t_r}$ is 
\[
    \ln 2\cdot E_r \approx c_rt_r\left(\ln N_r - \ln t_r\right) \quad(\text{when }N_r\gg t_r)
\]
To force $E_n = E_r$, let's consider fixing two varibles and change one: 
\begin{enumerate}
  \item Fix $(c_n, t_n) = (c_r, t_r)$, then $N_r = \mathrm{e}\cdot N_n$. 
  \item Fix $(N_n, c_n) = (N_r, c_r)$, then $t_n<t_r<t_n+1$. 
  \item Fix $(N_n, t_n) = (N_r, t_r)$, then $\frac{c_r}{c_n} = 1 + \frac{1}{N_n - t_n}\approx 1.05$. 
\end{enumerate}
All of these three changes of variables don't seem to lead to noticeable changes in the amortized expanding time for each OLE correlation. 
\fi

We focus on the following setting of this protocol, mentioned in \Cref{sec:PCG_OLE_protocol}, which achieves good computational efficiency and security simultaneously: 
\begin{enumerate}
    \item The hardness assumption is the QA-SD assumption for the group algebra $\mathbb{F}_q\left[\mathbb{G}\right]$ where $\mathbb{G} = \ZZ_p[X]/F(X)$ for a prime $p$ and a deg-$N$ polynomial $F(X)\in\ZZ_p[X]$, while the noise distribution is the regular weight-$t$ distribution $\cRHW_{R,t}$. 
    \item The polynomial ring $R = \ZZ_p[X]/F(X)$ where $F(X) = X^N+1$ is cyclotomic with $N$ being a power of two. 
    \item The underlying field's order $p$ is prime such that that $F(X)$ splits completely into linear factors modulo $p$, in order to convert an OLE correlations over $R$ to $N$ OLE correlations over $\ZZ_p$. 
\end{enumerate}
\begin{remark}
    In fact, using either the big-state DMPF, the PBC-based DMPF, or the OKVS-based DMPF schemes to share the low-weight polynomials does not need the restriction of the noise distribution to be regular. \red{However our experiments show that restricting to the regular noise distribution still leads to faster $\Expand$ time, so we reserve this setting. }\Yaxin{Tentatively. Double check with final implementation. }
\end{remark}

Leveraging the regular noise distribution and the cyclotomic polynomial $F$, the secret sharing task for generating one OLE correlation can be achieved using $c^2t\times \DMPF_{t,2N/t, \ZZ_p}$. Then the seed size, $\Gen$ time and $\Expand$ time of this PCG protocol are: 
\begin{itemize}
    \item The seed size is $ct(\log N+\log p)$ bits for specifying $\vec{e_b}$ plus the $c^2t\times $ keysize of $\DMPF_{t,2N/t, \ZZ_p}$. 
    \item The $\Gen$ time is $c^2$ multiplications in the ring $R$ plus $c^2t\times$ $\DMPF_{t,2N/t, \ZZ_p}.\Gen$. 
    \item The $\Expand$ time is $c^2$ multiplications in the ring $R$ plus $c^2t\times$ $\DMPF_{t,2N/t, \ZZ_p}.\FullEval$. 
\end{itemize}