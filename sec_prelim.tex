\subsection{Basic Notations}



 \paragraph{Point and multi-point functions.} Given a domain size $N$ and Abelian group $\GG$, a \emph{point function} $f_{\alpha,\beta}:[N]\rightarrow\GG$ for $\alpha\in[N]$ and $\beta\in\GG$ evaluates to $\beta$ on input $\alpha$ and to $0\in\GG$ on all other inputs. We denote by $\hat{f}_{\alpha,\beta}=(N,\hat{\GG},\alpha,\beta)$ the representation of such a point function. A \emph{$t$-point function} $f_{A,B}:[N]\rightarrow \GG$ for $A=(\alpha_1,\cdots\alpha_t)\in[N]^t$ and $B=(\beta_1,\cdots,\beta_t)\in \GG^t$ evaluates to $\beta_i$ on input $\alpha_i$ for $1\le i\le t$ and to $0$ on all other inputs. Denote $\hat{f}_{A,B}(N,\hat{\GG},t,A,B)$ the representation of such a $t$-point function. Call the collection of all $t$-point functions for all $t$ \emph{multi-point functions}. 
 
\Enote{MPF. Also representation of groups.}
 
\subsection{Distributed Multi-Point Functions}

\Enote{should directly adapt to multi-point function case}

We begin by defining a slightly generalized notion of distributed point functions (DPFs), which accounts for the extra parameter $\GG'$. \Yaxin{What is $\GG'$?}

%\yuval{Made some small stylistic changes below (similar changes may apply elsewhere). Should citations to GI14,BGI16. }
\begin{definition}[DPF \cite{EC:GilIsh14,CCS:BoyGilIsh16}]\label{def:dpf}
A 
%$t$-private $m$-server 
(2-party)
\emph{distributed point function (DPF)}
%, or $(m,t)$-DPF for short, 
is a triple of algorithms %$\Pi=(\Gen,\Eval_0,\ldots,\Eval_{m-1})$ 
$\Pi=(\Gen,\Eval_0,\Eval_1)$
with the following syntax: 
\begin{itemize}
    \item $\Gen(1^\lambda,\hat{f}_{\alpha,\beta})\rightarrow (k_0,k_1)$: On input security parameter $\lambda\in\NN$ and point function description $\hat{f}_{\alpha,\beta}=(N,\hat{\GG},\alpha,\beta)$, the (randomized) key generation algorithm $\Gen$ returns a pair of keys $k_0,k_1\in\{0,1\}^*$. \Yaxin{Matan points out: we want efficient procedures, i.e., $|k_b|\in\poly(\lambda)$. Stress it here or add efficiency requirement?} 
    We assume that $N$ and $\GG$ are determined by each key.
    \item $\Eval_b(k_b,x)\rightarrow y_b$: On input key $k_b\in\{0,1\}^*$ and input $x\in[N]$ the (deterministic) evaluation algorithm of server $b$, $\Eval_b$ returns 
    %a group element 
    $y_b\in\GG$.
\end{itemize}
%The algorithms $\Pi=(\Gen,\Eval_0,\ldots,\Eval_{m-1})$ should 
We require $\Pi$ to satisfy the following requirements:
\begin{itemize}
    \item \textbf{Correctness:} For every $\lambda$, $\hat{f}=\hat{f}_{\alpha,\beta}=(N,\hat{\GG},\alpha,\beta)$ such that $\beta\in\GG$, and $x\in[N]$, for $b=0,1$,  
    $$\Pr\left[(k_0,k_1)\leftarrow\Gen(1^\lambda,\hat{f}), \sum_{i=0}^{1}\Eval_b(k_b,x)=f_{\alpha,\beta}(x)\right]=1$$
    \item \textbf{Security:} Consider the following semantic security challenge experiment for corrupted server $b\in\{0,1\}$:
    \begin{enumerate}
        \item The adversary produces two point function descriptions $(\hat{f}^0=(N,\hat\GG,\alpha_0,\beta_0),\hat{f}^1=(N,\hat\GG,\alpha_1,\beta_1))\leftarrow\mathcal{A}(1^\lambda)$, where $\alpha_b\in[N]$ and $\beta_b\in\GG$.
        \item The challenger samples $b\gets\{0,1\}$ and $(k_0,k_1)\leftarrow\Gen(1^\lambda,\hat{f}^b)$.
        \item The adversary outputs a guess $b'\leftarrow\mathcal{A}(k_b)$.
    \end{enumerate}
    Denote by $\Adv(1^\lambda,\mathcal{A},i)=\Pr[b=b']-1/2$ the advantage of $\mathcal{A}$ in guessing $b$ in the above experiment. For every non-uniform polynomial time adversary $\mathcal{A}$ there exists a negligible function $\nu$ such that $\Adv(1^\lambda,\mathcal{A},i) \le \nu(\lambda)$ for all $\lambda \in \NN$.
%    For circuit size bound $S=S(\lambda)$ and advantage bound $\epsilon(\lambda)$, we say that $\Pi$ is $(S,\epsilon)$-secure if for all $i\in\{0,1\}$ and all non-uniform adversaries $\mathcal{A}$ of size $S(\lambda)$ and sufficiently large $\lambda$, we have $\Adv(1^\lambda,\mathcal{A},i)\leq\epsilon(\lambda)$. We say that $\Pi$ is:
%    \begin{itemize}
%        \item \emph{Computationally $\epsilon$-secure} if it is $(S,\epsilon)$-secure for all polynomials $S$.
%        \item \emph{Computationally secure} if it is $(S,1/S)$-secure for all polynomials $S$.
%        %\item \emph{Statistically $\epsilon$-secure} if it is $(S,\epsilon)$-secure for all $S$.
%        %\item \emph{Perfectly secure} if it is statistically $0$-secure.
%    \end{itemize}
\end{itemize}
%If the security threshold $t$ is unspecified, we assume it is $t=1$.
\end{definition}

\begin{definition}[DMPF]\label{def:dmpf}
  A 
  %$t$-private $m$-server 
  (2-party)
  \emph{distributed multi-point function (DMPF)}
  %, or $(m,t)$-DPF for short, 
  is a triple of algorithms %$\Pi=(\Gen,\Eval_0,\ldots,\Eval_{m-1})$ 
  $\Pi=(\Gen,\Eval_0,\Eval_1)$
  with the following syntax: 
  \begin{itemize}
      \item $\Gen(1^\lambda,\hat{f}_{A,B})\rightarrow (k_0,k_1)$: On input security parameter $\lambda\in\NN$ and point function description $\hat{f}_{A,B}=(N,\hat{\GG},t,A,B)$, the (randomized) key generation algorithm $\Gen$ returns a pair of keys $k_0,k_1\in\{0,1\}^*$. \Yaxin{On Matan's behalf: same comment as well. Maybe $|k_i|=\poly(\lambda,t)$. }
      \item $\Eval_b(k_b,x)\rightarrow y_b$: On input key $k_b\in\{0,1\}^*$ and input $x\in[N]$ the (deterministic) evaluation algorithm of server $b$, $\Eval_b$ returns $y_b\in\GG$.
  \end{itemize}
  We require $\Pi$ to satisfy the following requirements:
  \begin{itemize}
      \item \textbf{Correctness:} For every $\lambda$, $\hat{f}=\hat{f}_{A,B}=(N,\hat{\GG},t,A,B)$ such that $B\in\GG^t$, and $x\in[N]$, for $b=0,1$,
      $$\Pr\left[(k_0,k_1)\leftarrow\Gen(1^\lambda,\hat{f}), \sum_{i=0}^{1}\Eval_b(k_b,x)=f_{A,B}(x)\right]=1$$
      \item \textbf{Security:} Consider the following semantic security challenge experiment for corrupted server $b\in\{0,1\}$:
      \begin{enumerate}
          \item The adversary produces two $t$-point function descriptions $(\hat{f}^0=(N,\hat\GG,t,A_0,B_0),\hat{f}^1=(N,\hat\GG,t,A_1,B_1))\leftarrow\mathcal{A}(1^\lambda)$, where $\alpha_b\in[N]$ and $\beta_b\in\GG$.
          \item The challenger samples $b\gets\{0,1\}$ and $(k_0,k_1)\leftarrow\Gen(1^\lambda,\hat{f}^b)$.
          \item The adversary outputs a guess $b'\leftarrow\mathcal{A}(k_b)$.
      \end{enumerate}
      Denote by $\Adv(1^\lambda,\mathcal{A},i)=\Pr[b=b']-1/2$ the advantage of $\mathcal{A}$ in guessing $b$ in the above experiment. For every non-uniform polynomial time adversary $\mathcal{A}$ there exists a negligible function $\nu$ such that $\Adv(1^\lambda,\mathcal{A},i) \le \nu(\lambda)$ for all $\lambda \in \NN$.
  \end{itemize}
  \end{definition}
 
 We will also be interested in applying the evaluation algorithm on \emph{all} inputs. Given a DMPF $(\Gen,\Eval_0,\Eval_1)$, we denote by $\FullEval_b$ an algorithm which computes $\Eval_b$ on every input $x$. Hence, $\FullEval_b$ receives only a key $k_b$ as input.



\subsection{Batch Code}
We introduce batch code and probabilistic batch code, which can be used to construct DMPF (see \cref{constr:DMPF_batch_code}). 
\begin{definition}[Batch Code\cite{10.1145/1007352.1007396}]
  An $(N,M,t,m)$-batch code over alphabet $\Sigma$ is given by a pair of efficient algorithms $(\Encode,\Decode)$ such that:
  \begin{itemize}
    \item $\Encode(x\in\Sigma^N)\rightarrow (C_1,C_2,\cdots,C_m)$: Any string $x\in\Sigma^N$ is encoded into an $m$-tuple of strings $C_1,C_2,\cdots C_m\in\Sigma^*$ (called buckets) of total length $M$.
    \item $\Decode(I,C_1,C_2,\cdots,C_m)\rightarrow \{x[i]\}_{i\in I}$: On input a set $I$ of $t$ distinct indices in $[N]$ and $m$ buckets, recover $t$ coordinates of $x$ indexed by $I$ by reading at most one coordinate from each of the $m$ buckets. 
  \end{itemize}
\end{definition}

An $(N,M,t,m)$-batch code can be represented by an $(N,m)$-bipartite graph $G=(U,V,E)$ where each edge $(u_i,v_j)\in E$ corresponds to $\Encode$ assigning $x[i]$ to the bucket $C_j$, while it is guaranteed that any subset $S\subseteq U$ such that $|S|=t$ has a perfect matching to $V$. \Yaxin{Add example instantiation (random regular bipartite graph) and explain it is not efficient. }

\begin{definition}[Probabilistic Batch Code (PBC)\cite{cryptoeprint:2017/1142}]
An\\ $(N,M,t,m,\epsilon)$-probabilistic batch code over alphabet $\Sigma$ is a randomized $(N,M,t,m)$-batch code with public randomness $r$ (and possibly private randomness for each sub-procedure) such that for any string $x$ and any set $I$ of $t$ distinct indices in $[N]$, 
\[
  \Pr[\Decode_r(I,\Encode_r(x))\rightarrow \{x[i]\}_{i\in I}] = 1-\epsilon
\]
where the probability is taken over the public randomness $r$ and private randomness of $\Encode$ and $\Decode$ algorithms. 
\end{definition}

We mention Cuckoo hashing algorithm\cite{10.1007/3-540-44676-1_10} as a concrete instantiation of PBC\cite{cryptoeprint:2017/1142}.

\paragraph{$w$-way cuckoo hashing}Given $t$ balls, $m=et$ buckets ($e$ is some expansion parameter that is bigger than 1), and $w$ independent hash functions $h_1, h_2,\cdots, h_w$ randomly mapping every ball to a bucket, allocates all balls to the buckets such that each bucket contains at most one ball through the following process: 
\begin{itemize}
  \item[1.] Choose an arbitrary unallocated ball $b$. If there is no unallocated ball, output the allocation. 
  \item[2.] Choose a random hash function $h_i$ compute the bucket index $h_i(b)$. If this bucket is empty, then allocate $b$ to this bucket and go to step 1. If this bucket is not empty and filled with ball $b'$, then evict $b'$, allocate $b$ to this bucket set $b'$ the current unallocated ball, and repeat step 2. 
\end{itemize}
If the algorithm terminates then its output is an allocation of balls to buckets such that each bucket contains at most one ball. However there is no guarantee that the algorithm will terminate - it may end up in a loop and keeps running forever. To fixed this problem, the algorithm should be given a fixed amount of time to run, or equipped with a loop detection process to guarantee termination. We call it a failure whenver the algorithm fails to output a proper allocation where each bucket contains at most one ball. 

\Yaxin{Add asymptotic parameters? Also find evident theorem saying cuckoo hashing is efficient. }

\paragraph{The failure probability of cuckoo hashing}Let's denote the failure probability of $w$-way cuckoo hashing to be $\epsilon=2^{-\lambda_{\stat}}$. In practice we usually consider the statistical security parameter $\lambda_{\stat}$ to be $30$ or $40$. The empirical result in \cite{chen_fast_2017} shows for $w=3$, $m=16384$, $\lambda_\stat = 124.4e-144.6$ where $e$ is the expansion parameter that $m=et$. For $w=3$, $m=8192$, $\lambda_\stat=125e-145$. However we use cuckoo hashing to construct DMPF for $t$-point functions, in which case we'd also care about $t$ being small, say $2,3$ or $100$, and $m$ should not be too large. In this sense the previous empirical results are not complete. \Yaxin{\cite{cryptoeprint:2017/1142} uses $w=3, e=1.5, t>200$ and $\lambda_\stat\approx 40$ and claims it follows the analysis from \cite{chen_fast_2017}, but I don't see how...}

The balls, buckets and hash functions can be represented by a $w$-regular $(t,m)$-bipartite graph $G=(U,V,E)$ where each left node has $w$ neighbors, and each edge $(u_i,v_j)\in E$ corresponds to $h_l(i)=j$ for some $1\le l\le w$. In this graph representation the $w$-way cuckoo hashing essentially computes a perfect matching from $U$ to $V$. Therefore one can construct a PBC from cuckoo hashing. 

\begin{construction}[PBC from cuckoo hashing]
  Given $w$-way cuckoo hashing as a sub-procedure allocating $t$ balls to $m$ buckets with failure probability $\epsilon$, an $(N,wN,t,m,\epsilon)$-PBC is as follows: 
  \begin{itemize}
    \item $\Encode_r(x\in\Sigma^N)\rightarrow (C_1,\cdots,C_m)$: Use $r$ to determine $w$ independent random hash functions $h_1,h_2,\cdots h_w$ that maps from $[N]$ to $[m]$. Initialize $C_1,\cdots,C_m$ to be empty. For each $i\in[N]$, append $x[i]$ to $C_{h_j(i)}$ for $1\le j\le w$. 
    \item $\Decode_r(I, C_1,\cdots, C_m)\rightarrow \{x[i]\}_{i\in I}$: Determine $h_1,\cdots, h_w$ as in $\Encode$. For $I$ of size $t$, allocate $I$ to $[m]$ using $w$-way cuckoo hashing. For each $i\in I$, fetch $x[i]$ from $C_j$ where $i$ is allocated to $j$. 
  \end{itemize}
\end{construction}
\subsection{Oblivious Key-Value Stores}
\begin{definition}[Oblivious Key-Value Stores (OKVS)\cite{cryptoeprint:2021/883,cryptoeprint:2022/320}]\label{def:OKVS}
  An Oblivious Key-Value Stores scheme is a pair of randomized algorithms $(\Encode_r,\Decode_r)$ with respect to a statistical security parameter $\lambda_{\sf stat}$ and a computational security parameter $\lambda$, a randomness space $\{0,1\}^\kappa$, a key space $\mathcal{K}$, a value space $\mathcal{V}$, input length $t$ and output length $m$. The algorithms are of the following syntax: 
  \begin{itemize}
    \item $\Encode_r(\{(k_1,v_1),(k_2,v_2),\cdots,(k_t,v_t)\})\rightarrow P$: On input $t$ key-value pairs with distinct keys, the encode algorithm with randomness $r$ in the randomness space outputs an encoding $P\in\mathcal{V}^m\cup\bot$.
    \item $\Decode_r(P,k)\rightarrow v$: On input an encoding from $\mathcal{V}^m$ and a key $k\in\mathcal{K}$, output a value $v$. 
  \end{itemize}
  We require the scheme to satisfy
  \begin{itemize}
    \item For all $S\in(\mathcal{K}\times\mathcal{V})^t$, $\Pr_{r\leftarrow\{0,1\}^\kappa}[\Encode_r(S)=\bot]\le 2^{-\lambda_{\sf stat}}$. 
    \item For all $S\in(\mathcal{K}\times \mathcal{V})^t$ and $r\in \{0,1\}^\kappa$ such that $\Encode_r(S)\rightarrow P\not=\bot$, it is the case that $\Decode_r(P,k)\rightarrow v$ whenever $(k,v)\in S$. 
    \item \textbf{Obliviousness: }Given any distinct key sets $\{k_1^0,k_2^0,\cdots,k_t^0\}$ and $\{k_1^1,k_2^1,\cdots,k_t^1\}$ that are different, if they are paired with random values then their encodings are computationally indistinguishable, i.e., 
  \begin{align*}
    &\{r, \Encode_r(\{(k_1^0,v_1),\cdots,(k_t^0,v_t)\})\}_{v_1,\cdots,v_t\leftarrow \mathcal{V},r\leftarrow\{0,1\}^\kappa}\\
    \approx_c &\{r, \Encode_r(\{(k_1^1,v_1),\cdots,(k_t^1,v_t)\})\}_{v_1,\cdots,v_t\leftarrow \mathcal{V},r\leftarrow\{0,1\}^\kappa}
  \end{align*}
  \end{itemize}
One can obtain a \emph{linear OKVS} if in addition require:
\begin{itemize}
  \item \textbf{Linearity: }There exists a function family $\{\row_r:\mathcal{K}\rightarrow\mathcal{V}^m\}_{r\in\{0,1\}^\kappa}$ such that $\Decode_r(P,k) = \ipd{\row_r(k)}{P}$. 
\end{itemize}
\end{definition}
The $\Encode$ process for a linear OKVS is the process of sampling a random $P$ from the set of solutions of the linear system $\{\ipd{\row_r(k_i)}{P} = v_i\}_{1\le i\le t}$. 

We evaluate an OKVS scheme by its encoding size (output length $m$), encoding time and decoding time. We stress the following two (linear) OKVS constructions:
\begin{construction}[Polynomial]
  Suppose $\mathcal{K} = \mathcal{V}=\FF$ is a field. Set 
  \begin{itemize}
    \item $\Encode(\{(k_i,v_i)\}_{1\le i\le t}) \rightarrow P$ where $P$ is the coeffients of a $(t-1)$-degree $\FF$-polynomial $g_P$ that $g_P(k_i) = v_i$ for $1\le i\le t$. 
    \item $\Decode(P,k)\rightarrow g_P(k)$. 
  \end{itemize}
\end{construction}
The polynomial OKVS possesses an optimal encoding size $m=n$, but the $\Encode$ process is a polynomial interpolation which is only known to be achieved in time $O(t\log^2t)$. The time for a single decoding is $O(t)$ and that for batched decodings is (amortized) $O(\log^2 t)$. 

An alternative construction that has near optimal encoding size but much better running time is as follows. 
\begin{construction}[3-Hash Garbled Cuckoo Table (3H-GCT)\cite{cryptoeprint:2021/883,cryptoeprint:2022/320}]
  Suppose $\mathcal{V}=\FF$ is a field. Set $\row_r(k):=\row_r^{\sf sparse}(k)||\row_r^{\sf dense}(k)$ where $\row_r^{\sf sparse}$ outputs a uniformly random weight-$w$ vector in $\{0,1\}^{m_1}$, and $\row_r^{\sf dense}(k)$ outputs a short dense vector in $\FF^{m_2}$. 
  \begin{itemize}
    \item $\Encode(\{(k_i,v_i)\}_{1\le i\le t}) \rightarrow P$ where $P$ is solved from the system $\{\ipd{\row_r(k_i)}{P} = v_i\}_{1\le i\le t}$ using the triangulation algorithm in \cite{cryptoeprint:2022/320}. 
    \item $\Decode(P,k)\rightarrow \ipd{\row_r(k)}{P}$. 
  \end{itemize}
  We denote $m_1=et$, where $e$ is an expandion parameter indicating the rough blowup to store $t$ pairs. In practice the number of dense columns $m_2$ is usually set to a small constant. 
\end{construction}
This OKVS construction features a linear encoding time, constant decoding time (the constant relates to $w$ and $m_2$) while having a linear encoding size. 

\Yaxin{TBD: Carefully(!) recompute the comparison table for OKVS. }

We'll mostly use the expansion parameter $e$ and the number of dense columns $m_2:=\hat{g}$ (where $\hat{g}$ is a parameter relating to the equation system solving process) according to the analysis in \cite{cryptoeprint:2022/320}: Given $w$, $t$ and $\lambda_\stat$, the choices of the $e$ and $\hat{g}$ are fixed through the following steps: 
\begin{itemize}
  \item Set $e^* = \begin{cases}
    1.223 & w=3\\
    1.293 & w=4\\
    0.1485w+0.6845 & w\ge 5
  \end{cases}$.
  \item Compute $\alpha:=0.55 \log_2 t + 0.093w^3-1.01w^2 + 2.92w-0.13$.
  \item $e:=e^*+ 2^{-\alpha}(\lambda_\stat+9.2)$. 
  \item $\hat{g}:=\frac{\lambda_\stat}{(w-2)\log_2(et)}$. 
\end{itemize}


\Yaxin{Fix $t$ and $\lambda_\stat$, we want to find the best choice of $w$. The adavantageous choices of $w$ in \cite{cryptoeprint:2022/320} are $w=3$ and $w=5$. From the first sight when $w$ is smaller $e$ can be smaller but $\hat{g}$ will be larger. Since $e$ stands for number of $\FF$-ADD's and $\hat{g}$ stands for number of $\FF$-MULT's in decoding, previously I thought $\hat{g}$ is the dominating factor of $\Decode$ running time. However table 1 in \cite{cryptoeprint:2022/320} suggests that $w=3$ outruns nearly all of other choices of $w$ while $w=5$ is almost 3 times slower in decoding time. This may suggest there are some other heavy computations other than $\FF$-MULT that need to be considered when evaluating running time. 

The range of $t$ previous literature \cite{cryptoeprint:2021/883,cryptoeprint:2022/320} have considered in their empirical results are also limited, which will be one of our problems. We want to cover small $t$, say $t<100$, while previous literature aiming for constructing PSI protocols usually consider very large $t$. }

One may also let $row_r^{\sf dense}$ output a short dense vector in $\{0,1\}^{m_2}$, which avoids multiplication of large field elements in the encoding and decoding processes. To achieve same level of security one could simply set $m_2=\hat{g}+\lambda_\stat$, as proposed in \cite{cryptoeprint:2021/883,cryptoeprint:2022/320}. 
\Yaxin{TBD: mention some connections to cuckoo hashing?}
