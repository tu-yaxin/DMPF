\section{Preliminary}

\subsection{Basic Notations}



 \paragraph{Point and multi-point functions.} Given a domain size $N$ and Abelian group $\GG$, a \emph{point function} $f_{\alpha,\beta}:[N]\rightarrow\GG$ for $\alpha\in[N]$ and $\beta\in\GG$ evaluates to $\beta$ on input $\alpha$ and to $0\in\GG$ on all other inputs. We denote by $\hat{f}_{\alpha,\beta}=(N,\hat{\GG},\alpha,\beta)$ the representation of such a point function. A \emph{$t$-point function} $f_{A,B}:[N]\rightarrow \GG$ for $A=(\alpha_1,\cdots\alpha_t)\in[N]^t$ and $B=(\beta_1,\cdots,\beta_t)\in \GG^t$ evaluates to $\beta_i$ on input $\alpha_i$ for $1\le i\le t$ and to $0$ on all other inputs. Denote $\hat{f}_{A,B}(N,\hat{\GG},t,A,B)$ the representation of such a $t$-point function. Call the collection of all $t$-point functions for all $t$ \emph{multi-point functions}. 
 
\Enote{MPF. Also representation of groups.}
 
\subsection{Distributed Multi-Point Functions}

\Enote{should directly adapt to multi-point function case}

We begin by defining a slightly generalized notion of distributed point functions (DPFs), which accounts for the extra parameter $\GG'$. \Yaxin{What is $\GG'$?}

%\yuval{Made some small stylistic changes below (similar changes may apply elsewhere). Should citations to GI14,BGI16. }
\begin{definition}[DPF \cite{EC:GilIsh14,CCS:BoyGilIsh16}]\label{def:dpf}
A 
%$t$-private $m$-server 
(2-party)
\emph{distributed point function (DPF)}
%, or $(m,t)$-DPF for short, 
is a triple of algorithms %$\Pi=(\Gen,\Eval_0,\ldots,\Eval_{m-1})$ 
$\Pi=(\Gen,\Eval_0,\Eval_1)$
with the following syntax: 
\begin{itemize}
    \item $\Gen(1^\lambda,\hat{f}_{\alpha,\beta})\rightarrow (k_0,k_1)$: On input security parameter $\lambda\in\NN$ and point function description $\hat{f}_{\alpha,\beta}=(N,\hat{\GG},\alpha,\beta)$, the (randomized) key generation algorithm $\Gen$ returns a pair of keys $k_0,k_1\in\{0,1\}^*$. \Yaxin{Matan points out: we want efficient procedures, i.e., $|k_b|\in\poly(\lambda)$. Stress it here or add efficiency requirement?} 
    We assume that $N$ and $\GG$ are determined by each key.
    \item $\Eval_b(k_b,x)\rightarrow y_b$: On input key $k_b\in\{0,1\}^*$ and input $x\in[N]$ the (deterministic) evaluation algorithm of server $b$, $\Eval_b$ returns 
    %a group element 
    $y_b\in\GG$.
\end{itemize}
%The algorithms $\Pi=(\Gen,\Eval_0,\ldots,\Eval_{m-1})$ should 
We require $\Pi$ to satisfy the following requirements:
\begin{itemize}
    \item \textbf{Correctness:} For every $\lambda$, $\hat{f}=\hat{f}_{\alpha,\beta}=(N,\hat{\GG},\alpha,\beta)$ such that $\beta\in\GG$, and $x\in[N]$, for $b=0,1$,  
    $$\Pr\left[(k_0,k_1)\leftarrow\Gen(1^\lambda,\hat{f}), \sum_{i=0}^{1}\Eval_b(k_b,x)=f_{\alpha,\beta}(x)\right]=1$$
    \item \textbf{Security:} Consider the following semantic security challenge experiment for corrupted server $b\in\{0,1\}$:
    \begin{enumerate}
        \item The adversary produces two point function descriptions $(\hat{f}^0=(N,\hat\GG,\alpha_0,\beta_0),\hat{f}^1=(N,\hat\GG,\alpha_1,\beta_1))\leftarrow\mathcal{A}(1^\lambda)$, where $\alpha_b\in[N]$ and $\beta_b\in\GG$.
        \item The challenger samples $b\gets\{0,1\}$ and $(k_0,k_1)\leftarrow\Gen(1^\lambda,\hat{f}^b)$.
        \item The adversary outputs a guess $b'\leftarrow\mathcal{A}(k_b)$.
    \end{enumerate}
    Denote by $\Adv(1^\lambda,\mathcal{A},i)=\Pr[b=b']-1/2$ the advantage of $\mathcal{A}$ in guessing $b$ in the above experiment. For every non-uniform polynomial time adversary $\mathcal{A}$ there exists a negligible function $\nu$ such that $\Adv(1^\lambda,\mathcal{A},i) \le \nu(\lambda)$ for all $\lambda \in \NN$.
%    For circuit size bound $S=S(\lambda)$ and advantage bound $\epsilon(\lambda)$, we say that $\Pi$ is $(S,\epsilon)$-secure if for all $i\in\{0,1\}$ and all non-uniform adversaries $\mathcal{A}$ of size $S(\lambda)$ and sufficiently large $\lambda$, we have $\Adv(1^\lambda,\mathcal{A},i)\leq\epsilon(\lambda)$. We say that $\Pi$ is:
%    \begin{itemize}
%        \item \emph{Computationally $\epsilon$-secure} if it is $(S,\epsilon)$-secure for all polynomials $S$.
%        \item \emph{Computationally secure} if it is $(S,1/S)$-secure for all polynomials $S$.
%        %\item \emph{Statistically $\epsilon$-secure} if it is $(S,\epsilon)$-secure for all $S$.
%        %\item \emph{Perfectly secure} if it is statistically $0$-secure.
%    \end{itemize}
\end{itemize}
%If the security threshold $t$ is unspecified, we assume it is $t=1$.
\end{definition}

\begin{definition}[DMPF]\label{def:dmpf}
  A 
  %$t$-private $m$-server 
  (2-party)
  \emph{distributed multi-point function (DMPF)}
  %, or $(m,t)$-DPF for short, 
  is a triple of algorithms %$\Pi=(\Gen,\Eval_0,\ldots,\Eval_{m-1})$ 
  $\Pi=(\Gen,\Eval_0,\Eval_1)$
  with the following syntax: 
  \begin{itemize}
      \item $\Gen(1^\lambda,\hat{f}_{A,B})\rightarrow (k_0,k_1)$: On input security parameter $\lambda\in\NN$ and point function description $\hat{f}_{A,B}=(N,\hat{\GG},t,A,B)$, the (randomized) key generation algorithm $\Gen$ returns a pair of keys $k_0,k_1\in\{0,1\}^*$. \Yaxin{On Matan's behalf: same comment as well. Maybe $|k_i|=\poly(\lambda,t)$. }
      \item $\Eval_b(1^\lambda, k_b,x)\rightarrow y_b$: On input key $k_b\in\{0,1\}^*$ and input $x\in[N]$ the (deterministic) evaluation algorithm of server $b$, $\Eval_b$ returns $y_b\in\GG$.
  \end{itemize}
  We require $\Pi$ to satisfy the following requirements:
  \begin{itemize}
      \item \textbf{Correctness:} For every $\lambda$, $\hat{f}=\hat{f}_{A,B}=(N,\hat{\GG},t,A,B)$ such that $B\in\GG^t$, and $x\in[N]$, for $b=0,1$,
      $$\Pr\left[(k_0,k_1)\leftarrow\Gen(1^\lambda,\hat{f}), \sum_{i=0}^{1}\Eval_b(k_b,x)=f_{A,B}(x)\right]=1$$
      \item \textbf{Security:} Consider the following semantic security challenge experiment for corrupted server $b\in\{0,1\}$:
      \begin{enumerate}
          \item The adversary produces two $t$-point function descriptions $(\hat{f}^0=(N,\hat\GG,t,A_0,B_0),\hat{f}^1=(N,\hat\GG,t,A_1,B_1))\leftarrow\mathcal{A}(1^\lambda)$, where $\alpha_b\in[N]$ and $\beta_b\in\GG$.
          \item The challenger samples $b\gets\{0,1\}$ and $(k_0,k_1)\leftarrow\Gen(1^\lambda,\hat{f}^b)$.
          \item The adversary outputs a guess $b'\leftarrow\mathcal{A}(k_b)$.
      \end{enumerate}
      Denote by $\Adv(1^\lambda,\mathcal{A},i)=\Pr[b=b']-1/2$ the advantage of $\mathcal{A}$ in guessing $b$ in the above experiment. For every non-uniform polynomial time adversary $\mathcal{A}$ there exists a negligible function $\nu$ such that $\Adv(1^\lambda,\mathcal{A},i) \le \nu(\lambda)$ for all $\lambda \in \NN$.
  \end{itemize}
  \end{definition}
 
 We will also be interested in applying the evaluation algorithm on \emph{all} inputs. Given a DMPF $(\Gen,\Eval_0,\Eval_1)$, we denote by $\FullEval_b$ an algorithm which computes $\Eval_b$ on every input $x$. Hence, $\FullEval_b$ receives only a key $k_b$ as input.

 One can construct a DMPF scheme for $t$-point functions by simply summing $t$ DPFs. We denote this DMPF scheme as the na\"ive construction. 
\begin{construction}[Na\"ive construction of DMPF]
  Given DPF for domain of size $N$ and output group $\GG$, we can construct a DMPF scheme for $t$-point functions with domain size $N$ and output group $\GG$ as follows: 
  \begin{itemize}
    \item $\Gen(1^\lambda, \hat{f}_{A, B})\rightarrow (k_0, k_1)$: Suppose $A = \{\alpha_1,\dots, \alpha_t\}$ and $B = \{\beta_1,\dots, \beta_t\}$. For $1\le i\le t$, invoke DPF$.\Gen(1^\lambda, \hat{f}_{\alpha_i, \beta_i})\rightarrow (k_0^i, k_1^i)$. Set $(k_0, k_1) = (\{k_0^i\}_{i\in [t]}, \{k_1^i\}_{i\in [t]})$. 
    \item $\Eval_b(k_b, x)\rightarrow y_b$: Compute $y_b = \sum_{i\in [t]}$DPF$.\Eval_b(k_b^i, x)$. 
    \item $\FullEval_b(k_b)\rightarrow Y_b$: Compute $Y_b = \sum_{i\in [t]}$DPF$.\FullEval_b(k_b^i, x)$. 
  \end{itemize}
\end{construction}
When the DPF scheme is correct and secure, the na\"ive construction of DMPF is also correct and secure. We note that the keysize and running time of $\Gen$, $\Eval$ and $\FullEval$ of the na\"ive construction of DMPF equals $t\times $ the keysize and $t\times$ the running time of $\Gen$, $\Eval$ and $\FullEval$ of DPF, respectively. We aim to provide DMPF schemes that has $\Eval$ and $\FullEval$ time almost independent to $t$. 
 
 

\subsection{Batch Code}
We introduce batch code and probabilistic batch code, which can be used to construct DMPF (see \cref{con:DMPF_batch_code}). 
\begin{definition}[Batch Code\cite{10.1145/1007352.1007396}]
  An $(N,M,t,m)$-batch code over alphabet $\Sigma$ is given by a pair of efficient algorithms $(\Encode,\Decode)$ such that:
  \begin{itemize}
    \item $\Encode(x\in\Sigma^N)\rightarrow (C_1,C_2,\cdots,C_m)$: Any string $x\in\Sigma^N$ is encoded into an $m$-tuple of codewords $C_1,C_2,\cdots C_m\in\Sigma^*$ of total length $M$.
    \item $\Decode(I,C_1,C_2,\cdots,C_m)\rightarrow \{x[i]\}_{i\in I}$: On input a set $I$ of $t$ distinct indices in $[N]$ and $m$ codewords, recover $t$ coordinates of $x$ indexed by $I$ by reading at most one coordinate from each of the $m$ codeword. 
  \end{itemize}
\end{definition}

We will focus on a special class of batch code called combinatorial batch code (CBC)\cite{10.1145/1007352.1007396,cryptoeprint:2008/306}, where each codeword $C_i$ is a subset of $x$ and during decoding the elements read from $m$ codewords should cover the desired set $\{x[i]\}_{i\in I}$. 
%Note that the existence of an $(N,M,t,m)$-CBC is equivalent to the existence of an $(N,M,t,m)$-CBC where the encoding and decoding are independent of the input string $x$. Thus we can only consider the latter CBC.
 In this case, the encoding algorithm is equivalent to replicating and allocating the indices in $[N]$ to $m$ buckets, and the decoding algorithm is equivalent to finding a prefect matching from the size-$t$ subset $I\subseteq[N]$ to the $m$ buckets, while indicating which position in each bucket should be read. 
%The latter $(N,M,t,m)$-CBC can be represented by an $(N,m)$-bipartite graph $G=(U,V,E)$ where each edge $(u_i,v_j)\in E$ corresponds to $\Encode$ assigning $x[i]$ to the codeword $C_j$ (we call it a bucket), while it is guaranteed that any subset $T\subseteq U$ such that $|T|=t$ has a perfect matching to $V$. 

\Yaxin{Add example instantiation (random regular bipartite graph) and explain it is not efficient? }

It is useful to relax the definition of batch code, allowing failure probability when decoding. 

\begin{definition}[Probabilistic Batch Code \cite{cryptoeprint:2017/1142}]
An\\ $(N,M,t,m,\epsilon)$-probabilistic batch code over alphabet $\Sigma$ is a randomized $(N,M,t,m)$-batch code with \textbf{public randomness} $r$ (and possibly private randomness for each sub-procedure) such that for any string $x$ and any set $I$ of $t$ distinct indices in $[N]$, 
\[
  \Pr[\Decode_r(I,\Encode_r(x))\rightarrow \{x[i]\}_{i\in I}] = 1-\epsilon
\]
where the probability is taken over the public randomness $r$ and private randomness of  $\Encode$ and $\Decode$ algorithms. 
\end{definition}

Similarly, a probabilistic CBC (PCBC) will just be a CBC with failure probability when decoding. We mention Cuckoo hashing algorithm\cite{10.1007/3-540-44676-1_10} as a concrete instantiation of PCBC\cite{cryptoeprint:2017/1142,yeo_cuckoo_2023}.

\paragraph{$w$-way cuckoo hashing.}Given $t$ balls, $m=et$ buckets ($e$ is some expansion parameter that is bigger than 1), and $w$ independent hash functions $h_1, h_2,\cdots, h_w$ randomly mapping every ball to a bucket, allocates all balls to the buckets such that each bucket contains at most one ball through the following process: 
\begin{itemize}
  \item[1.] Choose an arbitrary unallocated ball $b$. If there is no unallocated ball, output the allocation. 
  \item[2.] Choose a random hash function $h_i$ compute the bucket index $h_i(b)$. If this bucket is empty, then allocate $b$ to this bucket and go to step 1. If this bucket is not empty and filled with ball $b'$, then evict $b'$, allocate $b$ to this bucket, set $b'$ the current unallocated ball, and repeat step 2. 
\end{itemize}
If the algorithm terminates then its output is an allocation of balls to buckets such that each bucket contains at most one ball. However there is no guarantee that the algorithm will terminate - it may end up in a loop and keeps running forever. To fixed this problem, the algorithm should be given a fixed amount of time to run, or equipped with a loop detection process to guarantee termination. We call it a \emph{failure} whenver the algorithm fails to output a proper allocation where each bucket contains at most one ball. 

\paragraph{The failure probability of cuckoo hashing.}Let's denote the failure probability of $w$-way cuckoo hashing to be $\epsilon=2^{-\lambda_{\stat}}$. In practice we usually consider the statistical security parameter $\lambda_{\stat}$ to be $40$. The relations among the number of balls $t$, the number of hash functions $w$, the number of buckets $m$ and the security parameters $\lambda_\stat$ are listed in \Cref{tab:cuckoo_hashing_prm}. 

\begin{table*}
  \renewcommand\arraystretch{1.5}
  %\scalebox{1}{
  \begin{threeparttable}
  \caption{he relations among the number of balls $t$, the number of hash functions $w$, the number of buckets $m$ and the security parameters $\lambda_\stat$ in cuckoo hashing. }
  \label{tab:cuckoo_hashing_prm}
    \begin{tabular}{cccccc}
      \toprule 
      %Header
      &  Type&$t$ &$\lambda_\stat$ & $w$ & $e = m/t$  \\
       

      \midrule
      \cite[Theorem 1]{yeo_cuckoo_2023}\tnote{$\dag$}& Asymptotic & & & $O(\sqrt{\lambda_{\stat}\log t})$ & $O(1)$ \\
     
      \cline{1-6}
      \cite{cryptoeprint:2021/580}& Asymptotic & & & 3 & $O(\lambda_\stat+\log t)$ \\

      \cline{1-6}
      \cite[Appendix B]{cryptoeprint:2018/579} & Empirical & $t\ge 4$ & \makecell{$\lambda_\stat = a_t\cdot e - b_t - \log t$\\$a_t = 123.5\cdot {\sf CDF_{Normal}}(x=t, \mu = 6.3, \sigma = 2.3)$\\$b_t = 120\cdot {\sf CDF_{Normal}}(x=t, \mu = 6.45, \sigma = 2.18)$} & 3\tnote{$\ddag$} & $e$ \\

      \cline{1-6}
      \makecell{\cite{cryptoeprint:2021/580}\\ simplifying the above} & Empirical & $t\ge 30$\tnote{*} & \makecell{$\lambda_\stat = 123.5 e -120 - \log t$} & 3 & $e$\\

      \cline{1-6}
      \multirow{2}{*}{\cite{chen_fast_2017}\tnote{**}} &\multirow{2}{*}{ Empirical }& $11041$ & $40\,(\lambda_\stat = 124.4 e - 144.6)$ & 3 &$m=2^{14},\,e\approx 1.5$\\
      \cline{3-6}
      & & $5535$ & $40\,(\lambda_\stat = 125 e - 145)$ & 3 &$m=2^{13}, \, e\approx 1.5$\\
      
      \bottomrule
    \end{tabular}	
    \begin{tablenotes}
      \item [$\dag$] $O(\sqrt{\lambda_\stat \log t})$ queries to the hash functions and supposes the hash functions from a $O(t\sqrt{\lambda_\stat \log t})$-wise independent hash function family. 
      \item [$\ddag$] Parameters are only slightly different for $w>3$. 
      \item [*] Should extend to smaller $t$ like $t = 16, 25$.
      \item [**]It first fixes $m = 2^{13}, 2^{14}$ and then computes the correlation between $\lambda_\stat$ and $e$.   
      \end{tablenotes}
  \end{threeparttable}
  %}
\end{table*}

%However we use cuckoo hashing to construct DMPF for $t$-point functions, in which case we'd also care about $t$ being small, say $2,3$ or $100$, and $m$ should not be too large. In this sense the previous empirical results are not complete. 

With the $t$ balls replicated and allocated to $m$ buckets, the cuckoo hashing algorithm essentially finds a perfect matching from $t$ balls to $m$ buckets, which coincides with the form of (probabilistic) CBC decoding. Therefore a PCBC follows directly from a cuckoo hashing scheme: \Yaxin{Dec 31: The following construction is mentioned in \cite{yeo_cuckoo_2023}. There are several points to note: 

(1) \cite{yeo_cuckoo_2023} modified the hash functions' domain in the following way: it divides the $m$ buckets evenly to $w$ blocks, and for $1\le i\le w$, $h_i:[N]\rightarrow [m/w]$ maps an element to a bucket in the $i$th block. The paper does this to claim better asymptotic provable success probability of cuckoo hashing, but using superconstant number ($w = \lambda_\stat/\log\log N$) of hash functions, which does not align with empirical results that suggests constant number (say 3) of hash functions. I think to us this means that if making $h_i$ to map to the $i$th block could be useful in implementation somehow (although I doubt this), then it also makes sense to do this modification. 

(2) It should be mentioned that both the capacity of cuckoo-hashing bins (which is 1 here) and the number of lookup in each $C_i$ that PCBC is allowed (also 1 here) can be simultaneously generalized to any number $l$ along with different parameters and overheads, but the paper still applied only $l=1$ case to applications like batch PIR, and I haven't seen any efficient empirical parameters and results for $l>1$ setting. However it is plausible to use general $l$ along with $O(t/l)$ buckets, each expanded to a $\DMPF_l$ truth table. It may be mentioned as a future direction in the end. }

\begin{construction}[PCBC from cuckoo hashing]
  Given $w$-way cuckoo hashing as a sub-procedure allocating $t$ balls to $m$ buckets with failure probability $\epsilon$, an $(N,wN,t,m,\epsilon)$-PCBC is as follows: 
  \begin{itemize}
    \item $\Encode_r(x\in\Sigma^N)\rightarrow (C_1,\cdots,C_m)$: Use $r$ to determine $w$ independent random hash functions $h_1,h_2,\cdots h_w$ that maps from $[N]$ to $[m]$.  Let $C_j$ be $\{x[i]:h_l(i) = j$ for some $l\in [w]\}$, in ascending order of $i$. 
    \item $\Decode_r(I, C_1,\cdots, C_m)\rightarrow \{x[i]\}_{i\in I}$: Determine $h_1,\cdots, h_w$ as in $\Encode$. For $I$ of size $t$, find a perfect matching from $I$ to $[m]$ using a $w$-way cuckoo hashing scheme. For each $i\in I$, fetch $x[i]$ from $C_j$ where $i$ and $j$ are matched in the perfect matching. Note that $x[i]$ can be found in the $k$th position of $C_j$ where $i$ is the $k$th smallest index of $\{i:h_l(i) = j$ for some $l\in [w]\}$. 
  \end{itemize}
\end{construction}
%The above PCBC scheme has its encoding and decoding process independent of the content of the input string $x$. 
An ambiguous point in $\Decode_r$ is how to find the index of $x[i]$ in $C_j$ it is mapped to. We  display two solutions to this index finding problem: 
\begin{enumerate}
  \item When $N$ is a feasible number, one can directly compute the entire hash tables derived by $h_1,\dots, h_w$ and compute the index of $x[i]$ in $C_j$. 
  \item One can implement $w$ hash functions by a single \emph{random permutation} $P$ mapping from $[w]\times [N]$ to $[m]\times [B]$, where $B = wN/m$. Invocation of $h_i(j)$ is done by computing $P(i,j)$, which outputs the bucket number in $[m]$ and the index in $[B]$. Note that in this case $h_1,\dots,h_w$ are not independent random hash functions, but as long as they \Yaxin{satisfy some sufficient independence property. To be clarified. }This solution is noted in \cite{cryptoeprint:2021/580} where $P$ is realized by a PRP. 
\end{enumerate}
\subsection{DMPF Construction from CBC}
We display the construction of DMPF from black-box usage of DPF basing on PCBC with appropriate parameters, which has been discussed in previous literature\cite{cryptoeprint:2019/273,cryptoeprint:2021/580}. As discussed before, we assume that the PCBC encoding and decoding are oblivious of the content of the input string.  
\begin{construction}[DMPF from DPF basing on PCBC]\label{con:DMPF_batch_code}
  Given DPF for any domain of size no larger than $N$ and output group $\GG$, and an $(N,M,t,m,\epsilon)$-PCBC with alphabet $\Sigma=\GG$, we can construct a DMPF scheme for $t$-point functions with domain size $N$ and output group $\GG$ as follows: 
  \begin{itemize}
    \item $\Gen(1^\lambda, \hat{f}_{A,B})\rightarrow (k_0,k_1)$: Suppose $A=\{\alpha_1,\cdots,\alpha_t\}$ and $B=\{\beta_1,\cdots,\beta_t\}$. Compute $\Encode([N])\rightarrow (C_1,\cdots,C_m)$ according to the PCBC. Then run $\Decode(A, C_1,\cdots, C_m)$ to determine a perfect matching from $A$ to $\{C_1,\cdots,C_m\}$. For $1\le i\le m$, let $f_i:[|C_i|]\rightarrow \GG$ be the following: 
    \begin{itemize}
      \item If $C_i$ is assigned none of $A$ by the perfect matching, then set $f_i$ to be the all-zero function. 
      \item If exactly one $\alpha_j$ of $A$ is assigned to the $l$th position of $C_i$, then set $f_i$ to be the point function that outputs $\beta_j$ on $l$ and 0 elsewhere. 
    \end{itemize}
    For $1\le i\le m$, invoke DPF$.\Gen(1^\lambda, f_i)\rightarrow (k_0^i,k_1^i)$. Set $(k_0,k_1)=(\{k_0^i\}_{i\in [m]}, \{k_1^i\}_{i\in [m]})$. If $\Decode$ fails then run $\Encode$ and $\Decode$ again with fresh randomness. 
    \item $\Eval_b(k_b,x)\rightarrow y_b$: Follow $\Encode([N])$ to determine the positions $l_{j_1},l_{j_2},\cdots, l_{j_s}$ such that the $x$ is sent to the $l_{j_i}$-th position of $C_{j_i}$ (since the allocation of indices is oblivious of the content of $TT$). Compute $y_b=\sum_{i=1}^s$DPF$.\Eval_b(k_b^{j_i},l_i)$. 
    \item $\FullEval_b(k_b)\rightarrow Y_b$: Compute $Y_b^i = $DPF$.\FullEval_b(k_b^i)$ for $1\le i\le m$. For each input $x\in [N]$, follow $\Encode([N]])$ to choose a position $l_x$ in bucket $C_{j_x}$ that $x$ is sent to. Let $Y_b[x]\gets Y_b^{j_x}[l_x]$. 
  \end{itemize}
The scheme is correct with overwhelming probability and has distinguish advantage $<2\epsilon$. 
\end{construction}
Note that if one use CBC instead of PCBC then the DMPF scheme is perfectly correct and secure. 

When instantiating PCBC using $w$-way cuckoo hashing, the \emph{key generation time} is roughly the time for computing cuckoo hashing algorithm, the time for finding $t$ indices for $t$ elements in the buckets, plus the total time of all DPF$.\Gen(1^\lambda, f_i)$. The \emph{evaluation time} is roughly the time for finding $w$ indices for one element in the buckets plus the total time of all DPF$.\Eval_b(k_b^{j_i},l_i)$. Similarly, the \emph{full-domain evaluation time} is roughly the time for finding $N$ indices for $N$ elements in the buckets plus the total time of all DPF$.\FullEval_b(k_b^{j})$ for $j=1,\dots,m$. 

\subsection{Oblivious Key-Value Stores}\label{sec:prelim_okvs}
We introduce the notion of Oblivious key-value stores (OKVS) which can be used to construct DMPF. OKVS was originally proposed as a primitive for private set intersection (PSI) protocols (see \cite{cryptoeprint:2021/883,cryptoeprint:2022/320}). 
\begin{definition}[Oblivious Key-Value Stores (OKVS)\cite{cryptoeprint:2021/883,cryptoeprint:2022/320}]\label{def:OKVS}
  An Oblivious Key-Value Stores scheme is a pair of randomized algorithms $(\Encode_r,\Decode_r)$ with respect to a statistical security parameter $\lambda_{\sf stat}$ and a computational security parameter $\lambda$, a randomness space $\{0,1\}^\kappa$, a key space $\mathcal{K}$, a value space $\mathcal{V}$, input length $t$ and output length $m$. The algorithms are of the following syntax: 
  \begin{itemize}
    \item $\Encode_r(\{(k_1,v_1),(k_2,v_2),\cdots,(k_t,v_t)\})\rightarrow P$: On input $t$ key-value pairs with distinct keys, the encode algorithm with randomness $r$ in the randomness space outputs an encoding $P\in\mathcal{V}^m\cup\bot$.
    \item $\Decode_r(P,k)\rightarrow v$: On input an encoding from $\mathcal{V}^m$ and a key $k\in\mathcal{K}$, output a value $v$. 
  \end{itemize}
  We require the scheme to satisfy
  \begin{itemize}
    \item For all $S\in(\mathcal{K}\times\mathcal{V})^t$, $\Pr_{r\leftarrow\{0,1\}^\kappa}[\Encode_r(S)=\bot]\le 2^{-\lambda_{\sf stat}}$. 
    \item For all $S\in(\mathcal{K}\times \mathcal{V})^t$ and $r\in \{0,1\}^\kappa$ such that $\Encode_r(S)\rightarrow P\not=\bot$, it is the case that $\Decode_r(P,k)\rightarrow v$ whenever $(k,v)\in S$. 
    \item \textbf{Obliviousness: }Given any distinct key sets $\{k_1^0,k_2^0,\cdots,k_t^0\}$ and $\{k_1^1,k_2^1,\cdots,k_t^1\}$ that are different, if they are paired with random values then their encodings are computationally indistinguishable, i.e., 
  \begin{align*}
    &\{r, \Encode_r(\{(k_1^0,v_1),\cdots,(k_t^0,v_t)\})\}_{v_1,\cdots,v_t\leftarrow \mathcal{V},r\leftarrow\{0,1\}^\kappa}\\
    \approx_c &\{r, \Encode_r(\{(k_1^1,v_1),\cdots,(k_t^1,v_t)\})\}_{v_1,\cdots,v_t\leftarrow \mathcal{V},r\leftarrow\{0,1\}^\kappa}
  \end{align*}
  \end{itemize}
One can obtain a \emph{linear OKVS} if in addition require:
\begin{itemize}
  \item \textbf{Linearity: }There exists a function family $\{\row_r:\mathcal{K}\rightarrow\mathcal{V}^m\}_{r\in\{0,1\}^\kappa}$ such that $\Decode_r(P,k) = \ipd{\row_r(k)}{P}$. 
\end{itemize}
\end{definition}
The $\Encode$ process for a linear OKVS is the process of sampling a random $P$ from the set of solutions of the linear system $\{\ipd{\row_r(k_i)}{P} = v_i\}_{1\le i\le t}$. 

We evaluate an OKVS scheme by its rate ($\frac{\text{input length }t}{\text{output length }m}$), encoding time and decoding time. 

The most na\"ive OKVS construction is encoding $S = \{(k_i, v_i)\}_{1\le i\le t}$ to a random truth table $TT:\mathcal{K}\rightarrow \mathcal{V}$ such that $TT(k_i) = v_i$ for all $1\le i\le t$. Note that to ensure obliviousness, for $k$ not appearing in $S$, the encoding should set $TT(k)$ to a random value. However this na\"ive construction is very inefficient since it requires the encoding size to be $m=|\mathcal{K}|$, and hence its rate $\frac{t}{|\mathcal{K}|}$ can be tiny. 

A well-known, optimal-rate OKVS construction is encoding $t$ key-value pairs using a deg-$t$ polynomial: 
\begin{construction}[Polynomial]\label{con:OKVS_polynomial}
  Suppose $\mathcal{K} = \mathcal{V}=\FF$ is a field. Set 
  \begin{itemize}
    \item $\Encode(\{(k_i,v_i)\}_{1\le i\le t}) \rightarrow P$ where $P$ is the coeffients of a $(t-1)$-degree $\FF$-polynomial $g_P$ that $g_P(k_i) = v_i$ for $1\le i\le t$. 
    \item $\Decode(P,k)\rightarrow g_P(k)$. 
  \end{itemize}
\end{construction}
The polynomial OKVS possesses an optimal encoding size $m=n$, but the $\Encode$ process is a polynomial interpolation which is only known to be achieved in time $O(t\log^2t)$. The time for a single decoding is $O(t)$ and that for batched decodings is (amortized) $O(\log^2 t)$. 

In the sequel we stress two alternative (linear) OKVS constructions that has near optimal encoding size but much better running time. 

\begin{construction}[RR22\cite{cryptoeprint:2021/883,cryptoeprint:2022/320}]\label{con:OKVS_sparse_matrix}
  Suppose $\mathcal{V}=\FF$ is a field. Set $\row_r(k):=\row_r^{\sf sparse}(k)||\row_r^{\sf dense}(k)$ where $\row_r^{\sf sparse}(k)$ outputs a uniformly random weight-$w$ vector in $\{0,1\}^{m_1}$, and $\row_r^{\sf dense}(k)$ outputs a short dense vector in $\FF^{m_2}$. 
  \begin{itemize}
    \item $\Encode_r(\{(k_i,v_i)\}_{1\le i\le t}) \rightarrow P$ where $P$ is randomly chosen from the solutions of the system $\{\ipd{\row_r(k_i)}{P} = v_i\}_{1\le i\le t}$, solved by the triangulation algorithm in \cite{cryptoeprint:2022/320}. If the system has no solution then output $\bot$. 
    \item $\Decode_r(P,k)\rightarrow \ipd{\row_r(k)}{P}$. 
  \end{itemize}
  We denote $m_1=et$, where $e$ is an expansion parameter indicating the rough blowup to store $t$ pairs. In practice the number of dense columns $m_2$ is usually set to a small constant. 
\end{construction}
This OKVS construction features an efficient encoding process, constant decoding time ($(w+m_2)$ additions and $m_2$ multiplications in $\FF$) while having a linear encoding size. 

$\Encode$ may output $\bot$ if the matrix formed by $\{\row_r(k_i)\}_{1\le i\le t}$ is not full-rank. Therefore we need to adjust the parameters $m_1=et$ and $m_2$ to ensure negligible error probability (represented by the statistical security parameter $\lambda_\stat$). The expansion parameter $e$ and the number of dense columns $m_2:=\hat{g}$ (where $\hat{g}$ is a parameter relating to the equation system solving process) are given by the analysis in \cite{cryptoeprint:2022/320}, with the range of $N$ from $2^6$ to $2^{18}$: Given $w$, $t$ and $\lambda_\stat$, the choices of the $e$ and $\hat{g}$ are fixed through the following steps: 
\begin{itemize}
  \item Set $e^* = \begin{cases}
    1.223 & w=3\\
    1.293 & w=4\\
    0.1485w+0.6845 & w\ge 5
  \end{cases}$.
  \item Compute $\alpha:=0.55 \log_2 t + 0.093w^3-1.01w^2 + 2.92w-0.13$.
  \item $e:=e^*+ 2^{-\alpha}(\lambda_\stat+9.2)$. 
  \item $\hat{g}:=\frac{\lambda_\stat}{(w-2)\log_2(et)}$. 
\end{itemize}


\Yaxin{Fix $t$ and $\lambda_\stat$, we want to find the best choice of $w$. The adavantageous choices of $w$ in \cite{cryptoeprint:2022/320} are $w=3$ and $w=5$. From the first sight when $w$ is smaller $e$ can be smaller but $\hat{g}$ will be larger. Since $w+\hat{g}$ stands for number of $\FF$-ADD's and $\hat{g}$ stands for number of $\FF$-MULT's in decoding, previously I thought $\hat{g}$ is the dominating factor of $\Decode$ running time. However table 1 in \cite{cryptoeprint:2022/320} suggests that $w=3$ outruns nearly all of other choices of $w$ while $w=5$ is almost 3 times slower in decoding time. This may suggest there are some other heavy computations other than $\FF$-MULT that need to be considered when evaluating running time. 

The range of $t$ previous literature \cite{cryptoeprint:2021/883,cryptoeprint:2022/320} have considered in their empirical results are also limited, which will be one of our problems. We want to cover small $t$, say $t<100$, while previous literature aiming for constructing PSI protocols usually consider very large $t$. }

One may let $row_r^{\sf dense}$ output a short dense vector in $\{0,1\}^{m_2}$ to avoid multiplication of large field elements in the encoding and decoding processes. To achieve same level of security one could simply set $m_2=\hat{g}+\lambda_\stat$, as proposed in \cite{cryptoeprint:2021/883,cryptoeprint:2022/320}. As indicated by the empirical results in \cite{cryptoeprint:2022/320}, this binary scheme is usually not as efficient as the original design. Therefore we mostly refer to \cref{con:OKVS_sparse_matrix}. 

\begin{construction}[RB-OKVS\cite{cryptoeprint:2023/903}]\label{con:OKVS_ribbon}
  Suppose $\mathcal{V} = \GG$ is a group. Let $\row_r(k)$ output a $\{0,1\}^m$ vector consisting of a width-$w$ random band. Formally speaking, $\row_r(k)$ first determine a starting point $1\le i\le m-w+1$ for the band, and then determine random $w$-bit string to fill in the positions $[i,i+w-1]$ of $\row_r(k)$ and leave the rest as 0 entries. 
  \begin{itemize}
    \item $\Encode_r(\{(k_i,v_i)\}_{1\le i\le t})\rightarrow P$ where $P$ is randomly chosen from the random band matrix system $\{\ipd{\row_r(k_i)}{P}=v_i\}_{1\le i\le t}$. If the system has no solution then output $\bot$. 
    \item $\Decode_r(P,k)\rightarrow \ipd{\row_r(k)}{P}$. 
  \end{itemize}
  Denote $m=et$ where $e>1$ is an expansion parameter indicating the blowup to store $t$ pairs. 
\end{construction}

The encoding time is equivalent to solving a random band matrix system, which can be efficiently done in $O(Nw+n\log n)$ time \cite{cryptoeprint:2023/903}. The decoding time is $w$ additions in $\FF$ and the rate can be very close to 1. 

Again, to guarantee the success of $\Encode$, the random band matrix must be full-rank with overwhelming probability. According to \cite{cryptoeprint:2023/903}, fixing $e>1$ and taking $w = O(\lambda_\stat/(e-1)+\log N)$ ensures the correctness and obliviousness with probability $2^{-\lambda_\stat}$ and $2^{-w}$, respectively. Practically, $e=1.03,1.05,1.07,1.1$ are taken while $w$ being several hundred to reach the security $\lambda_\stat=40$, with the choice of $N$ varying from $2^{10}$ to $2^{20}$. 

According to the comparison in \cite{cryptoeprint:2023/903} of the RR22-OKVS (\cref{con:OKVS_sparse_matrix}) and the RB-OKVS (\cref{con:OKVS_ribbon}) with the choices of $N = 2^{16}, 2^{20}, 2^{24}$, the RB-OKVS has a better rate and features a tradeoff between rate and encoding/decoding time (one can choose to have better rate with longer encoding/decoding time). The RB-OVS has better encoding time while the RR22-OKVS has better decoding time. 

\Yaxin{Maybe (and how to) put a (quantitative) summarizing table of OKVS efficiency here?}

\medskip

In our later sections, we will give the decoding efficiency of the OKVS the most priority. To this end, we refer to the RR22-OKVS (\cref{con:OKVS_sparse_matrix}) when instantiating OKVS. One may switch to other OKVS constructions depending on different needs in practice. 
