\documentclass[sigconf]{acmart}
\AtBeginDocument{%
  \providecommand\BibTeX{{%
    Bib\TeX}}}

%% Rights management information.  This information is sent to you
%% when you complete the rights form.  These commands have SAMPLE
%% values in them; it is your responsibility as an author to replace
%% the commands and values with those provided to you when you
%% complete the rights form.
\setcopyright{acmcopyright}
\copyrightyear{tbd}
\acmYear{tbd}
\acmDOI{tbd}

%% These commands are for a PROCEEDINGS abstract or paper.
\acmConference[Conference acronym 'XX]{Make sure to enter the correct conference title from your rights confirmation emai}{tbd}{tbd}
%%
%%  Uncomment \acmBooktitle if the title of the proceedings is different
%%  from ``Proceedings of ...''!
%%
%%\acmBooktitle{Woodstock '18: ACM Symposium on Neural Gaze Detection,
%%  June 03--05, 2018, Woodstock, NY}
\acmPrice{15.00}
\acmISBN{tbd}


\usepackage{amsmath,amsfonts}
\usepackage{amsthm}
\usepackage{algorithm}
\usepackage{algpseudocode}
\usepackage{cleveref}
    
%% New commands goes here
\newcommand{\Enote}[1]{\color{purple}Enote: #1\color{black}}
\newcommand{\Yaxin}[1]{\color{purple}Yaxin: #1\color{black}}

\newcommand{\Gen}{{\sf Gen}}
\newcommand{\Eval}{{\sf Eval}}
\newcommand{\FullEval}{{\sf FullEval}}
\newcommand{\Encode}{{\sf Encode}}
\newcommand{\Decode}{{\sf Decode}}
\newcommand{\row}{{\sf row}}

\newcommand{\Adv}{{\sf Adv}}
    
\newcommand{\GG}{\mathbb{G}}
\newcommand{\NN}{\mathbb{N}}
\newcommand{\ZZ}{\mathbb{Z}}
\newcommand{\FF}{\mathbb{F}}

\newcommand{\ipd}[2]{\langle #1, #2 \rangle}

\newtheorem{theorem}{Theorem}
\newtheorem{lemma}[theorem]{Lemma}
\newtheorem{claim}[theorem]{Claim}
\newtheorem{fact}[theorem]{Fact}
\newtheorem{definition}[theorem]{Definition}
\newtheorem{example}[theorem]{Example}
\newtheorem{remark}[theorem]{Remark}
\newtheorem{construction}{Construction}
\newtheorem{corollary}[theorem]{Corollary}
\newtheorem{proposition}[theorem]{Proposition}
%%
\begin{document}

%%
%% The "title" command has an optional parameter,
%% allowing the author to define a "short title" to be used in page headers.
\title{The Name of the Title Is Hope}

\author{tbd}
%\authornote{}
%\email{tbd}
%\affiliation{%
  %\institution{tbd}
  %\streetaddress{P.O. Box 1212}
  %\city{tbd}
  %\state{tbd}
  %\country{tbd}
  %\postcode{tbd}
%}

%%
%% By default, the full list of authors will be used in the page
%% headers. Often, this list is too long, and will overlap
%% other information printed in the page headers. This command allows
%% the author to define a more concise list
%% of authors' names for this purpose.
\renewcommand{\shortauthors}{tbd}

\begin{abstract}
  tbd. 
\end{abstract}

%%
%% The code below is generated by the tool at http://dl.acm.org/ccs.cfm.
\begin{CCSXML}
  <ccs2012>
     <concept>
         <concept_id>10003752.10003777.10003788</concept_id>
         <concept_desc>Theory of computation~Cryptographic primitives</concept_desc>
         <concept_significance>500</concept_significance>
         </concept>
   </ccs2012>
\end{CCSXML}
  
\ccsdesc[500]{Theory of computation~Cryptographic primitives}

%%
%% Keywords. The author(s) should pick words that accurately describe
%% the work being presented. Separate the keywords with commas.
\keywords{tbd}

%\received{20 February 2007}
%\received[revised]{12 March 2009}
%\received[accepted]{5 June 2009}

\maketitle

\section{Introduction}
tbd

\section{Preliminary}
\subsection{Basic Notations}



 \paragraph{Point and multi-point functions.} Given a domain size $N$ and Abelian group $\GG$, a \emph{point function} $f_{\alpha,\beta}:[N]\rightarrow\GG$ for $\alpha\in[N]$ and $\beta\in\GG$ evaluates to $\beta$ on input $\alpha$ and to $0\in\GG$ on all other inputs. We denote by $\hat{f}_{\alpha,\beta}=(N,\hat{\GG},\alpha,\beta)$ the representation of such a point function. A \emph{multi-point function} $f_{A,B}:[N]\rightarrow \GG$ for $A=(\alpha_1,\cdots\alpha_t)\in[N]^t$ and $B=(\beta_1,\cdots,\beta_t)\in \GG^t$ evaluates to $\beta_i$ on input $\alpha_i$ for $1\le i\le t$ and to $0$ on all other inputs. Denote $\hat{f}_{A,B}(N,\hat{\GG},A,B)$ the representation of such a point function. 
 
\Enote{MPF. Also representation of groups.}
 
\subsection{Distributed Multi-Point Functions}

\Enote{should directly adapt to multi-point function case}

We begin by defining a slightly generalized notion of distributed point functions (DPFs), which accounts for the extra parameter $\GG'$.

%\yuval{Made some small stylistic changes below (similar changes may apply elsewhere). Should citations to GI14,BGI16. }
\begin{definition}[DPF \cite{EC:GilIsh14,CCS:BoyGilIsh16}]\label{def:dpf}
A 
%$t$-private $m$-server 
(2-party)
\emph{distributed point function (DPF)}
%, or $(m,t)$-DPF for short, 
is a triple of algorithms %$\Pi=(\Gen,\Eval_0,\ldots,\Eval_{m-1})$ 
$\Pi=(\Gen,\Eval_0,\Eval_1)$
with the following syntax: 
\begin{itemize}
    \item $\Gen(1^\lambda,\hat{f}_{\alpha,\beta})\rightarrow (k_0,k_1)$: On input security parameter $\lambda\in\NN$ and point function description $\hat{f}_{\alpha,\beta}=(N,\hat{\GG},\alpha,\beta)$, the (randomized) key generation algorithm $\Gen$ returns a pair of keys $k_0,k_1\in\{0,1\}^*$. We assume that $N$ and $\GG$ are determined by each key.
    \item $\Eval_i(k_i,x)\rightarrow y_i$: On input key $k_i\in\{0,1\}^*$ and input $x\in[N]$ the (deterministic) evaluation algorithm of server $i$, $\Eval_i$ returns 
    %a group element 
    $y_i\in\GG$.
\end{itemize}
%The algorithms $\Pi=(\Gen,\Eval_0,\ldots,\Eval_{m-1})$ should 
We require $\Pi$ to satisfy the following requirements:
\begin{itemize}
    \item \textbf{Correctness:} For every $\lambda$, $\hat{f}=\hat{f}_{\alpha,\beta}=(N,\hat{\GG},\alpha,\beta)$ such that $\beta\in\GG$, and $x\in[N]$, if $(k_0,k_1)\leftarrow\Gen(1^\lambda,\hat{f})$, then $$\Pr\left[\sum_{i=0}^{1}\Eval_i(k_i,x)=f_{\alpha,\beta}(x)\right]=1$$
    \item \textbf{Security:} Consider the following semantic security challenge experiment for corrupted server $i\in\{0,1\}$:
    \begin{enumerate}
        \item The adversary produces two point function descriptions $(\hat{f}^0=(N,\hat\GG,\alpha_0,\beta_0),\hat{f}^1=(N,\hat\GG,\alpha_1,\beta_1))\leftarrow\mathcal{A}(1^\lambda)$, where $\alpha_i\in[N]$ and $\beta_i\in\GG$.
        \item The challenger samples $b\gets\{0,1\}$ and $(k_0,k_1)\leftarrow\Gen(1^\lambda,\hat{f}^b)$.
        \item The adversary outputs a guess $b'\leftarrow\mathcal{A}(k_i)$.
    \end{enumerate}
    Denote by $\Adv(1^\lambda,\mathcal{A},i)=\Pr[b=b']-1/2$ the advantage of $\mathcal{A}$ in guessing $b$ in the above experiment. For every non-uniform polynomial time adversary $\mathcal{A}$ there exists a negligible function $\nu$ such that $\Adv(1^\lambda,\mathcal{A},i) \le \nu(\lambda)$ for all $\lambda \in \NN$.
%    For circuit size bound $S=S(\lambda)$ and advantage bound $\epsilon(\lambda)$, we say that $\Pi$ is $(S,\epsilon)$-secure if for all $i\in\{0,1\}$ and all non-uniform adversaries $\mathcal{A}$ of size $S(\lambda)$ and sufficiently large $\lambda$, we have $\Adv(1^\lambda,\mathcal{A},i)\leq\epsilon(\lambda)$. We say that $\Pi$ is:
%    \begin{itemize}
%        \item \emph{Computationally $\epsilon$-secure} if it is $(S,\epsilon)$-secure for all polynomials $S$.
%        \item \emph{Computationally secure} if it is $(S,1/S)$-secure for all polynomials $S$.
%        %\item \emph{Statistically $\epsilon$-secure} if it is $(S,\epsilon)$-secure for all $S$.
%        %\item \emph{Perfectly secure} if it is statistically $0$-secure.
%    \end{itemize}
\end{itemize}
%If the security threshold $t$ is unspecified, we assume it is $t=1$.
\end{definition}
 
 We will also be interested in applying the evaluation algorithm on \emph{all} inputs. Given a DPF $(\Gen,\Eval_0,\Eval_1)$, we denote by $\FullEval_i$ an algorithm which computes $\Eval_i$ on every input $x$. Hence, $\FullEval_i$ receives only a key $k_i$ as input.



\subsection{Batch Codes}
combinatorial/probabilistic batch codes, with cuckoo hashing a concrete instantiation

\subsection{Oblivious Key-Value Stores}
\begin{definition}[OKVS\cite{cryptoeprint:2021/883,cryptoeprint:2022/320}]
  An Oblivious Key-Value Stores (OKVS) scheme is a pair of randomized algorithms $(\Encode_r,\Decode_r)$ with respect to a statistical security parameter $\lambda_{\sf stat}$ and a computational security parameter $\lambda$, a randomness space $\{0,1\}^\kappa$, a key space $\mathcal{K}$, a value space $\mathcal{V}$, input length $n$ and output length $m$. The algorithms are of the following syntax: 
  \begin{itemize}
    \item $\Encode_r(\{(k_1,v_1),(k_2,v_2),\cdots,(k_n,v_n)\})\rightarrow P$: On input $n$ key-value pairs with distinct keys, the encode algorithm with randomness $r$ in the randomness space outputs an encoding $P\in\mathcal{V}^m\cup\bot$.
    \item $\Decode_r(P,k)\rightarrow v$: On input a (nonempty) encoding from $\mathcal{V}^m$ and a key $k\in\mathcal{K}$, output a value $v$. 
  \end{itemize}
  We require the scheme to satisfy
  \begin{itemize}
    \item \textbf{Correctness: }For every $S\in(\mathcal{K}\times\mathcal{V})^n$, $\Pr_{r\leftarrow\{0,1\}^\kappa}[\Encode_r(S)=\bot]\le 2^{-\lambda_{\sf stat}}$. 
    \item \textbf{Obliviousness: }For any distinct key sets $\{k_1^0,k_2^0,\cdots,k_n^0\}$ and $\{k_1^1,k_2^1,\cdots,k_n^1\}$ that are different, if they are paired with random values then their encodings are computationally indistinguishable, i.e., 
  \begin{align*}
    &\{r, \Encode_r(\{(k_1^0,v_1),\cdots,(k_n^0,v_n)\})\}_{v_1,\cdots,v_n\leftarrow \mathcal{V},r\leftarrow\{0,1\}^\kappa}\\
    \approx_c &\{r, \Encode_r(\{(k_1^1,v_1),\cdots,(k_n^1,v_n)\})\}_{v_1,\cdots,v_n\leftarrow \mathcal{V},r\leftarrow\{0,1\}^\kappa}
  \end{align*}
  \end{itemize}
One can obtain a \emph{linear OKVS} if in addition require:
\begin{itemize}
  \item \textbf{Linearity: }There exists a function family $\{\row_r:\mathcal{K}\rightarrow\mathcal{V}^m\}_{r\in\{0,1\}^\kappa}$ such that $\Decode_r(P,k) = \ipd{\row_r(k)}{P}$. 
\end{itemize}
\end{definition}
The $\Encode$ process for a linear OKVS is the process of sampling a random $P$ from the set of solutions of the linear system $\{\ipd{\row_r(k_i)}{P} = v_i\}_{1\le i\le n}$. 

We evaluate an OKVS scheme by its encoding size (output length $m$), encoding time and decoding time. We stress the following two (linear) OKVS constructions:
\begin{construction}[Polynomial]
  Suppose $\mathcal{K} = \mathcal{V}=\FF$ is a field. Set 
  \begin{itemize}
    \item $\Encode(\{(k_i,v_i)\}_{1\le i\le n}) \rightarrow P$ where $P$ is the coeffients of a $(n-1)$-degree $\FF$-polynomial $g_P$ that $g_P(k_i) = v_i$ for $1\le i\le n$. 
    \item $\Decode(P,k)\rightarrow g_P(k)$. 
  \end{itemize}
\end{construction}
The polynomial OKVS possesses optimal encoding size, but the $\Encode$ process is a polynomial interpolation which is only known to be achieved in super linear time. 
\begin{construction}[3H-GCT\cite{cryptoeprint:2021/883,cryptoeprint:2022/320}]
  Suppose $\mathcal{V}=\FF$ is a field. Set $\row_r(k):=\row_r^{\sf sparse}(k)||\row_r^{\sf dense}(k)$ where $\row_r^{\sf sparse}$ outputs a uniformly random vector in $\{0,1\}^{m_1}$ of hamming weight 3, and $\row_r^{\sf dense}(k)$ outputs a short dense vector in $\FF^{m_2}$. 
  \begin{itemize}
    \item $\Encode(\{(k_i,v_i)\}_{1\le i\le n}) \rightarrow P$ where $P$ is solved from the system $\{\ipd{\row_r(k_i)}{P} = v_i\}_{1\le i\le n}$ using the triangulation algorithm in \cite{cryptoeprint:2022/320}. 
    \item $\Decode(P,k)\rightarrow \ipd{\row_r(k)}{P}$. 
  \end{itemize}
\end{construction}

TBD: mention some connections to cuckoo hashing






\section{New DMPF constructions}

\subsection{Big-State DMPF}
display the big-state DMPF (plus distributed gen)
%We display the big-state DMPF scheme in figure \ref{fig:big-state}. 
\iffalse
\begin{figure}\label{fig:DMPF_paradigm}
  \fbox{\parbox{\linewidth}{
  \begin{algorithmic}
  \Procedure{Gen}{$1^\lambda, A$}
  \State $t\leftarrow |A|, n\gets |A[1]|$.
  \State For $0\le i\le n-1$, let $A^{(i)}$ be the sorted and deduplicated list of $i$-bit prefixes of strings in $A$. Specifically, $A^{(0)} = [\epsilon]$. 
  \State Let $G:\{0,1\}^\lambda\rightarrow \{0,1\}^{2\lambda+2t}$ be a public PRG. 
  \State Set $S_b^{(0)} = [r_b]$ and $T_b^{(0)} = [b||0^{m-1}]$ for $b = 0,1$ where $r_0,r_1$ are sampled independently and randomly from $\{0,1\}^\lambda$.
  \For{$i=1$ to $n$}
  \State Let $CW^{(i)}, S_0^{(i)}, T_0^{(i)}, S_1^{(i)}, T_1^{(i)}$ be empty lists. 
  \For{$l = 1$ to $A^{(i-1)}$}
  \State Parse $G(S_b^{(i-1)}[l]) = s_b^L||t_b^L||s_b^R||t_b^R$ for $b = 0,1$ where $s_b^L,s_b^R\in\{0,1\}^\lambda$ and $t_b^L,t_b^R\in\{0,1\}^t$.
  \If{$A^{(i-1)}[l]||0\in A^{(i)}$ and $A^{(i-1)}[l]||1\not\in A^{(i)}$}
  \State $d\gets$ the index of $A^{(i-1)}[l]||0$ in $A^{(i)}$. 
  \State Append $s_0^R\oplus s_1^R||t_0^L\oplus t_1^L\oplus e_d||t_0^R\oplus t_1^R$ to $CW^{(i)}$ where $e_d = 0^{d-1}10^{t-d}$.
  \ElsIf{$A^{(i-1)}[l]||1\in A^{(i)}$ and $A^{(i-1)}[l]||0\not\in A^{(i)}$}
  \State $d\gets$ the index of $A^{(i-1)}[l]||1$ in $A^{(i)}$. 
  \State Append $s_0^L\oplus s_1^L||t_0^L\oplus t_1^L||t_0^R\oplus t_1^R\oplus e_d$ to $CW^{(i)}$.
  \Else\Comment{both $A^{(i-1)}[l]||0$ and $A^{(i-1)}[l]||1$ are in $A^{(i)}$.}
  \State $d\gets$ the index of $A^{(i-1)}[l]||0$ in $A^{(i)}$. 
  \State Randomly sample $r$ from $\{0,1\}^\lambda$.
  \State Append $r||t_0^L\oplus t_1^L\oplus e_d||t_0^R\oplus t_1^R\oplus e_{d+1}$ to $CW^{(i)}$.
  \EndIf
  \EndFor
  \State Randomly and independently sample $t-|CW^{(i)}|$ strings from $\{0,1\}^{\lambda+2t}$ .
  \State If $i=n$ then skip the following for-loop. 
  \For{$l = 1$ to $|A^{(i-1)}|$}
  \State Parse $G(S_b^{(i-1)}[l]) = s_b^L||t_b^L||s_b^R||t_b^R$ for $b = 0,1$.
  \State Parse $T_b^{(i-1)}[l]\cdot CW^{(i)} = \Delta s_b||\Delta t_b^L||\Delta t_b^R$ for $b = 0,1$.
  \If{$A^{(i-1)}[l]||0\in A^{(i)}$ and $A^{(i-1)}[l]||1\not\in A^{(i)}$}
  \State Append $s_b^L\oplus \Delta s_b$ to $S_b^{(i)}$ and $t_b^L\oplus \Delta t_b^L$ to $T_b^{(i)}$, for $b = 0,1$.
  \ElsIf{$A^{(i-1)}[l]||1\in A^{(i)}$ and $A^{(i-1)}[l]||0\not\in A^{(i)}$}
  \State Append $s_b^R\oplus \Delta s_b$ to $S_b^{(i)}$ and $t_b^R\oplus \Delta t_b^R$ to $T_b^{(i)}$, for $b = 0,1$.
  \Else
  \State Append $s_b^L\oplus \Delta s_b$ to $S_b^{(i)}$ and $t_b^L\oplus \Delta t_b^L$ to $T_b^{(i)}$, for $b = 0,1$.
  \State Append $s_b^R\oplus \Delta s_b$ to $S_b^{(i)}$ and $t_b^R\oplus \Delta t_b^R$ to $T_b^{(i)}$, for $b = 0,1$.
  \EndIf
  \EndFor
  \EndFor
  \For{$l=1$ to $t$}\Comment{convert layer}
  \State Append $(-1)^{T_0^{(n)}[l][l]}\cdot\big(G_{convert}(S_0^{(n)}[l])-G_{convert}(S_1^{(n)}[l])-B[l]\big)$ to $CW^{(n+1)}$. 
  \EndFor
  \State Set $k_b \gets (S_b^{(0)}, CW^{(1)},CW^{(2)},\cdots,CW^{(n)})$.
  \State \textbf{return} $(k_0,k_1)$.
  \EndProcedure
  \Procedure{Eval}{$1^\lambda, b,k_b,x$}
  \State Parse $k_b = ([s],CW^{(1)},CW^{(2)},\cdots,CW^{(n)})$. 
  \State $t\leftarrow$ number of rows of any $CW^{(i)}$.
  \State $c\gets b||0^{t-1}$.
  \For{$i = 1$ to $n$}
  \State Parse $c\cdot CW^{(i)} = \Delta s||\Delta t^0||\Delta t^1$ where $\Delta s\in\{0,1\}^\lambda$ and $\Delta t^0,\Delta t^1\in \{0,1\}^t$. 
  \State Parse $G(s) = s^0||t^0||s^1||t^1$. 
  \State $s||c\gets (s^{x[i]}||t^{x[i]})\oplus (\Delta s||\Delta t^{x[i]})$
  \EndFor
  \State \textbf{return} $s||\bigoplus_{j = 1}^t c[j]$
  \EndProcedure
  \end{algorithmic}
  }}
  \caption{The big-state DMPF scheme}
\end{figure}
\fi

\subsection{Batch-Code DMPF}
display the batch-code DMPF 

\subsection{OKVS-based DMPF}
display the OKVS-based DMPF (plus distributed gen)

\subsection{Comparison}
Comparison table dependent to PRG \& F-MUL(list the formulas?)\\
analyze tradeoff\\
distributed gen advantage

\section{Applications}
\subsection{PCG for OLE from Ring-LPN}
Characterize parameters\\
show nonregular optimization\\
plug in new DMPF and show overall optimization
\subsection{PSI-WCA}
plug in new DMPF and analyze advantage interval\\
plug in distributed gen
\subsection{Heavy-hitters}
private heavy-hitter\\
or parallel ORAM?
\section{Acknowledgments}
tbd


\bibliographystyle{ACM-Reference-Format}
\bibliography{references}


%%
%% If your work has an appendix, this is the place to put it.
\appendix
\section{Batch-code DMPF scheme}
\section{Security Proofs}

\end{document}
\endinput
%%
%% End of file `sample-sigconf.tex'.
