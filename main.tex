%%
%% This is file `sample-sigconf.tex',
%% generated with the docstrip utility.
%%
%% The original source files were:
%%
%% samples.dtx  (with options: `sigconf')
%% 
%% IMPORTANT NOTICE:
%% 
%% For the copyright see the source file.
%% 
%% Any modified versions of this file must be renamed
%% with new filenames distinct from sample-sigconf.tex.
%% 
%% For distribution of the original source see the terms
%% for copying and modification in the file samples.dtx.
%% 
%% This generated file may be distributed as long as the
%% original source files, as listed above, are part of the
%% same distribution. (The sources need not necessarily be
%% in the same archive or directory.)
%%
%%
%% Commands for TeXCount
%TC:macro \cite [option:text,text]
%TC:macro \citep [option:text,text]
%TC:macro \citet [option:text,text]
%TC:envir table 0 1
%TC:envir table* 0 1
%TC:envir tabular [ignore] word
%TC:envir displaymath 0 word
%TC:envir math 0 word
%TC:envir comment 0 0
%%
%%
%% The first command in your LaTeX source must be the \documentclass
%% command.
%%
%% For submission and review of your manuscript please change the
%% command to \documentclass[manuscript, screen, review]{acmart}.
%%
%% When submitting camera ready or to TAPS, please change the command
%% to \documentclass[sigconf]{acmart} or whichever template is required
%% for your publication.
%%
%%
\documentclass[sigconf]{acmart}

%%
%% \BibTeX command to typeset BibTeX logo in the docs
\AtBeginDocument{%
  \providecommand\BibTeX{{%
    Bib\TeX}}}

%% Rights management information.  This information is sent to you
%% when you complete the rights form.  These commands have SAMPLE
%% values in them; it is your responsibility as an author to replace
%% the commands and values with those provided to you when you
%% complete the rights form.
\setcopyright{acmcopyright}
\copyrightyear{tbd}
\acmYear{tbd}
\acmDOI{tbd}

%% These commands are for a PROCEEDINGS abstract or paper.
\acmConference[Conference acronym 'XX]{Make sure to enter the correct conference title from your rights confirmation emai}{tbd}{tbd}
%%
%%  Uncomment \acmBooktitle if the title of the proceedings is different
%%  from ``Proceedings of ...''!
%%
%%\acmBooktitle{Woodstock '18: ACM Symposium on Neural Gaze Detection,
%%  June 03--05, 2018, Woodstock, NY}
\acmPrice{15.00}
\acmISBN{tbd}


%%
%% Submission ID.
%% Use this when submitting an article to a sponsored event. You'll
%% receive a unique submission ID from the organizers
%% of the event, and this ID should be used as the parameter to this command.
%%\acmSubmissionID{123-A56-BU3}

%%
%% For managing citations, it is recommended to use bibliography
%% files in BibTeX format.
%%
%% You can then either use BibTeX with the ACM-Reference-Format style,
%% or BibLaTeX with the acmnumeric or acmauthoryear sytles, that include
%% support for advanced citation of software artefact from the
%% biblatex-software package, also separately available on CTAN.
%%
%% Look at the sample-*-biblatex.tex files for templates showcasing
%% the biblatex styles.
%%

%%
%% The majority of ACM publications use numbered citations and
%% references.  The command \citestyle{authoryear} switches to the
%% "author year" style.
%%
%% If you are preparing content for an event
%% sponsored by ACM SIGGRAPH, you must use the "author year" style of
%% citations and references.
%% Uncommenting
%% the next command will enable that style.
%%\citestyle{acmauthoryear}
\usepackage{amsmath,amsfonts}
\usepackage{amsthm}
\usepackage{algorithm}
\usepackage{algpseudocode}
\usepackage{cleveref}

\newtheorem{theorem}{Theorem}
\newtheorem{lemma}[theorem]{Lemma}
\newtheorem{claim}[theorem]{Claim}
\newtheorem{fact}[theorem]{Fact}
\newtheorem{definition}[theorem]{Definition}
\newtheorem{example}[theorem]{Example}
\newtheorem{remark}[theorem]{Remark}
\newtheorem{construction}[theorem]{Construction}
\newtheorem{corollary}[theorem]{Corollary}
\newtheorem{proposition}[theorem]{Proposition}
%%
%% end of the preamble, start of the body of the document source.
\begin{document}

%%
%% The "title" command has an optional parameter,
%% allowing the author to define a "short title" to be used in page headers.
\title{The Name of the Title Is Hope}

%%
%% The "author" command and its associated commands are used to define
%% the authors and their affiliations.
%% Of note is the shared affiliation of the first two authors, and the
%% "authornote" and "authornotemark" commands
%% used to denote shared contribution to the research.
\author{tbd}
%\authornote{}
%\email{tbd}
%\affiliation{%
  %\institution{tbd}
  %\streetaddress{P.O. Box 1212}
  %\city{tbd}
  %\state{tbd}
  %\country{tbd}
  %\postcode{tbd}
%}

%%
%% By default, the full list of authors will be used in the page
%% headers. Often, this list is too long, and will overlap
%% other information printed in the page headers. This command allows
%% the author to define a more concise list
%% of authors' names for this purpose.
\renewcommand{\shortauthors}{tbd}

%%
%% The abstract is a short summary of the work to be presented in the
%% article.
\begin{abstract}
  tbd. 
\end{abstract}

%%
%% The code below is generated by the tool at http://dl.acm.org/ccs.cfm.
%% Please copy and paste the code instead of the example below.
%%
\begin{CCSXML}
  <ccs2012>
     <concept>
         <concept_id>10003752.10003777.10003788</concept_id>
         <concept_desc>Theory of computation~Cryptographic primitives</concept_desc>
         <concept_significance>500</concept_significance>
         </concept>
   </ccs2012>
\end{CCSXML}
  
\ccsdesc[500]{Theory of computation~Cryptographic primitives}

%%
%% Keywords. The author(s) should pick words that accurately describe
%% the work being presented. Separate the keywords with commas.
\keywords{tbd}
%% A "teaser" image appears between the author and affiliation
%% information and the body of the document, and typically spans the
%% page.


%\received{20 February 2007}
%\received[revised]{12 March 2009}
%\received[accepted]{5 June 2009}

%%
%% This command processes the author and affiliation and title
%% information and builds the first part of the formatted document.
\maketitle

\section{Introduction}
tbd

\section{Preliminary}
\subsection{Basic Notations}

\subsection{Distributed Point Function}
Definition of distributed (multi-)point function, naive construction
\subsection{Batch Codes}
combinatorial/probabilistic batch codes, with cuckoo hashing a concrete instantiation
\subsection{Oblivious Key-Value Stores}
definition of OKVS, concrete instantiations(polynomial, sparse matrix). mention some connections to cuckoo hashing
\section{New DMPF constructions}
\subsection{Big-State DMPF}
display the big-state DMPF (plus distributed gen)
%We display the big-state DMPF scheme in figure \ref{fig:big-state}. 
\iffalse
\begin{figure}\label{fig:big-state}
  \fbox{\parbox{\linewidth}{
  \begin{algorithmic}[1]
  \Procedure{Gen}{$1^\lambda, A$}
  \State $t\leftarrow |A|, n\gets |A[1]|$.
  \State For $0\le i\le n-1$, let $A^{(i)}$ be the sorted and deduplicated list of $i$-bit prefixes of strings in $A$. Specifically, $A^{(0)} = [\epsilon]$. 
  \State Let $G:\{0,1\}^\lambda\rightarrow \{0,1\}^{2\lambda+2t}$ be a public PRG. 
  \State Set $S_b^{(0)} = [r_b]$ and $T_b^{(0)} = [b||0^{m-1}]$ for $b = 0,1$ where $r_0,r_1$ are sampled independently and randomly from $\{0,1\}^\lambda$.
  \For{$i=1$ to $n$}
  \State Let $CW^{(i)}, S_0^{(i)}, T_0^{(i)}, S_1^{(i)}, T_1^{(i)}$ be empty lists. 
  \For{$l = 1$ to $A^{(i-1)}$}
  \State Parse $G(S_b^{(i-1)}[l]) = s_b^L||t_b^L||s_b^R||t_b^R$ for $b = 0,1$ where $s_b^L,s_b^R\in\{0,1\}^\lambda$ and $t_b^L,t_b^R\in\{0,1\}^t$.
  \If{$A^{(i-1)}[l]||0\in A^{(i)}$ and $A^{(i-1)}[l]||1\not\in A^{(i)}$}
  \State $d\gets$ the index of $A^{(i-1)}[l]||0$ in $A^{(i)}$. 
  \State Append $s_0^R\oplus s_1^R||t_0^L\oplus t_1^L\oplus e_d||t_0^R\oplus t_1^R$ to $CW^{(i)}$ where $e_d = 0^{d-1}10^{t-d}$.
  \ElsIf{$A^{(i-1)}[l]||1\in A^{(i)}$ and $A^{(i-1)}[l]||0\not\in A^{(i)}$}
  \State $d\gets$ the index of $A^{(i-1)}[l]||1$ in $A^{(i)}$. 
  \State Append $s_0^L\oplus s_1^L||t_0^L\oplus t_1^L||t_0^R\oplus t_1^R\oplus e_d$ to $CW^{(i)}$.
  \Else\Comment{both $A^{(i-1)}[l]||0$ and $A^{(i-1)}[l]||1$ are in $A^{(i)}$.}
  \State $d\gets$ the index of $A^{(i-1)}[l]||0$ in $A^{(i)}$. 
  \State Randomly sample $r$ from $\{0,1\}^\lambda$.
  \State Append $r||t_0^L\oplus t_1^L\oplus e_d||t_0^R\oplus t_1^R\oplus e_{d+1}$ to $CW^{(i)}$.
  \EndIf
  \EndFor
  \State Randomly and independently sample $t-|CW^{(i)}|$ strings from $\{0,1\}^{\lambda+2t}$ .
  \State If $i=n$ then skip the following for-loop. 
  \For{$l = 1$ to $|A^{(i-1)}|$}
  \State Parse $G(S_b^{(i-1)}[l]) = s_b^L||t_b^L||s_b^R||t_b^R$ for $b = 0,1$.
  \State Parse $T_b^{(i-1)}[l]\cdot CW^{(i)} = \Delta s_b||\Delta t_b^L||\Delta t_b^R$ for $b = 0,1$.
  \If{$A^{(i-1)}[l]||0\in A^{(i)}$ and $A^{(i-1)}[l]||1\not\in A^{(i)}$}
  \State Append $s_b^L\oplus \Delta s_b$ to $S_b^{(i)}$ and $t_b^L\oplus \Delta t_b^L$ to $T_b^{(i)}$, for $b = 0,1$.
  \ElsIf{$A^{(i-1)}[l]||1\in A^{(i)}$ and $A^{(i-1)}[l]||0\not\in A^{(i)}$}
  \State Append $s_b^R\oplus \Delta s_b$ to $S_b^{(i)}$ and $t_b^R\oplus \Delta t_b^R$ to $T_b^{(i)}$, for $b = 0,1$.
  \Else
  \State Append $s_b^L\oplus \Delta s_b$ to $S_b^{(i)}$ and $t_b^L\oplus \Delta t_b^L$ to $T_b^{(i)}$, for $b = 0,1$.
  \State Append $s_b^R\oplus \Delta s_b$ to $S_b^{(i)}$ and $t_b^R\oplus \Delta t_b^R$ to $T_b^{(i)}$, for $b = 0,1$.
  \EndIf
  \EndFor
  \EndFor
  \For{$l=1$ to $t$}\Comment{convert layer}
  \State Append $(-1)^{T_0^{(n)}[l][l]}\cdot\big(G_{convert}(S_0^{(n)}[l])-G_{convert}(S_1^{(n)}[l])-B[l]\big)$ to $CW^{(n+1)}$. 
  \EndFor
  \State Set $k_b \gets (S_b^{(0)}, CW^{(1)},CW^{(2)},\cdots,CW^{(n)})$.
  \State \textbf{return} $(k_0,k_1)$.
  \EndProcedure
  \Procedure{Eval}{$1^\lambda, b,k_b,x$}
  \State Parse $k_b = ([s],CW^{(1)},CW^{(2)},\cdots,CW^{(n)})$. 
  \State $t\leftarrow$ number of rows of any $CW^{(i)}$.
  \State $c\gets b||0^{t-1}$.
  \For{$i = 1$ to $n$}
  \State Parse $c\cdot CW^{(i)} = \Delta s||\Delta t^0||\Delta t^1$ where $\Delta s\in\{0,1\}^\lambda$ and $\Delta t^0,\Delta t^1\in \{0,1\}^t$. 
  \State Parse $G(s) = s^0||t^0||s^1||t^1$. 
  \State $s||c\gets (s^{x[i]}||t^{x[i]})\oplus (\Delta s||\Delta t^{x[i]})$
  \EndFor
  \State \textbf{return} $s||\bigoplus_{j = 1}^t c[j]$
  \EndProcedure
  \end{algorithmic}
  }}
  \caption{The big-state DMPF scheme}
\end{figure}
\fi

\subsection{Batch-Code DMPF}
display the batch-code DMPF 

\subsection{OKVS-based DMPF}
display the OKVS-based DMPF (plus distributed gen)

\subsection{Comparison}
Comparison table dependent to PRG \& F-MUL(list the formulas?)\\
analyze tradeoff\\
distributed gen advantage

\section{Applications}
\subsection{PCG for OLE from Ring-LPN}
Characterize parameters\\
show nonregular optimization\\
plug in new DMPF and show overall optimization
\subsection{PSI-WCA}
plug in new DMPF and analyze advantage interval\\
plug in distributed gen
\subsection{Heavy-hitters}
private heavy-hitter\\
or parallel ORAM?
\section{Acknowledgments}
tbd


\bibliographystyle{ACM-Reference-Format}
\bibliography{references}


%%
%% If your work has an appendix, this is the place to put it.
\appendix
\section{Batch-code DMPF scheme}
\section{Security Proofs}

\end{document}
\endinput
%%
%% End of file `sample-sigconf.tex'.
