\documentclass[sigconf]{acmart}
\AtBeginDocument{%
  \providecommand\BibTeX{{%
    Bib\TeX}}}

%% Rights management information.  This information is sent to you
%% when you complete the rights form.  These commands have SAMPLE
%% values in them; it is your responsibility as an author to replace
%% the commands and values with those provided to you when you
%% complete the rights form.
\setcopyright{acmcopyright}
\copyrightyear{tbd}
\acmYear{tbd}
\acmDOI{tbd}

%% These commands are for a PROCEEDINGS abstract or paper.
\acmConference[Conference acronym 'XX]{Make sure to enter the correct conference title from your rights confirmation emai}{tbd}{tbd}
%%
%%  Uncomment \acmBooktitle if the title of the proceedings is different
%%  from ``Proceedings of ...''!
%%
%%\acmBooktitle{Woodstock '18: ACM Symposium on Neural Gaze Detection,
%%  June 03--05, 2018, Woodstock, NY}
\acmPrice{15.00}
\acmISBN{tbd}


\usepackage{amsmath,amsfonts}
\usepackage{amsthm}
\usepackage{algorithm}
\usepackage{algpseudocode}
\usepackage{cleveref}
\usepackage{fancybox}
    
%% New commands goes here
\newcommand{\Enote}[1]{\color{purple}Enote: #1\color{black}}
\newcommand{\Yaxin}[1]{\color{purple}Yaxin: #1\color{black}}

\newcommand{\Gen}{{\sf Gen}}
\newcommand{\Eval}{{\sf Eval}}
\newcommand{\FullEval}{{\sf FullEval}}
\newcommand{\Encode}{{\sf Encode}}
\newcommand{\Decode}{{\sf Decode}}
\newcommand{\row}{{\sf row}}
\newcommand{\seed}{{\sf seed}}
\newcommand{\sign}{{\sf sign}}
\newcommand{\map}{{\sf map}}

\newcommand{\OKVS}{{\sf OKVS}}
\newcommand{\Adv}{{\sf Adv}}
    
\newcommand{\GG}{\mathbb{G}}
\newcommand{\NN}{\mathbb{N}}
\newcommand{\ZZ}{\mathbb{Z}}
\newcommand{\FF}{\mathbb{F}}

\newcommand{\ipd}[2]{\langle #1, #2 \rangle}

\newtheorem{theorem}{Theorem}
\newtheorem{lemma}[theorem]{Lemma}
\newtheorem{claim}[theorem]{Claim}
\newtheorem{fact}[theorem]{Fact}
\newtheorem{definition}[theorem]{Definition}
\newtheorem{example}[theorem]{Example}
\newtheorem{remark}[theorem]{Remark}
\newtheorem{construction}{Construction}
\newtheorem{corollary}[theorem]{Corollary}
\newtheorem{proposition}[theorem]{Proposition}
%%
\begin{document}

%%
%% The "title" command has an optional parameter,
%% allowing the author to define a "short title" to be used in page headers.
\title{The Name of the Title Is Hope}

\author{tbd}
%\authornote{}
%\email{tbd}
%\affiliation{%
  %\institution{tbd}
  %\streetaddress{P.O. Box 1212}
  %\city{tbd}
  %\state{tbd}
  %\country{tbd}
  %\postcode{tbd}
%}

%%
%% By default, the full list of authors will be used in the page
%% headers. Often, this list is too long, and will overlap
%% other information printed in the page headers. This command allows
%% the author to define a more concise list
%% of authors' names for this purpose.
\renewcommand{\shortauthors}{tbd}

\begin{abstract}
  tbd. 
\end{abstract}

%%
%% The code below is generated by the tool at http://dl.acm.org/ccs.cfm.
\begin{CCSXML}
  <ccs2012>
     <concept>
         <concept_id>10003752.10003777.10003788</concept_id>
         <concept_desc>Theory of computation~Cryptographic primitives</concept_desc>
         <concept_significance>500</concept_significance>
         </concept>
   </ccs2012>
\end{CCSXML}
  
\ccsdesc[500]{Theory of computation~Cryptographic primitives}

%%
%% Keywords. The author(s) should pick words that accurately describe
%% the work being presented. Separate the keywords with commas.
\keywords{tbd}

%\received{20 February 2007}
%\received[revised]{12 March 2009}
%\received[accepted]{5 June 2009}

\maketitle

\section{Introduction}
tbd

\section{Preliminary}
  \input{prelim.tex}
\section{New DMPF constructions}
In this section, we display two new constructions of DMPF that follow the same construction paradigm shown in \cref{fig:DMPF_paradigm}. 

We begin by introducing the DMPF paradigm in \cref{fig:DMPF_paradigm}, which is based on the idea of the DPF construction in \cite{CCS:BoyGilIsh16}. Each key $k_b(b=0,1)$ generated by ${\sf Gen}(\hat{f}_{A,B})$ can span a height-$n$ ($n$ is the input length of $\hat{f}_{A,B}$) complete binary tree $T_b$ (call it the evaluation tree), with which party $b$ can evaluate the input $x=x_1\cdots x_n$ by starting from the root of this tree, on the $i$th layer going left if $x_i=0$ and going right if $x_i=1$, until reaching a leaf node then computing the result according to this leaf node. 

Each node of this tree is associated with a $\lambda$-bit seed and a $l$-bit sign. For a parent node on the $i$th layer with seed $\sf sd$ and sign $\sf sig$, its children's seeds and signs are generated by $\sf PRG(sd)\oplus Correction$, where the $\sf Correction$ is determined by the parent node's position, its sign $\sf sig$ and a correction word $CW^{(i)}$ associated with that layer (computed by the method $\sf Correct()$). On a leaf node on the last layer, its seed $\sf sd$ will generate a random element in the output group, which will be corrected by adding a $\sf Correction$ determined by the leaf node's position, its sign and the last correction word $CW^{(n+1)}$ (computed by the method $\sf ConvCorrect()$). 

Call any path from the root a leaf corresponding to an input string in $A$ an accepting path. To force the correctness, we maintain the following invariance on the evaluation trees $T_0$, $T_1$ of the two parties: 
\begin{itemize}
  \item If a node is not on any accepting path, then $T_0$ and $T_1$ assign to it with the same seed and sign. 
  \item If a node is on an accepting path, then $T_0$ and $T_1$ assign to it with different signs that controls the corrections on its children (or on the output if the node is on the last layer). 
\end{itemize}

The paradigm contains four methods ($\sf GenCW$, $\sf GenConvCW$, $\sf Correct$, $\sf ConvCorrect$) and the sign length $l$ to be determined by different constructions. We make the following restrictions on the methods in order to guarantee the invariance on the evaluation trees: 

${\sf M}(\bar{x},\sign, CW) =\sum_{i=1}^l\sign[i]\cdot{\sf M}(\bar{x},0^{i-1}10^{l-i},CW)$ for all $\sf M\in\{Correct, ConvCorrect\}$, input $\bar{x}$ and $CW$.

\begin{figure*}
  \fbox{\parbox{\linewidth}{
  \begin{algorithmic}
    \State \textbf{Public parameters: }
    \State The $t$-point function family $\{f_{A,B}\}$ with $t$ an upperbound of the number of nonzero points, input domain $[N]=\{0,1\}^n$ and the output group $\GG$. 
    \State Suppose there is a public PRG $G:\{0,1\}^\lambda\rightarrow \{0,1\}^{2\lambda+2l}$. Parse $G = G_0\|G_1$ to the left half and right half. 
    \State Suppose there is a public PRG $G_{\sf convert}:\{0,1\}^\lambda\rightarrow \GG$. 
    \item[]
    \Procedure{Gen}{$1^\lambda, \hat{f}_{A,B}$}
    \State Denote $A = (\alpha_1,\cdots,\alpha_t)$ in lexicographical order, $B = (\beta_1,\cdots,\beta_t)$. If $|A|<t$, extend $A$ to size-$t$ with arbitrary $\{0,1\}^n$ strings and $B$ with 0's. 
    \State For $0\le i\le n-1$, let $A^{(i)}$ denote the sorted and deduplicated list of $i$-bit prefixes of strings in $A$. Specifically, $A^{(0)} = [\epsilon]$. 
    \State For $0\le i\le n-1$ and $b=0,1$, initialize empty lists $\seed_b^{(i)}$ and $\sign_b^{(i)}$. 
    \State ${\sf Initialize}(\{\seed_b^{(0)},\sign_b^{(0)}\}_{b=0,1})$. 
    \For{$i=1$ to $n$}
    \State $CW^{(i)}\gets {\sf GenCW}(i,A,\{\seed_b^{(i-1)},\sign_b^{(i-1)}\}_{b=0,1})$. 
      \For{$k = 1$ to $|A^{(i-1)}|$ and $z=0,1$}
        \State Compute $C_{\seed,b}\|C_{\sign^0,b}\|C_{\sign^1,b}\gets {\sf Correct}(A^{(i-1)}[k], \sign_b^{(i-1)}[k], CW^{(i)})$ for $b=0,1$. 
        \If{$A^{(i-1)}[k]\|z\in A^{(i)}$}
        \State Append the first $\lambda$ bit of $G_z(\seed_b^{(i-1)}[k])\oplus(C_{\seed,b}\|C_{\sign^z,b})$ to $\seed_b^{(i)}$ and the rest to $\sign_b^{(i)}$. 
        \EndIf
      \EndFor
    \EndFor
    \State $CW^{(n+1)}\gets{\sf GenConvCW}(A,B,\{\seed_b^{(n)},\sign_b^{(n)}\}_{b=0,1})$. 
    \State Set $k_b \gets (\seed_b^{(0)},\sign_b^{(0)}, CW^{(1)},CW^{(2)},\cdots,CW^{(n+1)})$.
    \State \textbf{return} $(k_0,k_1)$.
    \EndProcedure
    \item[]
    \Procedure{Eval\(_b\)}{$1^\lambda, k_b,x$}
    \State Parse $k_b = ([\seed],[\sign],CW^{(1)},CW^{(2)},\cdots,CW^{(n+1)})$. 
    \State Denote $x=x_1x_2\cdots x_n$. 
    \For{$i = 1$ to $n$}
      \State $C_\seed\|C_{\sign^0}\|C_{\sign^1}\gets {\sf Correct}(x_1\cdots x_{i-1},\sign,CW^{(i)})$.
      \State $\seed\|\sign\gets (\seed\oplus C_{\seed})\|(\sign\oplus C_{\sign^{x_i}})$. 
    \EndFor
    \State \Return $(-1)^b\cdot \big(G_{\sf convert}(\seed)+{\sf ConvCorrect}(x,\sign,CW^{(n+1)})\big)$. 
    \EndProcedure
    \item[]
    \Procedure{FullEval\(_b\)}{$1^\lambda,k_b$}
    \State Parse $k_b=(\seed^{(0)},\sign^{(0)},CW^{(1)},CW^{(2)},\cdots,CW^{(n+1)})$. 
    \State For $1\le i\le n$, ${\sf Path}^{(i)}\gets$ the lexicographical ordered list of $\{0,1\}^i$. ${\sf Path}^{(0)}\gets[\epsilon]$. 
    \For{$i=1$ to $n$}
      \For{$k = 1$ to $2^{i-1}$}
        \State $C_\seed\|C_{\sign^0}\|C_{\sign^1}\gets {\sf Correct}({\sf Path}{(i-1)}[k],\sign^{(i-1)}[k],CW^{(i)})$.
        \State $\seed^{(i)}[2k]\|\sign^{(i)}[2k]\gets G_0(\seed^{(i-1)}[k])\oplus (C_\seed\|C_{\sign^0})$.
        \State $\seed^{(i)}[2k+1]\|\sign^{(i)}[2k+1]\gets G_1(\seed^{(i-1)}[k])\oplus (C_\seed\|C_{\sign^1})$.
      \EndFor
    \EndFor
    \For{$k = 1$ to $2^n$}
      \State ${\sf Output}[k]\gets (-1)^b\cdot \big(G_{\sf convert}(\seed^{(n)}[k])+{\sf ConvCorrect}({\sf Path}[k],\sign^{(n)}[k],CW^{(n+1)})\big)$.
    \EndFor
    \State\Return $\sf Output$. 
    \EndProcedure
    \end{algorithmic}}}
  \caption{The paradigm of our DMPF schemes. We leave the PRG expand length $l$, methods $\sf Initialize, GenCW,$ $\sf GenConvCW, Correct, ConvCorrect$ to be determined by specific constructions. }
  \label{fig:DMPF_paradigm}
\end{figure*}

\subsection{Big-State DMPF}
TBD: explain
\begin{figure}[H]
  \fbox{\parbox{\linewidth}{
  \begin{algorithmic}
    \State Set $l\leftarrow t$, the upperbound of $|A|$. 
    \Procedure{Initialize}{$\{\seed_b^{(0)},\sign_b^{(0)}\}_{b=0,1}$}
    \State For $b=0,1$, let $\seed_b^{(0)} = [r_b]$ where $r_b\xleftarrow{\$}\{0,1\}^\lambda$. 
    \State For $b=0,1$, set $\sign_b^{(0)} = [b\|0^{t-1}]$. 
    \EndProcedure
    \item[]
    \Procedure{GenCW}{$i,A,\{\seed_b^{(i-1)},\sign_b^{(i-1)}\}_{b=0,1}$}
    \State Let $\{A^{(i)}\}_{0\le i\le n}$ be defined as in~\cref{fig:DMPF_paradigm}. 
    \State Sample a list $CW$ of $t$ random strings from $\{0,1\}^{\lambda+2t}$.  
    \For{$k = 1$ to $|A^{(i-1)}|$}
      \State Parse $G(\seed_b^{(i-1)}[k]) = \seed_b^0\|\sign_b^0\|\seed_b^1\|\sign_b^1$, for $b=0,1$, $\seed_b^0,\seed_b^1\in\{0,1\}^\lambda$ and $\sign_b^0,\sign_b^1\in\{0,1\}^t$. 
      \State Compute $\Delta\seed^c = \seed_0^c\oplus\seed_1^c$ and $\Delta \sign^c = \sign_0^c\oplus\sign_1^c$ for $c=0,1$. 
      \State Denote ${\sf path}\leftarrow A^{(i-1)}[k]$. 
      \If{both ${\sf path}\|z$ for $z=0,1$ are in $A^{(i)}$}
        \State $d\gets$ the index of ${\sf path}\|0$ in $A^{(i)}$.
        \State $CW[d]\gets r\|\Delta\sign^0\oplus e_d \|\Delta\sign^1\oplus e_{d+1}$ where $r\xleftarrow{\$}\{0,1\}^\lambda$, $e_d = 0^{d-1}10^{t-d}$. 
      \Else
        \State Let $z$ be such that ${\sf path}\|z\in A^{(i)}$. 
        \State $d\gets$ the index of ${\sf path}\|z$ in $A^{(i)}$. 
        \State $CW[d]\gets 
          \begin{cases}
            \Delta \seed^1\|\Delta\sign^0\oplus e_d\|\Delta\sign^1 & z=0\\
            \Delta \seed^0\|\Delta\sign^0\|\Delta\sign^1\oplus e_d & z=1
          \end{cases}$.        
      \EndIf 
    \EndFor
    \State\Return $CW$. 
    \EndProcedure
    \item[]
    \Procedure{GenConvCW}{$A,B,\{\seed_b^{(n)},\sign_b^{(n)}\}$}
      \State Sample a list $CW$ of $t$ random $\GG$-elements.  
      \For{$k = 1$ to $|A|$}
        \State $\Delta g\gets G_{\sf convert}(\seed_0^{(n)}[k]) - G_{\sf convert}(\seed_1^{(n)}[k])$. 
        \State$CW[k]\gets (-1)^{\sign_0^{(n)}[k][k]}(\Delta g-B[k])$.
      \EndFor
      \State \Return $CW$. 
    \EndProcedure
    \item[]
    \Procedure{Correct}{$\bar{x},\sign,CW$}
      \State \Return $C_{\seed}\|C_{\sign^0}\|C_{\sign^1}\gets\sum_{i=1}^t \sign[i]\cdot CW[i]$, where $C_{\sign^0}$ and $C_{\sign^1}$ are $t$-bit. 
    \EndProcedure
    \item[]
    \Procedure{ConvCorrect}{$x,\sign,CW$}
      \State \Return $\sum_{i=1}^t \sign[i]\cdot CW[i]$. 
    \EndProcedure
  \end{algorithmic}}}
  \caption{The parameter $l$ and methods' setting that turns the paradigm of DMPF in~\cref{fig:DMPF_paradigm} into the big-state DMPF. }
  \label{fig:DMPF_big-state}
\end{figure}

\subsection{Batch-Code DMPF}
display the batch-code DMPF 

\subsection{OKVS-based DMPF}
TBD: explain
\begin{figure}
  \fbox{\parbox{\linewidth}{
  \begin{algorithmic}
    \State Set $l\leftarrow 1$. 
    \State For $1\le i\le n$, let $\OKVS_i$ be an OKVS scheme (\cref{def:OKVS}) with key space $\mathcal{K} = \{0,1\}^{i-1}$, value space $\mathcal{V} = \{0,1\}^{\lambda+2}$ and input length $t$. 
    \State let $\OKVS_{\sf convert}$ be an OKVS scheme with key space $\mathcal{K} = \{0,1\}^n$, value space $\mathcal{V} = \GG$ and input length $t$. 
    \item[]
    \Procedure{Initialize}{$\{\seed_b^{(0)},\sign_b^{(0)}\}_{b=0,1}$}
    \State For $b=0,1$, let $\seed_b^{(0)} = [r_b\xleftarrow{\$}\{0,1\}^\lambda]$ and $\sign_b^{(0)} = [b]$. 
    \EndProcedure
    \item[]
    \Procedure{GenCW}{$i,A,\{\seed_b^{(i-1)},\sign_b^{(i-1)}\}_{b=0,1}$}
    \State Let $\{A^{(i)}\}_{0\le i\le n}$ be defined as in~\cref{fig:DMPF_paradigm}. 
    \State Sample a list $V$ of $t$ random strings from $\{0,1\}^{\lambda+2}$.  
    \For{$k = 1$ to $|A^{(i-1)}|$}
      \State Parse $G(\seed_b^{(i-1)}[k]) = \seed_b^0\|\sign_b^0\|\seed_b^1\|\sign_b^1$, for $b=0,1$, $\seed_b^0,\seed_b^1\in\{0,1\}^\lambda$ and $\sign_b^0,\sign_b^1\in\{0,1\}$. 
      \State Compute $\Delta\seed^c = \seed_0^c\oplus\seed_1^c$ and $\Delta \sign^c = \sign_0^c\oplus\sign_1^c$ for $c=0,1$. 
      \State Denote ${\sf path}\leftarrow A^{(i-1)}[k]$. 
      \If{both ${\sf path}\|z$ for $z=0,1$ are in $A^{(i)}$}
        \State $V[k]\gets r\|\Delta\sign^0\oplus 1\|\Delta\sign^1\oplus 1$, where $r\xleftarrow{\$}\{0,1\}^\lambda$. 
      \Else
        \State Let $z$ be such that ${\sf path}\|z\in A^{(i)}$. 
        \State $V[k]\gets \Delta \seed^1\|\Delta\sign^0\oplus (1-z)\|\Delta\sign^1\oplus z$.        
      \EndIf
    \EndFor
    \State \Return $\OKVS_i.\Encode(\{A^{(i-1)}[k], V[k]\}_{1\le k\le |A^{(i-1)}|})$. 
    \EndProcedure
    \item[]
    \Procedure{GenConvCW}{$A,B,\{\seed_b^{(n)},\sign_b^{(n)}\}$}
      \State Sample a list $V$ of $t$ random $\GG$-elements. 
      \For{$k = 1$ to $|A|$}
        \State $\Delta g\gets G_{\sf convert}(\seed_0^{(n)}[k]) - G_{\sf convert}(\seed_1^{(n)}[k])$. 
        \State$V[k]\gets (-1)^{\sign_0^{(n)}[k][k]}(\Delta g-B[k])$.
      \EndFor 
      \State \Return $\OKVS_{\sf convert}(\{A[k], V[k]\}_{1\le k\le t})$. 
    \EndProcedure
    \item[]
    \Procedure{Correct}{$\bar{x}, \sign,CW$}
      \State\Return $C_{\seed}\|C_{\sign^0}\|C_{\sign^1}\gets\sign\cdot\OKVS_i.\Decode(CW, \bar{x})$, where $C_{\sign^0}$ and $C_{\sign^1}$ are bits. 
    \EndProcedure
    \item[]
    \Procedure{ConvCorrect}{$x,\sign,CW$}
      \State \Return $\sign\cdot\OKVS_{\sf convert}.\Decode(CW,x)$. 
    \EndProcedure
  \end{algorithmic}}}
  \caption{The parameter $l$ and methods' setting that turns the paradigm of DMPF in~\cref{fig:DMPF_paradigm} into the OKVS-based DMPF. }
  \label{fig:DMPF_OKVS}
\end{figure}


\subsection{Comparison}
Comparison table dependent to PRG \& F-MUL(list the formulas?)\\
analyze tradeoff\\
distributed gen advantage
\subsection{Distributed Key Generation}
\section{Applications}
\subsection{PCG for OLE from Ring-LPN}
Characterize parameters\\
show nonregular optimization\\
plug in new DMPF and show overall optimization
\subsection{PSI-WCA}
plug in new DMPF and analyze advantage interval\\
plug in distributed gen
\subsection{Heavy-hitters}
private heavy-hitter\\
or parallel ORAM?
\section{Acknowledgments}
tbd


\bibliographystyle{ACM-Reference-Format}
\bibliography{references}


%%
%% If your work has an appendix, this is the place to put it.
\appendix
\section{Batch-code DMPF scheme}
\section{Security Proofs}

\end{document}
\endinput
%%
%% End of file `sample-sigconf.tex'.
