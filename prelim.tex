\subsection{Basic Notations}



 \paragraph{Point and multi-point functions.} Given a domain size $N$ and Abelian group $\GG$, a \emph{point function} $f_{\alpha,\beta}:[N]\rightarrow\GG$ for $\alpha\in[N]$ and $\beta\in\GG$ evaluates to $\beta$ on input $\alpha$ and to $0\in\GG$ on all other inputs. We denote by $\hat{f}_{\alpha,\beta}=(N,\hat{\GG},\alpha,\beta)$ the representation of such a point function. A \emph{$t$-point function} $f_{A,B}:[N]\rightarrow \GG$ for $A=(\alpha_1,\cdots\alpha_t)\in[N]^t$ and $B=(\beta_1,\cdots,\beta_t)\in \GG^t$ evaluates to $\beta_i$ on input $\alpha_i$ for $1\le i\le t$ and to $0$ on all other inputs. Denote $\hat{f}_{A,B}(N,\hat{\GG},t,A,B)$ the representation of such a $t$-point function. Call the collection of all $t$-point functions for all $t$ \emph{multi-point functions}. 
 
\Enote{MPF. Also representation of groups.}
 
\subsection{Distributed Multi-Point Functions}

\Enote{should directly adapt to multi-point function case}

We begin by defining a slightly generalized notion of distributed point functions (DPFs), which accounts for the extra parameter $\GG'$. \Yaxin{What is $\GG'$?}

%\yuval{Made some small stylistic changes below (similar changes may apply elsewhere). Should citations to GI14,BGI16. }
\begin{definition}[DPF \cite{EC:GilIsh14,CCS:BoyGilIsh16}]\label{def:dpf}
A 
%$t$-private $m$-server 
(2-party)
\emph{distributed point function (DPF)}
%, or $(m,t)$-DPF for short, 
is a triple of algorithms %$\Pi=(\Gen,\Eval_0,\ldots,\Eval_{m-1})$ 
$\Pi=(\Gen,\Eval_0,\Eval_1)$
with the following syntax: 
\begin{itemize}
    \item $\Gen(1^\lambda,\hat{f}_{\alpha,\beta})\rightarrow (k_0,k_1)$: On input security parameter $\lambda\in\NN$ and point function description $\hat{f}_{\alpha,\beta}=(N,\hat{\GG},\alpha,\beta)$, the (randomized) key generation algorithm $\Gen$ returns a pair of keys $k_0,k_1\in\{0,1\}^*$. We assume that $N$ and $\GG$ are determined by each key.
    \item $\Eval_i(k_i,x)\rightarrow y_i$: On input key $k_i\in\{0,1\}^*$ and input $x\in[N]$ the (deterministic) evaluation algorithm of server $i$, $\Eval_i$ returns 
    %a group element 
    $y_i\in\GG$.
\end{itemize}
%The algorithms $\Pi=(\Gen,\Eval_0,\ldots,\Eval_{m-1})$ should 
We require $\Pi$ to satisfy the following requirements:
\begin{itemize}
    \item \textbf{Correctness:} For every $\lambda$, $\hat{f}=\hat{f}_{\alpha,\beta}=(N,\hat{\GG},\alpha,\beta)$ such that $\beta\in\GG$, and $x\in[N]$, if $(k_0,k_1)\leftarrow\Gen(1^\lambda,\hat{f})$, then $$\Pr\left[\sum_{i=0}^{1}\Eval_i(k_i,x)=f_{\alpha,\beta}(x)\right]=1$$
    \item \textbf{Security:} Consider the following semantic security challenge experiment for corrupted server $i\in\{0,1\}$:
    \begin{enumerate}
        \item The adversary produces two point function descriptions $(\hat{f}^0=(N,\hat\GG,\alpha_0,\beta_0),\hat{f}^1=(N,\hat\GG,\alpha_1,\beta_1))\leftarrow\mathcal{A}(1^\lambda)$, where $\alpha_i\in[N]$ and $\beta_i\in\GG$.
        \item The challenger samples $b\gets\{0,1\}$ and $(k_0,k_1)\leftarrow\Gen(1^\lambda,\hat{f}^b)$.
        \item The adversary outputs a guess $b'\leftarrow\mathcal{A}(k_i)$.
    \end{enumerate}
    Denote by $\Adv(1^\lambda,\mathcal{A},i)=\Pr[b=b']-1/2$ the advantage of $\mathcal{A}$ in guessing $b$ in the above experiment. For every non-uniform polynomial time adversary $\mathcal{A}$ there exists a negligible function $\nu$ such that $\Adv(1^\lambda,\mathcal{A},i) \le \nu(\lambda)$ for all $\lambda \in \NN$.
%    For circuit size bound $S=S(\lambda)$ and advantage bound $\epsilon(\lambda)$, we say that $\Pi$ is $(S,\epsilon)$-secure if for all $i\in\{0,1\}$ and all non-uniform adversaries $\mathcal{A}$ of size $S(\lambda)$ and sufficiently large $\lambda$, we have $\Adv(1^\lambda,\mathcal{A},i)\leq\epsilon(\lambda)$. We say that $\Pi$ is:
%    \begin{itemize}
%        \item \emph{Computationally $\epsilon$-secure} if it is $(S,\epsilon)$-secure for all polynomials $S$.
%        \item \emph{Computationally secure} if it is $(S,1/S)$-secure for all polynomials $S$.
%        %\item \emph{Statistically $\epsilon$-secure} if it is $(S,\epsilon)$-secure for all $S$.
%        %\item \emph{Perfectly secure} if it is statistically $0$-secure.
%    \end{itemize}
\end{itemize}
%If the security threshold $t$ is unspecified, we assume it is $t=1$.
\end{definition}

\begin{definition}[DMPF]\label{def:dmpf}
  A 
  %$t$-private $m$-server 
  (2-party)
  \emph{distributed multi-point function (DMPF)}
  %, or $(m,t)$-DPF for short, 
  is a triple of algorithms %$\Pi=(\Gen,\Eval_0,\ldots,\Eval_{m-1})$ 
  $\Pi=(\Gen,\Eval_0,\Eval_1)$
  with the following syntax: 
  \begin{itemize}
      \item $\Gen(1^\lambda,\hat{f}_{A,B})\rightarrow (k_0,k_1)$: On input security parameter $\lambda\in\NN$ and point function description $\hat{f}_{A,B}=(N,\hat{\GG},t,A,B)$, the (randomized) key generation algorithm $\Gen$ returns a pair of keys $k_0,k_1\in\{0,1\}^*$. 
      \item $\Eval_i(k_i,x)\rightarrow y_i$: On input key $k_i\in\{0,1\}^*$ and input $x\in[N]$ the (deterministic) evaluation algorithm of server $i$, $\Eval_i$ returns $y_i\in\GG$.
  \end{itemize}
  We require $\Pi$ to satisfy the following requirements:
  \begin{itemize}
      \item \textbf{Correctness:} For every $\lambda$, $\hat{f}=\hat{f}_{\alpha,\beta}=(N,\hat{\GG},\alpha,\beta)$ such that $\beta\in\GG$, and $x\in[N]$, if $(k_0,k_1)\leftarrow\Gen(1^\lambda,\hat{f})$, then $$\Pr\left[\sum_{i=0}^{1}\Eval_i(k_i,x)=f_{\alpha,\beta}(x)\right]=1$$
      \item \textbf{Security:} Consider the following semantic security challenge experiment for corrupted server $i\in\{0,1\}$:
      \begin{enumerate}
          \item The adversary produces two $t$-point function descriptions $(\hat{f}^0=(N,\hat\GG,t,A_0,B_0),\hat{f}^1=(N,\hat\GG,t,A_1,B_1))\leftarrow\mathcal{A}(1^\lambda)$, where $\alpha_i\in[N]$ and $\beta_i\in\GG$.
          \item The challenger samples $b\gets\{0,1\}$ and $(k_0,k_1)\leftarrow\Gen(1^\lambda,\hat{f}^b)$.
          \item The adversary outputs a guess $b'\leftarrow\mathcal{A}(k_i)$.
      \end{enumerate}
      Denote by $\Adv(1^\lambda,\mathcal{A},i)=\Pr[b=b']-1/2$ the advantage of $\mathcal{A}$ in guessing $b$ in the above experiment. For every non-uniform polynomial time adversary $\mathcal{A}$ there exists a negligible function $\nu$ such that $\Adv(1^\lambda,\mathcal{A},i) \le \nu(\lambda)$ for all $\lambda \in \NN$.
  \end{itemize}
  \end{definition}
 
 We will also be interested in applying the evaluation algorithm on \emph{all} inputs. Given a DMPF $(\Gen,\Eval_0,\Eval_1)$, we denote by $\FullEval_i$ an algorithm which computes $\Eval_i$ on every input $x$. Hence, $\FullEval_i$ receives only a key $k_i$ as input.



\subsection{Batch Codes}
combinatorial/probabilistic batch codes, with cuckoo hashing a concrete instantiation

\subsection{Oblivious Key-Value Stores}
\begin{definition}[OKVS\cite{cryptoeprint:2021/883,cryptoeprint:2022/320}]\label{def:OKVS}
  An Oblivious Key-Value Stores (OKVS) scheme is a pair of randomized algorithms $(\Encode_r,\Decode_r)$ with respect to a statistical security parameter $\lambda_{\sf stat}$ and a computational security parameter $\lambda$, a randomness space $\{0,1\}^\kappa$, a key space $\mathcal{K}$, a value space $\mathcal{V}$, input length $n$ and output length $m(n)$. The algorithms are of the following syntax: 
  \begin{itemize}
    \item $\Encode_r(\{(k_1,v_1),(k_2,v_2),\cdots,(k_n,v_n)\})\rightarrow P$: On input $n$ key-value pairs with distinct keys, the encode algorithm with randomness $r$ in the randomness space outputs an encoding $P\in\mathcal{V}^m\cup\bot$.
    \item $\Decode_r(P,k)\rightarrow v$: On input an encoding from $\mathcal{V}^m$ and a key $k\in\mathcal{K}$, output a value $v$. 
  \end{itemize}
  We require the scheme to satisfy
  \begin{itemize}
    \item \textbf{Correctness: }For every $S\in(\mathcal{K}\times\mathcal{V})^n$, $\Pr_{r\leftarrow\{0,1\}^\kappa}[\Encode_r(S)=\bot]\le 2^{-\lambda_{\sf stat}}$. 
    \item \textbf{Obliviousness: }Given any distinct key sets $\{k_1^0,k_2^0,\cdots,k_n^0\}$ and $\{k_1^1,k_2^1,\cdots,k_n^1\}$ that are different, if they are paired with random values then their encodings are computationally indistinguishable, i.e., 
  \begin{align*}
    &\{r, \Encode_r(\{(k_1^0,v_1),\cdots,(k_n^0,v_n)\})\}_{v_1,\cdots,v_n\leftarrow \mathcal{V},r\leftarrow\{0,1\}^\kappa}\\
    \approx_c &\{r, \Encode_r(\{(k_1^1,v_1),\cdots,(k_n^1,v_n)\})\}_{v_1,\cdots,v_n\leftarrow \mathcal{V},r\leftarrow\{0,1\}^\kappa}
  \end{align*}
  \end{itemize}
One can obtain a \emph{linear OKVS} if in addition require:
\begin{itemize}
  \item \textbf{Linearity: }There exists a function family $\{\row_r:\mathcal{K}\rightarrow\mathcal{V}^m\}_{r\in\{0,1\}^\kappa}$ such that $\Decode_r(P,k) = \ipd{\row_r(k)}{P}$. 
\end{itemize}
\end{definition}
The $\Encode$ process for a linear OKVS is the process of sampling a random $P$ from the set of solutions of the linear system $\{\ipd{\row_r(k_i)}{P} = v_i\}_{1\le i\le n}$. 

We evaluate an OKVS scheme by its encoding size (output length $m$), encoding time and decoding time. We stress the following two (linear) OKVS constructions:
\begin{construction}[Polynomial]
  Suppose $\mathcal{K} = \mathcal{V}=\FF$ is a field. Set 
  \begin{itemize}
    \item $\Encode(\{(k_i,v_i)\}_{1\le i\le n}) \rightarrow P$ where $P$ is the coeffients of a $(n-1)$-degree $\FF$-polynomial $g_P$ that $g_P(k_i) = v_i$ for $1\le i\le n$. 
    \item $\Decode(P,k)\rightarrow g_P(k)$. 
  \end{itemize}
\end{construction}
The polynomial OKVS possesses an optimal encoding size $m=n$, but the $\Encode$ process is a polynomial interpolation which is only known to be achieved in time $O(n\log^2n)$. The time for a single decoding is $O(n)$ and that for batched decodings is (amortized) $O(\log^2 n)$. 

An alternative construction that has near optimal encoding size but much better running time is as follows. 
\begin{construction}[3-Hash Garbled Cuckoo Table (3H-GCT)\cite{cryptoeprint:2021/883,cryptoeprint:2022/320}]
  Suppose $\mathcal{V}=\FF$ is a field. Set $\row_r(k):=\row_r^{\sf sparse}(k)||\row_r^{\sf dense}(k)$ where $\row_r^{\sf sparse}$ outputs a uniformly random weight-$w$ vector in $\{0,1\}^{m_1}$, and $\row_r^{\sf dense}(k)$ outputs a short dense vector in $\FF^{m_2}$. 
  \begin{itemize}
    \item $\Encode(\{(k_i,v_i)\}_{1\le i\le n}) \rightarrow P$ where $P$ is solved from the system $\{\ipd{\row_r(k_i)}{P} = v_i\}_{1\le i\le n}$ using the triangulation algorithm in \cite{cryptoeprint:2022/320}. 
    \item $\Decode(P,k)\rightarrow \ipd{\row_r(k)}{P}$. 
  \end{itemize}
\end{construction}
This OKVS construction features a linear encoding time, constant decoding time while having a linear encoding size. 

TBD: Carefully(!) recompute the comparison table for OKVS and insert

We take $w=3$, the most common option that outruns other choices of $w$ in terms of running time. Restating the conclusion in  \cite{cryptoeprint:2022/320}: given $n$ and $\lambda_{\sf stat}$, the choices of $e$ and $\hat{g}$ are $e = 1.223+\frac{\lambda_{\sf stat}+9.2}{4.144 n^{0.55}}$ and $\hat{g}=\frac{\lambda_{\sf stat}}{\log_2(en)}$. 

TBD: mention some connections to cuckoo hashing
