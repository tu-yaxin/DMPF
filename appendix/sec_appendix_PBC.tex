\section{Batch Codes}\label{sec:batch_codes}
We introduce probabilistic batch codes, which permit small decoding errors. Looking ahead, these codes can be used to construct DMPF (see \cref{con:DMPF_PBC_detail}). 
\begin{definition}[Probabilistic Batch Code (PBC)\cite{cryptoeprint:2017/1142,10.1145/1007352.1007396,cryptoeprint:2022/1455}]\label{def:PBC}
  An $(N,M,t,m,l,\epsilon)$-$\PBC$ over alphabet $\Sigma$ is given by a pair of efficient algorithms $(\Encode,\Decode)$ with public randomness $r$ such that:
  \begin{itemize}
    \item $\Encode_r(x\in\Sigma^N)\rightarrow (C_1,C_2,\dots,C_m)$: Any string $x\in\Sigma^N$ is encoded into $m$ codewords (or `buckets') $C_1,C_2,\cdots C_m\in\Sigma^*$ of total length $M$.
    \item $\Decode_r(I,C_1,C_2,\dots,C_m)\rightarrow x[I]$: On input a set $I\subseteq[N]$ of $\le t$ distinct elements in $[N]$ and $m$ codewords, recover the subset $x[I]$ of $x$ indexed by $I$, while querying at most $l$ positions in each codeword. 
    \item \textbf{Correctness}: for any string $x$ and any set $I$ of $t$ distinct indices in $[N]$, 
    \[
    \begin{split}
      \Pr_r[&(C_1,\dots, C_m)\gets \Encode_r(x), \\
      &x[I]\not= \Decode_r(I,C_1,\dots,C_m)] \le \epsilon
    \end{split}
    \]
  \end{itemize}
  By default, we assume the batch code to be systematic, which means that each symbol of $x$ is encoded to some fixed positions in the buckets. This is formalized by the two sub-processes $\Encode_r$ and $\Decode_r$ respectively: 
  \begin{itemize}
    \item $\Position_r(k\in[N])\rightarrow C_{i_1}[j_1],C_{i_2}[j_2],\dots$: On input an index $k\in[N]$, output the sequence of positions in buckets that $x[k]$ is encoded to. 
    \item $\Schedule_r(k\in I)\rightarrow C_{i_1}[j_1],C_{i_2}[j_2],\dots$: For any $I\subseteq [N]$ such that $|I|\le t$, and $k\in I$, $\Schedule_r(k)$ outputs a set of positions in buckets relevant to $k$ that $\Decode_r$ reads when decoding to $x[I]$. For all $i\in [m]$, $|C_i\cap \bigcup_{k\in I}\Schedule_r(k)|\le l$. 
  \end{itemize}
\end{definition}

We will focus on the case $l=1$ and a special class of batch codes called Combinatorial Batch Codes (CBC)\cite{cryptoeprint:2017/1142,10.1145/1007352.1007396,cryptoeprint:2008/306}, where each codeword $C_i$ is a subset of $x$. In this case, $\Encode_r$ simply sends $x[k]$ to the positions defined by $\Position_r(k)$, and $\Decode_r$ recovers $x[I]$ by reading the symbols in positions $\bigcup_{k\in I}\Schedule_r(k)$ and rearranging the symbols it reads. Note that when $l=1$, $\Schedule$ algorithm implies finding a prefect matching from the size-$t$ subset $I\subseteq[N]$ to the $m$ buckets, in an $(N,m)$-bipartite graph where $k\in[N]$ is connected to $j\in[m]$ if and only if $x[k]$ is contained in $C_j$. 

A natural way to construct PBC is to define the allocation of symbols in $x$ to the buckets by a random $(N,m)$-bipartite graph where each left node has degree $w$ for a fixed parameter $w$. To implement $\Encode$ and $\Decode$ (or more specifically $\Position$ and $\Schedule$), we use the $w$-way cuckoo hashing algorithm\cite{10.1007/3-540-44676-1_10} as a concrete and efficient instantiation of PBC as in \cite{cryptoeprint:2017/1142,cryptoeprint:2018/579,cryptoeprint:2022/1455}. 

\paragraph{$w$-way cuckoo hashing algorithm}Given $t$ balls, $m=et$ buckets ($e$ is an expansion parameter that is bigger than 1), and $w$ independent random hash functions $h_1, h_2,\cdots, h_w$, each mapping the balls to the buckets, a $w$-way cuckoo hashing algorithm we describe here aims to allocate $t$ balls to $m$ buckets such that each ball is allocated to one of the buckets output by the $w$ hash functions, and each bucket contains at most one ball through the following process: 
\begin{itemize}
  \item[1.] Choose an arbitrary unallocated ball $b$. If there is no unallocated ball, output the allocation. 
  \item[2.] Choose a random hash function $h_i$, and  compute the bucket index $h_i(b)$. If this bucket is empty, then allocate $b$ to this bucket and go to step 1. If this bucket is not empty and filled with ball $b'$, then evict $b'$, allocate $b$ to this bucket,  and repeat step 2 with unallocated ball $b'$. 
\end{itemize}
If the algorithm terminates then it outputs a desired allocation of balls to buckets. To prevent it from running forever, one may set a fixed amount of running time. We call it a \emph{failure} whenver the algorithm fails to output a desired allocation where each bucket contains at most one ball. We'll next summarize known asymptotic and empirical results about the failure probability of cuckoo hashing. 

\paragraph{The failure probability of cuckoo hashing.}Denote the failure probability of $w$-way cuckoo hashing to be $\epsilon=2^{-\lambda_{\stat}}$. The empirical result from \cite[Appendix B]{cryptoeprint:2018/579} shows that when $w=3$ and $t\ge 4$, the failure probability is computed by 
\[\begin{split}
  \lambda_\stat &= a_t\cdot e - b_t-\log t\\
  a_t &= 123.5\cdot {\sf CDF_{Normal}}(x=t, \mu = 6.3, \sigma = 2.3)\\
  b_t =& 130\cdot {\sf CDF_{Normal}}(x=t, \mu = 6.45, \sigma = 2.18)
\end{split}
\] where $e = m/t$ is the expansion parameter and ${\sf CDF_{Normal}}(x,\mu,\sigma)$ is the cumulative distribution function on input $x$ for the normal distribution with expectation $\mu$ and standard deviation $\sigma$. The number of buckets $m$ that ensures at least $1-\epsilon$ success probability can be computed according to $\lambda_\stat$ and $t$. As an example, we set the statistical security parameter $\lambda_{\stat}=40$. When $t\ge 12$, the $\sf CDF_{Normal}$ values are $\ge 0.99$ and we can compute $e = \frac{170+\log t}{123.5}$ and $m = e\cdot t$ according to the concrete value of $t$. 

\begin{remark}
    There are different ways to generalize cuckoo hashing algorithm in order to achieve negligible failure probability. For instance, one may allow at most $l>1$ balls to be allocated to each bucket, instead of allowing at most one. One may also add an overflow stash with size $s$ as an additional cache-like bucket \cite{KMW10}. We mostly use cuckoo hashing with $l=1$ and $s=0$ to construct PBC and DMPF, but will also point out how general cuckoo hashing may help with these constructions.
\end{remark}

Next we present the instantiation of PBC using cuckoo hashing. 

%With the $t$ balls replicated and allocated to $m$ buckets, the cuckoo hashing algorithm essentially finds a perfect matching from $t$ balls to $m$ buckets, which coincides with the form of (probabilistic) CBC decoding. Therefore a PCBC follows directly from a cuckoo hashing scheme: 
%\Yaxin{Dec 31: The following construction is mentioned in \cite{cryptoeprint:2022/1455}. There are several points to note: 

%(1) \cite{cryptoeprint:2022/1455} modified the hash functions' domain in the following way: it divides the $m$ buckets evenly to $w$ blocks, and for $1\le i\le w$, $h_i:[N]\rightarrow [m/w]$ maps an element to a bucket in the $i$th block. The paper does this to claim better asymptotic provable success probability of cuckoo hashing, but using superconstant number ($w = \lambda_\stat/\log\log N$) of hash functions, which does not align with empirical results that suggests constant number (say 3) of hash functions. I think to us this means that if making $h_i$ to map to the $i$th block could be useful in implementation somehow (although I doubt this), then it also makes sense to do this modification. 

%(2) (Resolved) It should be mentioned that both the capacity of cuckoo-hashing bins (which is 1 here) and the number of lookup in each $C_i$ that PCBC is allowed (also 1 here) can be simultaneously generalized to any number $l$ along with different parameters and overheads, but the paper still applied only $l=1$ case to applications like batch PIR, and I haven't seen any efficient empirical parameters and results for $l>1$ setting. However it is plausible to use general $l$ along with $O(t/l)$ buckets, each expanded to a $\DMPF_l$ truth table. It may be mentioned as a future direction in the end. }

\begin{construction}[PBC from cuckoo hashing\cite{cryptoeprint:2017/1142,cryptoeprint:2021/580}]\label{con:PBC_from_cuckoo}
  Given $w$-way cuckoo hashing as a sub-procedure allocating $t$ balls to $m$ buckets with failure probability $\epsilon$, an $(N,wN,t,m,1,\epsilon)$-$\PBC$ is as follows: 
  \begin{itemize}
    \item $\Encode_r(x\in\Sigma^N)\rightarrow (C_1,\cdots,C_m)$: For all $k\in[N]$, replicate $x[k]$ in the positions indicated by $\Position_r(k)$. %Each output bucket $C_j$ will be $\{x[k]:h_i(k) = j$ for some $i\in [w]\}$, in ascending order of $k$. 
    \item $\Decode_r(I, C_1,\cdots, C_m)\rightarrow \{x[i]\}_{i\in I}$: For each $k\in I$, obtain $x[k]$ from the position indicated by $\Schedule_r(i)$. 
  \end{itemize}
  with the following sub-processes: 
  \begin{itemize}
    \item $\Position_r(k\in[N])\rightarrow C_{h_1(k)}[j_1],\dots,C_{h_w(k)}[j_w]$: Interpret $r$ as a (pseudo-)random permutation $P:[w]\times [N]\rightarrow [m]\times [B]$ where $B = wN/m$, such that for each $l\in[w]$ and $k\in[N]$, $P(l,k) = (i,j)$ indicating the position $C_i[j]$. Then $\Position_r(k)$ outputs $\{C_i[j]: (i,j)\gets P(l,k)\}_{l\in w}$
    \item $\Schedule_r$: Interpret $r$ as the (pseudo-)random permutation $P:(l,k)\mapsto (i,j)$ for $l\in[w]$ and $k\in[N]$. For $l\in[w]$, let $h_l:[N]\rightarrow [m]$ be $h_l(k) = i$ where $P(l,k) = (i,j)$. Then we have $w$ random hash functions $h_1,h_2,\cdots h_w$ that maps from $[N]$ to $[m]$.
    For $I$ of size at most $t$, run the $w$-way cuckoo hashing algorithm to allocate the indices in $I$ to $m$ buckets, such that $k\in I$ is allocated to the bucket $h_{l_k}(k)$, and each bucket contains at most one index in $I$. In the end, for $k\in I$, $\Schedule_r(k)$ outputs the position $C_i[j]$ such that $P(l_k,k) = (i,j)$. 
  \end{itemize}
\end{construction}
\begin{remark}
    By using the pseudorandom permutation $P:[w]\times [N]\rightarrow [m]\times [B]$, it is direct to obtain the positions in buckets where any $k\in[N]$ is encoded to, however the hash functions $h_1,\dots,h_w$ are not completely independent. Nevertheless, it is empirically verified in \cite{cryptoeprint:2021/580} that the cuckoo hashing algorithm still succeeds with sufficient probability. 
    %Computing $\Position_r$ and $\Schedule_r$ requires computing the order of some index $k\in[N]$ in a bucket $C_j$ it is mapped to, which may be inefficient when $N$ is large. address this issue by implementing $w$ hash functions by a single \emph{(pseudo-)random permutation} $P$ mapping from $[w]\times [N]$ to $[m]\times [B]$, where $B = wN/m$. Invocation of $h_i(j)$ is done by computing $P(i,j)$, which outputs the bucket number in $[m]$ and the index in $[B]$. Note that in this case $h_1,\dots,h_w$ are not independent random hash functions, but it is empirically verified 
\end{remark}
The probability of the above PBC scheme being incorrect is equal to the failure probability of $\Schedule_r$ on set $I$, which equals the cuckoo hashing failure probability $\epsilon$. 

\subsection{DMPF Construction from PBC}
Next we present the construction of DMPF from black-box usage of DPF basing on PBC with appropriate parameters, which has been discussed in previous literature \cite{cryptoeprint:2019/273,cryptoeprint:2019/1084,cryptoeprint:2021/580,foleage}. 
\begin{construction}[PBC-based DMPF]\label{con:DMPF_PBC_detail}
  Given $\DPF$ for any domain of size $\le N$ and output group $\GG$, and an $(N,M,t,m,1,\epsilon)$-$\PBC$ (which is systematic and combinatorial) with alphabet $\Sigma=\GG$, we can construct a $\DMPF$ for $t$-point functions with domain size $N$ and output group $\GG$ as follows: 
  \begin{itemize}
    \item $\Gen(1^\lambda, \hat{f}_{A,B})\rightarrow (k_0,k_1)$: Suppose $A=\{\alpha_1,\cdots,\alpha_t\}$ and $B=\{\beta_1,\cdots,\beta_t\}$. Suppose the $\PBC$ encodes to buckets $C_1,\dots,C_m$. Compute $\PBC.\Schedule(\alpha_k)\rightarrow C_{i_k}[j_k]$, allocating elements in $A$ to positions in the $m$ buckets.     
    %If $\PBC.\Schedule$ fails then run it again with a fresh public randomness. 
    
    For all $1\le k\le t$ For $1\le i\le m$, let $f_i:[|C_i|]\rightarrow \GG$ be the following: 
    \begin{itemize}
      \item If there is no such $k\in[t]$ that $i_k = i$, then set $f_i$ to be the all-zero function. 
      \item If there is exactly one $k\in[t]$ that  $i_k = i$, then set $f_i$ to be the point function that outputs $\beta_k$ on $j_k$ and 0 elsewhere. 
    \end{itemize}
    For $1\le i\le m$, invoke $\DPF.\Gen(1^\lambda, \hat{f}_i)\rightarrow (k_0^i,k_1^i)$. Set $(k_0,k_1)=(\{k_0^i\}_{i\in [m]}, \{k_1^i\}_{i\in [m]})$. 
    \item $\Eval_b(k_b,x)\rightarrow y_b$: Invoke $\PBC.\Position(x)$ to obtain the positions $C_{i_1}[j_1],\dots,C_{i_s}[j_s]$ to which $x$ is sent. Compute $y_b=\sum_{l=1}^s\DPF.\Eval_b(k_b^{i_l},j_l)$. 
    \item $\FullEval_b(k_b)\rightarrow Y_b$: Compute $Y_b^i = \DPF.\FullEval_b(k_b^i)$ for $1\le i\le m$. For all $x\in [N]$, invoke $\PBC.\Position(x)\rightarrow C_{i_1}[j_1],\dots,C_{i_s}[j_s]$, and set $Y_b[x]\gets \sum_{l=1}^sY_b^{i_l}[j_l]$. 
  \end{itemize}
The scheme is correct with at least $1-\epsilon$ probability and has distinguishing advantage $O(\epsilon)$. 
\end{construction}

When instantiating PBC using $w$-way cuckoo hashing as in \Cref{con:PBC_from_cuckoo}, the \emph{evaluation time} is the sum of the time for $\PBC.\Position(x)$ plus the time for $w$ invocations of $\DPF.\Eval$. Similarly, the \emph{full-domain evaluation time} is roughly the time for $N$ invocations of $\PBC.\Position$ plus $w$ invocations of $\DPF.\FullEval$. Therefore, one may expect the evaluation time for the above construction of DMPF to be dependent to $w$ instead of $t$. We will discuss about its efficiency in later sections. 
%the \emph{key generation time} is roughly the sum of time for computing cuckoo hashing algorithm, the time for finding $t$ indices for $t$ elements in the buckets, plus the total time of all DPF$.\Gen(1^\lambda, \hat{f}_i)$. The 
\begin{remark}
  As a future direction, it is intriguing to try to use a $(N,M,t,m,l,\epsilon)$-$\PBC$ where $l>1$ (say, instantiated by an $w$-way cuckoo hashing with bucket size $l$) to construct a DMPF scheme using $l$-DMPF with adjustable $l$, where the primitive $l$-DMPF can be implemented by any one of the DMPF schemes we discuss about in this paper. 
\end{remark}