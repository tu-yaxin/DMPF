\subsection{Pseudorandom Correlation Generator}\label{sec:PCG_prelim}
In this section we introduce pseudorandom correlation generator (PCG) as well as the protocol of PCG for OLE correlation from Ring-$\LPN$ assumption proposed in \cite{cryptoeprint:2022/1035}. The PCG protocol utilizes DPF to secret share the coefficients of a sparse polynomial (who has a small number of nonzero coefficients), which can alternatively be done by DMPF. 

A pseudorandom correlation generator (PCG, \cite{cryptoeprint:2019/273,cryptoeprint:2019/448}) is a primitive that generates and distributes a pair of short seeds, which can then be locally expanded to produce correlated pseudorandomness. The security of PCG is defined using the notions of correlation generators, and reverse-sampleable correlation generators, from \cite{cryptoeprint:2019/448}. 

\begin{definition}[Correlation generator]\label{def:correlation_generator}A p.p.t. algorithm $\cC$ is called a \emph{correlation generator}, if $\cC$ on input $1^\lambda$ outputs a pair of elements in $\{0, 1\}^n \times \{0, 1\}^n$ for $n \in \poly(\lambda)$. 
\end{definition} 

\begin{definition}[Reverse-sampleable correlation generator]\label{def:reverse-sampleable_correlation_generator)}
    Let $\cC$ be a correlation generator. We say $\cC$ is \emph{reverse sampleable} if there exists a p.p.t. algorithm $\sf RSample$ such that for $\sigma \in \{0, 1\}$ the correlation obtained via: \[
    \{(R'_0, R'_1) :(R_0, R_1)\gets \cC(1^\lambda), R'_\sigma := R_\sigma, R'_{1-\sigma}\gets {\sf RSample}(\sigma, R_\sigma)\}
    \]
    is computationally indistinguishable from $\cC(1^\lambda)$.
\end{definition}
 The following definition of pseudorandom correlation generators can be viewed as a generalization of the definition of the pseudorandom VOLE generator in [BCGI18]. 
 \begin{definition}[Pseudorandom Correlation Generator (PCG)]
     Let $\cC$ be a reverse-sampleable correlation generator. A pseudorandom correlation generator (PCG) for $\cC$ is a pair of algorithms $(\PCG.\Gen, $\linebreak$\PCG.\Expand)$ with the following syntax: 
     \begin{itemize}
         \item $\PCG.\Gen(1^\lambda)$ is a p.p.t. algorithm that given a security parameter $\lambda$, outputs a pair of seeds $(k_0,k_1)$.
         \item $\PCG.\Expand(\sigma, k_\sigma)$ is a polynomial-time algorithm that given a party index $\sigma\in\{0,1\}$ and a seed $k_\sigma$, outputs a bit string $R_\sigma\in\{0,1\}^n$.
     \end{itemize} 
     The algorithms $(\PCG.\Gen, \PCG.\Expand)$ should satisfy the following:
     \begin{itemize}
         \item \textbf{Correctness: }The correlation obtained via: \[
         \begin{split}
             \{(R_0, R_1):\,(k_0, k_1) &\gets \PCG.\Gen(1^\lambda), \\R_\sigma&\gets \PCG.\Expand(\sigma, k_\sigma) \text{ for } \sigma\in\{0,1\}\}
         \end{split}\] is computationally indistinguishable from $C(1^\lambda)$.
         \item \textbf{Security: }For any $\sigma\in\{0,1\}$, the following two distributions are computationally indistinguishable:
         \[\{(k_{1-\sigma}, R_\sigma): (k_0, k_1) \gets \PCG.\Gen(1^\lambda),R_\sigma\gets \PCG.\Expand(\sigma, k_\sigma)\}\]
         and
         \[
         \begin{split}
            \{(k_{1-\sigma}, R_\sigma): (k_0, k_1)&\gets \PCG.\Gen(1^\lambda),\\R_{1-\sigma}&\gets \PCG.\Expand(\sigma, k_{1-\sigma}),\\R_\sigma&\gets{\sf RSample}(\sigma, R_{1-\sigma})\}
         \end{split}
         \]
         where $\sf RSample$ is the reverse sampling algorithm for correlation $\cC$.
     \end{itemize}
 \end{definition} 
To avoid the trivial solution where $\PCG.\Gen$ simply outputs a sample from $\cC$, we are only interested in constructions where the seed size is significantly shorter than the output size.

\subsubsection{PCG for OLE correlation from Ring-$\LPN$ assumption \cite{cryptoeprint:2022/1035}}\label{sec:PCG_OLE_protocol}
Next we introduce the construction of PCG for the OLE relation basing on Ring-$\LPN$ assumption, proposed in \cite{cryptoeprint:2022/1035}. The OLE relation over any ring $R$ is the following: 
\[
\{\left((x_0, z_0),(x_1,z_1)\right): x_0,x_1,z_0\gets R, z_1 = x_0\cdot x_1 - z_0\}
\]

The hardness assumption made use of is a variant of Ring-$\LPN$ assumption, called module-$\LPN$: 
\begin{definition}[Module-$\LPN$]\label{def:module-LPN}
    Let $c\ge 2$ be an integer, $R = \ZZ_p[X]/F(X)$ for a prime $p$ and a deg-$N$ polynomial $F(X)\in \ZZ_p[X]$, and $\mathcal{HW}_{R,t}$ be the uniform distribution over `weight-$t$' polynomials (who has $t$ nonzero coefficients) in $R$ whose degree is less than $N$ and has at most $t$ nonzero coefficients. 
     For $R=R(\lambda)$, $t=t(\lambda)$ and $m=m(\lambda)$, we say that the module-$\LPN$ problem $R^c$-$\LPN$ is hard if for every nonuniform polynomial-time probabilistic distinguisher $\cA$, it holds that 
    \begin{align*}
        |&\Pr[\cA(\{\vec{a}^{(i)}, \ipd{\vec{a}^{(i)}}{\vec{s}}+\vec{e}^{(i)}\}_{i\in [m]})] \\
        -&\Pr[\cA(\{\vec{a}^{(i)},\vec{u}^{(i)}\}_{i\in [m]})]| \le \mathsf{negl}(\lambda)
    \end{align*}
    where the probabilities are taken over the randomness of $\cA$, random samples $\vec{a}^{(1)},\cdots, \vec{a}^{(m)}\leftarrow R^{c-1}$, $\vec{u}^{(1)},\cdots, \vec{u}^{(m)}\leftarrow R$, $\vec{s}\leftarrow \cHW_{R,t}^{c-1}$, and $\vec{e}^{(1)},\cdots, \vec{e}^{(m)}\leftarrow \cHW_{R,t}$. 

    When we only consider $m=1$, each $R^c$-$\LPN$ instance $ \ipd{\vec{a}}{\vec{s}}+\vec{e}$ can be restated as $\ipd{\vec{a'}}{\vec{e'}}$ where $\vec{a'}=1||\vec{a}$ and $\vec{e'}\leftarrow \cHW_{R,t}^c$. 
\end{definition}

For a polynomial ring $R = \ZZ_p[X]/F(X)$ for a prime $p$ and a deg-$N$ polynomial $F(X)\in \ZZ_p[X]$, the PCG protocol in \cite{cryptoeprint:2022/1035} generates and expands seeds for the OLE correlation $(x_0,z_1),(x_0,z_1)$ over $R$ in the following way: 
\begin{itemize}
    \item Fix public random inputs $\vec{a} = (1, a_1,a_2,\dots,a_{c-1})$ where $a_1,\dots,a_{c-1}$ are random polynomials from $R$. 
    \item $\Gen(1^\lambda)\rightarrow (k_0,k_1)$: For $\sigma = 0,1$, sample $\vec{e_\sigma}\gets \cHW^c_{R,t}$. Then $\vec{e_0}\otimes \vec{e_1}$ is a vector of $c^2$ entries, each of which is a weight-$t^2$ deg-$(2N-2)$ polynomial who is the multiplication of two weight-$t$ deg-$(N-1)$ polynomials. Invoke $\DMPF.\Gen$ to get the secret sharing keys $k_0^{i},k_1^i$ for the $i$th entry ($1\le i\le c^2)$ that shares the $t^2$ nonzero coefficients among $2N$ coefficients. For $\sigma = 0,1$, set $k_\sigma$ to be the description of $\vec{e_\sigma}$ and the set of $\DMPF$ keys $(k_\sigma^i)_{i\in[c^2]}$. 
    \item $\Expand(\sigma, k_\sigma)$: Obtain $\vec{e_\sigma}$ and $(k_\sigma^i)_{i\in[c^2]}$ from $k_\sigma$. Compute a length-$c^2$ vector $\vec{u_\sigma}$, where the $i$th entry is a deg-$(2N-2)$ polynomial whose coefficients are from $\DMPF.\FullEval(\sigma, k_\sigma^i)$. Set $x_\sigma = \ipd{\vec{a}}{\vec{e_\sigma}}$, and $z_\sigma = \ipd{\vec{a}\otimes\vec{a}}{\vec{u_\sigma}}$. 
\end{itemize}
Note that since $\ipd{\vec{a}}{ \vec{e_0}}\cdot \ipd{\vec{a}}{\vec{e_1}} = \ipd{\vec{a}\otimes \vec{a}}{\vec{e_0}\otimes \vec{e_1}} = \ipd{\vec{a}\otimes\vec{a}}{\vec{u_0}} + \ipd{\vec{a}\otimes\vec{a}}{\vec{u_1}}$, the pairs $(x_0,z_0)$ and $(x_1,z_1)$ computed by $\Expand$ satisfy OLE correlation over $R$. 

\begin{remark}[From OLE over $R$ to OLE over $\ZZ_p$]\label{rem:use_reducible_ring}
    One can convert an OLE correlation over ring $R$ to \textbf{$N$ OLE correlations over $\ZZ_p$} if the polynomial $F(X)$ splits into $N$ distinct linear factors modulo $p$\cite{cryptoeprint:2022/1035}. The security analysis also applies to this choice of $F$.  
\end{remark}

For the sake of computational efficiency, in the sequel we will consider $F$ being a cyclotomic polynomial $F(X) = X^N+1$ where $N$ is a power of two. In this case each entry of $\vec{e_0}\otimes \vec{e_1}$ can be reduced to a deg-$(N-1)$ polynomial with weight at most $t^2$, and can be shared using $\DMPF$ with domain size $N$. 

 \paragraph{Using the regular noise distribution to split the $\DMPF$ domain}
A previous optimization in \cite{cryptoeprint:2022/1035}, aiming to share the entries of $\vec{e_0}\otimes \vec{e_1}$ more efficiently through $\DPF$, is to substitute $\cHW_{R,t}$ with the distribution of random regular weight-$t$ deg-$(N-1)$ polynomials denoted as $\cRHW_{R,t}$. Each regular weight-$t$ polynomial $e$ contains exactly one nonzero coefficient $e_j$ in the range of degree $[j\cdot N/t, (j+1)\cdot N/t-1]$ for $j=0,\cdots t-1$. When multiplying two regular weight-$t$ polynomials $e$ and $f$, $e_i\cdot f_j$ contributes to a coefficient in the range of degree $[(i+j)\cdot N/t, (i+j+2)\cdot N/t-2]$. When $F(X) = X^N+1$ is cyclotomic, for $i+j\ge t$ this range is reduced to the range of degree $[(i+j-t)N/t, (i+j+2-t)N/t-2]$. Therefore the deg-$(2N-2)$ polynomial $e\cdot f$ can be shared by invoking $t$ $\DMPF$'s corresponding to the ranges of degree $\{[k\cdot N/t, (k+2)\cdot N/t-2]\}_{0\le k<t}$ in $e\cdot f$, such that each $\DMPF$ has $t$ accepting inputs, domain size $N$ and output group $\ZZ_p$. In \cite{cryptoeprint:2022/1035}, these $\DMPF$'s are realized by summing up $t$ DPF's. In \Cref{sec:implementation} we'll discuss how realizing $\DMPF$ in different ways can accelerate the PCG seed expansion. 

%Then each $\DMPF^{(k)}$ in the set $\{\DMPF^{(k)}\}_{1\le k\le 2t-1}$ can be instantiated by one of the mentioned schemes: sum of $\DPF$s (used in \cite{cryptoeprint:2022/1035}), CBC-based, big-state, or $\OKVS$-based $\DMPF$. We note that the efficiency (in terms of $\FullEval$ time) of all these instantiations are more or less related to the number of nonzero inputs in the targeted $\DMPF$, therefore using $\cRHW_{R,t}$ instead to $\cHW_{R,t}$ reduces the number of nonzero inputs from $t^2$ to at most $t$ may be beneficial in some occasions.

%An alternative way to split $\DMPF_{t^2, 2N, \ZZ_p}$, basing on the previous observation, is to share the coefficients of $e\cdot f$ by invoking $\{\DMPF'^{(k)} = \DMPF_{2\min(k+1, 2t-k)-1, N/t, \ZZ_p}\}_{0\le k\le 2t-1}$, where $\DMPF'^{(k)}$'s domain corresponds to the coefficients in the range of degree $[kN/t, (k+1)N/t -1]$. Since the number of nonzero coefficients in $[kN/t, (k+1)N/t -1]$ is upperbounded by the sum of number of nonzero coefficients in $[(k-1)N/t, (k+1)N/t -2]$ and in $[kN/t, (k+2)N/t -2]$, $\DMPF'^{(k)}$ has at most $2\min(k+1, 2t-k)-1$ nonzero inputs. The advantage of using $\{\DMPF'^{(k)}\}_{0\le k\le 2t-1}$ is that the previous concatenation through $\{\DMPF^{(k)}\}_{1\le k\le 2t-1}$ creates overlapping ranges, which doubles the number of PRG invocations when realizing by CBC-based, big-state, and $\OKVS$-based $\DMPF$. By using $\{\DMPF'^{(k)}\}_{0\le k\le 2t-1}$, the ranges are not overlapping and therefore maintains the minimal PRG invocations, while also preserving a relatively small number of nonzero inputs (at most $2t-1$) in each $\DMPF'^{(k)}$. 

%Previously, \cite{cryptoeprint:2022/1035} uses sum of DPFs to instantiate $\{\DMPF^{(k)}\}_{1\le k\le 2t-1}$ in the first optimized design with regular noise distribution. It indicates using batch code to achieve DMPF as another optimization but not in the clear. We'll analyze the cost of this PCG protocol under the following settings:
%\begin{itemize}
   % \item[(1)]with noise distribution $\cHW_{R,t}$ and each multiplication of sparse polynomials is shared by $\DMPF_{t^2, 2N, \mathbb{Z}_p}$
    %\item[(2)]ith noise distribution $\cRHW_{R,t}$ and each multiplication of regular sparse polynomials is shared by $$\{\DMPF^{(k)} = \DMPF_{\min(k,2t-k), 2N/t, \ZZ_p}\}_{1\le k\le 2t-1}$$
    %\item[(3)]ith noise distribution $\cRHW_{R,t}$ and each multiplication of sparse polynomials is shared by $$\{\DMPF'^{(k)} = \DMPF_{2\min(k+1, 2t-k)-1, N/t, \ZZ_p}\}_{0\le k\le 2t-1}$$
%\end{itemize}
%\Yaxin{Dec 27: It is also mentioned using more advanced noise distributions to avoid overlapping ranges. For now it remains to give the concrete costs for (1) $\DMPF_{t^2, 2N, \mathbb{Z}_p}$; (2)$\DMPF_{1/2/\dots/t, 2N/t, \ZZ_p}$; (3)$\DMPF_{1/2/\dots/(2t-1), N/t, \ZZ_p}$ under sum of DPFs/CBC-based/big-state/OKVS-based instantiations. }


\paragraph{Security analysis}The tuple of parameters $(N,c,t)$ are set such that the corresponding $R^c$-$\LPN$ problem is secure with computational security parameter $\lambda$, i.e., any attack should have cost $O(2^\lambda)$. Note that to gain computational efficiency, $F(X) = X^N+1$ is set to be a cyclotomic polynomial with $N$ being a power of two, and the noise distribution is set to be $\cRHW_{R,t}$ that is regular. \cite{cryptoeprint:2022/1035} analyzed generic attacks for the $\LPN$ problem as well as attacks just for the ring-$\LPN$ problem, which can both take advantage of the cyclotomic $F$ and the regular noise, and then estimated the lowerbound of number of algebraic field operations used in the best attacks (SD or ISD family) to be $2^i\cdot c^{w_i}$, where \[
w_i=ct-2^ic+((2^i-1)c+ct)\cdot \left(1-2^{-i}\right)^{t-1}
\] is the expected number of total noisy coordinates after taking modulo of the $2^i$th cyclotomic polynomial, and
\[
i:={\arg\min}_{1\le i\le \log N}\left\{i:w_i<2^i\right\}
\]
One can then determine $(N,c,t)$ by following the restriction $2^ic^{w_i} \ge 2^\lambda$. In our application we adjust the lowerbound $2^i c^{w_i}$ to $2^{2.8i}c^{w_i}$ by a tighter estimation in \cite{cryptoeprint:2022/712}, and determine $(N,c,t)$ by the restriction $2^{2.8i}c^{w_i}\ge 2^\lambda$. 
%\Yaxin{In \cite{cryptoeprint:2022/1035} the parameters are taken to be $(\lambda, N, c, t) \in\{ (128, 2^{20}, 8, 5), (128, 2^{20}, 4, 16), (128, 2^{20}, 2, 76)\}$. Using the new lowerbound the setting of parameters becomes $(\lambda, N, c, t) \in\{ (128, 2^{20}, 8, 5), (128, 2^{20}, 4, 14), (128, 2^{20}, 2, 66)\}$. In fact, $N$ is independent to other parameters because the attack only works with the ring $R_i = \ZZ_p[X]/f_i(X)$ where $f_i$ is a factor of $F$. }

%The three parameters are comparable in efficiency in \cite{cryptoeprint:2022/1035} because they used the second $\DMPF$ instantiation (concatenation of DPFs under $\cRHW_t$ distribution) whose cost scales with $c^2t$. By using the new $\DMPF$ instantiations we may see significant difference among the three tuples of parameters. 

%An $R^c$-$\LPN$ instance $a\cdot e$ can be viewed as a (dual-)$\LPN_{cN,N,\cHW_{N,t}^{\otimes c}}$ instance $\{H, H\cdot \vec{e}\}$, where $H\in\ZZ_p^{N\times cN}$ is a concatenation of $c$ circular matrices representing multiplication with ${a}$, and $\vec{e}\in \ZZ_p^{cN}$ with distribution $\cHW_{N,t}^{\otimes c}$ represents the concatenation of coefficients of $e$. The bit security of the $R^c$-$\LPN$ problem is equivalent to the bit security of the described (dual-)$\LPN$ problem. As in \cite{cryptoeprint:2022/1035}, we consider the bit security of the described (dual-)$\LPN$ problem to be the same as the bit security of the standard (dual-)$\LPN$ problem, whose error distribution is $\cHW_{cN, ct}$, the random weight-$ct$ noises. 

\subsection{Private Set Intersection}\label{sec:PSI_prelim}
A private set intersection (PSI) protocol \cite{freedman_efficient_2004} allows two parties with input $X,Y\subseteq [N]$ being two sets of the domain $[N]$ to learn about their intersection $X\cap Y$ without revealing additional information of $X$ or $Y$. We denote by PSI-WCA (weighted cardinality) a variant of PSI that computes the weighted cardinality of elements in $X\cap Y$ where the weights are determined by a weight function $W:N\rightarrow \GG$ where $\GG$ is an Abelian group. 

We will be interested in \emph{unbalanced} PSI-WCA where $|X|\gg |Y|$ and the output should be received by the party holding $Y$. In this problem we call the party holding $X$ as the server, and the party holding $Y$ as the client. If further the big set $X$ is held by two non-colluding servers, then such an unbalanced PSI-WCA protocol can be constructed from DMPF, as suggested in \cite{cryptoeprint:2020/1599}: 
\begin{itemize}
  \item The client invokes $\DMPF.\Gen(1^\lambda, \hat{f}_{Y,W(Y)})\rightarrow (k_0,k_1)$, where $w(Y)$ is the set of weights of elements in $Y$. Then the client send $k_0$ to server 0 and $k_1$ to server 1. 
  \item Server $b$ computes $s_b=\sum_{x\in X}\DMPF.\Eval_b(1^\lambda, k_b,x)$ and send it back to the client. 
  \item The client computes $s_0+s_1$, which will be the weighted cardinality of $X\cap Y$. 
\end{itemize}

This protocol leaks $N,\GG$ as well as the upperbound of $|Y|$ to the servers. Typically we set $N = 2^{128}$, $|X|$ as large as $2^{20}$ and $|Y|$ to be arbitrary with $|Y|\ll |X|$. 

The cost of our unbalanced PSI-WCA can be computed as follows: 
\begin{itemize}
  \item The communication cost is the keysize of $\DMPF_{n_Y,N,\GG}$. 
  \item The client's computation time equals to the running time of $\DMPF_{n_Y,N,\GG}.\Gen$. 
  \item The server's computation time equals to $n_X\times$ $\DMPF_{n_Y,N,\GG}.\Eval$. 
\end{itemize}