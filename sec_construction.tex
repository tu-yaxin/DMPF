\section{New DMPF constructions}
In this section, we display two new constructions of DMPF that follow the same paradigm shown in \cref{fig:DMPF_paradigm}. 

We begin by introducing the DMPF paradigm in \cref{fig:DMPF_paradigm}, which is based on the idea of the DPF construction in \cite{CCS:BoyGilIsh16}. Each key $k_b(b=0,1)$ generated by ${\sf Gen}(\hat{f}_{A,B})$ can span a height-$n$ ($n$ is the input length of $\hat{f}_{A,B}$) complete binary tree $T_b$ (call it the evaluation tree), with which party $b$ can evaluate the input $x=x_1\cdots x_n$ by starting from the root of this tree, on the $i$th layer going left if $x_i=0$ and going right if $x_i=1$, until reaching a leaf node then computing the result according to this leaf node. 

Each node of this tree is associated with a $\lambda$-bit seed and a $l$-bit sign. For a parent node on the $i$th layer with $\seed$ and $\sign$, its children's seeds and signs are generated by $\sf PRG(\seed)\oplus Correction$, where the $\sf Correction$ is determined by the parent node's position, its $\sign$ and a correction word $CW^{(i)}$ associated with that layer (computed by the method $\sf Correct()$). On a leaf node on the last layer, its $\seed$ will generate a random element in the output group, which will be corrected by adding a $\sf Correction$ determined by the leaf node's position, its sign and the last correction word $CW^{(n+1)}$ (computed by the method $\sf ConvCorrect()$). 

Call any path from the root a leaf corresponding to an input string in $A$ an accepting path. To force the correctness, we maintain the following invariance on the evaluation trees $T_0$, $T_1$ of the two parties: 
\begin{itemize}
  \item If a node is not on any accepting path, then $T_0$ and $T_1$ assign to it with the same seed and sign. 
  \item If a node is on an accepting path, then $T_0$ and $T_1$ assign to it with different signs that controls the corrections on its children (or on the output if the node is on the last layer). 
\end{itemize}

The paradigm contains four methods ($\sf GenCW$, $\sf GenConvCW$, $\sf Correct$, $\sf ConvCorrect$) and the sign length $l$ to be determined by different constructions. We make the following restrictions on the methods in order to guarantee the invariance on the evaluation trees: 

${\sf M}(\bar{x},\sign, CW) =\sum_{i=1}^l\sign[i]\cdot{\sf M}(\bar{x},0^{i-1}10^{l-i},CW)$ for all $\sf M\in\{Correct, ConvCorrect\}$, input $\bar{x}$ and $CW$.

\begin{figure*}
  \caption{The paradigm of our DMPF schemes. We leave the PRG expand length $l$, methods $\sf Initialize, GenCW,$ $\sf GenConvCW, Correct, ConvCorrect$ to be determined by specific constructions. }
  \label{fig:DMPF_paradigm}
  \fbox{\parbox{\linewidth}{
  \begin{algorithmic}
    \State \textbf{Public parameters: }
    \State The $t$-point function family $\{f_{A,B}\}$ with $t$ an upperbound of the number of nonzero points, input domain $[N]=\{0,1\}^n$ and the output group $\GG$. 
    \State Suppose there is a public PRG $G:\{0,1\}^\lambda\rightarrow \{0,1\}^{2\lambda+2l}$. Parse $G(x) = G_0(x)\|G_1(x)$ to the left half and right half. 
    \State Suppose there is a public PRG $G_{\sf convert}:\{0,1\}^\lambda\rightarrow \GG$. 
    \item[]
    \Procedure{Gen}{$1^\lambda, \hat{f}_{A,B}$}
    \State Denote $A = (\alpha_1,\cdots,\alpha_t)$ in lexicographically order, $B = (\beta_1,\cdots,\beta_t)$. If $|A|<t$, extend $A$ to size-$t$ with arbitrary $\{0,1\}^n$ strings and $B$ with 0's. 
    \State For $0\le i\le n-1$, let $A^{(i)}$ denote the sorted and deduplicated list of $i$-bit prefixes of strings in $A$. Specifically, $A^{(0)} = [\epsilon]$. 
    \State For $0\le i\le n-1$ and $b=0,1$, initialize empty lists $\seed_b^{(i)}$ and $\sign_b^{(i)}$. 
    \State ${\sf Initialize}(\{\seed_b^{(0)},\sign_b^{(0)}\}_{b=0,1})$. 
    \For{$i=1$ to $n$}
    \State $CW^{(i)}\gets {\sf GenCW}(i,A,\{\seed_b^{(i-1)},\sign_b^{(i-1)}\}_{b=0,1})$. 
      \For{$k = 1$ to $|A^{(i-1)}|$ and $z=0,1$}
        \State Compute $C_{\seed,b}\|C_{\sign^0,b}\|C_{\sign^1,b}\gets {\sf Correct}(A^{(i-1)}[k], \sign_b^{(i-1)}[k], CW^{(i)})$ for $b=0,1$. 
        \If{$A^{(i-1)}[k]\|z\in A^{(i)}$}
        \State Append the first $\lambda$ bit of $G_z(\seed_b^{(i-1)}[k])\oplus(C_{\seed,b}\|C_{\sign^z,b})$ to $\seed_b^{(i)}$ and the rest to $\sign_b^{(i)}$. 
        \EndIf
      \EndFor
    \EndFor
    \State $CW^{(n+1)}\gets{\sf GenConvCW}(A,B,\{\seed_b^{(n)},\sign_b^{(n)}\}_{b=0,1})$. 
    \State Set $k_b \gets (\seed_b^{(0)},\sign_b^{(0)}, CW^{(1)},CW^{(2)},\cdots,CW^{(n+1)})$.
    \State \textbf{return} $(k_0,k_1)$.
    \EndProcedure
    \item[]
    \Procedure{Eval\(_b\)}{$1^\lambda, k_b,x$}
    \State Parse $k_b = ([\seed],[\sign],CW^{(1)},CW^{(2)},\cdots,CW^{(n+1)})$. 
    \State Denote $x=x_1x_2\cdots x_n$. 
    \For{$i = 1$ to $n$}
      \State $C_\seed\|C_{\sign^0}\|C_{\sign^1}\gets {\sf Correct}(x_1\cdots x_{i-1},\sign,CW^{(i)})$.
      \State $\seed||\sign\gets G_{x_i}(\seed)$. 
      \State $\seed\|\sign\gets G_{x_i}(\seed)\oplus(C_{\seed}\|C_{\sign^{x_i}})$. 
    \EndFor
    \State \Return $(-1)^b\cdot \big(G_{\sf convert}(\seed)+{\sf ConvCorrect}(x,\sign,CW^{(n+1)})\big)$. 
    \EndProcedure
    \item[]
    \Procedure{FullEval\(_b\)}{$1^\lambda,k_b$}
    \State Parse $k_b=(\seed^{(0)},\sign^{(0)},CW^{(1)},CW^{(2)},\cdots,CW^{(n+1)})$. 
    \State For $1\le i\le n$, ${\sf Path}^{(i)}\gets$ the lexicographical ordered list of $\{0,1\}^i$. ${\sf Path}^{(0)}\gets[\epsilon]$. 
    \For{$i=1$ to $n$}
      \For{$k = 1$ to $2^{i-1}$}
        \State $C_\seed\|C_{\sign^0}\|C_{\sign^1}\gets {\sf Correct}({\sf Path}{(i-1)}[k],\sign^{(i-1)}[k],CW^{(i)})$.
        \State $\seed^{(i)}[2k]\|\sign^{(i)}[2k]\gets G_0(\seed^{(i-1)}[k])\oplus (C_\seed\|C_{\sign^0})$.
        \State $\seed^{(i)}[2k+1]\|\sign^{(i)}[2k+1]\gets G_1(\seed^{(i-1)}[k])\oplus (C_\seed\|C_{\sign^1})$.
      \EndFor
    \EndFor
    \For{$k = 1$ to $2^n$}
      \State ${\sf Output}[k]\gets (-1)^b\cdot \big(G_{\sf convert}(\seed^{(n)}[k])+{\sf ConvCorrect}({\sf Path}[k],\sign^{(n)}[k],CW^{(n+1)})\big)$.
    \EndFor
    \State\Return $\sf Output$. 
    \EndProcedure
    \end{algorithmic}}}
\end{figure*}

\newpage
\subsection{Big-State DMPF}
Displayed in \cref{fig:DMPF_big-state}.
TBD: explain
\begin{figure}
  \caption{The parameter $l$ and methods' setting that turns the paradigm of DMPF in~\cref{fig:DMPF_paradigm} into the big-state DMPF. }
  \label{fig:DMPF_big-state}
  \fbox{\parbox{\linewidth}{
  \begin{algorithmic}
    \State Set $l\leftarrow t$, the upperbound of $|A|$. 
    \Procedure{Initialize}{$\{\seed_b^{(0)},\sign_b^{(0)}\}_{b=0,1}$}
    \State For $b=0,1$, let $\seed_b^{(0)} = [r_b]$ where $r_b\xleftarrow{\$}\{0,1\}^\lambda$. 
    \State For $b=0,1$, set $\sign_b^{(0)} = [b\|0^{t-1}]$. 
    \EndProcedure
    \item[]
    \Procedure{GenCW}{$i,A,\{\seed_b^{(i-1)},\sign_b^{(i-1)}\}_{b=0,1}$}
    \State Let $\{A^{(i)}\}_{0\le i\le n}$ be defined as in~\cref{fig:DMPF_paradigm}. 
    \State Sample a list $CW$ of $t$ random strings from $\{0,1\}^{\lambda+2t}$.  
    \For{$k = 1$ to $|A^{(i-1)}|$}
      \State Parse $G(\seed_b^{(i-1)}[k]) = \seed_b^0\|\sign_b^0\|\seed_b^1\|\sign_b^1$, for $b=0,1$, $\seed_b^0,\seed_b^1\in\{0,1\}^\lambda$ and $\sign_b^0,\sign_b^1\in\{0,1\}^t$. 
      \State Compute $\Delta\seed^c = \seed_0^c\oplus\seed_1^c$ and $\Delta \sign^c = \sign_0^c\oplus\sign_1^c$ for $c=0,1$. 
      \State Denote ${\sf path}\leftarrow A^{(i-1)}[k]$. 
      \If{both ${\sf path}\|z$ for $z=0,1$ are in $A^{(i)}$}
        \State $d\gets$ the index of ${\sf path}\|0$ in $A^{(i)}$.
        \State $CW[d]\gets r\|\Delta\sign^0\oplus e_d \|\Delta\sign^1\oplus e_{d+1}$ where $r\xleftarrow{\$}\{0,1\}^\lambda$, $e_d = 0^{d-1}10^{t-d}$. 
      \Else
        \State Let $z$ be such that ${\sf path}\|z\in A^{(i)}$. 
        \State $d\gets$ the index of ${\sf path}\|z$ in $A^{(i)}$. 
        \State $CW[d]\gets 
          \begin{cases}
            \Delta \seed^1\|\Delta\sign^0\oplus e_d\|\Delta\sign^1 & z=0\\
            \Delta \seed^0\|\Delta\sign^0\|\Delta\sign^1\oplus e_d & z=1
          \end{cases}$.        
      \EndIf 
    \EndFor
    \State\Return $CW$. 
    \EndProcedure
    \item[]
    \Procedure{GenConvCW}{$A,B,\{\seed_b^{(n)},\sign_b^{(n)}\}$}
      \State Sample a list $CW$ of $t$ random $\GG$-elements.  
      \For{$k = 1$ to $|A|$}
        \State $\Delta g\gets G_{\sf convert}(\seed_0^{(n)}[k]) - G_{\sf convert}(\seed_1^{(n)}[k])$. 
        \State$CW[k]\gets (-1)^{\sign_0^{(n)}[k][k]}(\Delta g-B[k])$.
      \EndFor
      \State \Return $CW$. 
    \EndProcedure
    \item[]
    \Procedure{Correct}{$\bar{x},\sign,CW$}
      \State \Return $C_{\seed}\|C_{\sign^0}\|C_{\sign^1}\gets\sum_{i=1}^t \sign[i]\cdot CW[i]$, where $C_{\sign^0}$ and $C_{\sign^1}$ are $t$-bit. 
    \EndProcedure
    \item[]
    \Procedure{ConvCorrect}{$x,\sign,CW$}
      \State \Return $\sum_{i=1}^t \sign[i]\cdot CW[i]$. 
    \EndProcedure
  \end{algorithmic}}}
\end{figure}

\subsection{Batch-Code DMPF}
We display the construction of DMPF from black-box usage of DPF basing on PBC with appropriate parameters, which has been discussed in previous literature\cite{cryptoeprint:2019/273,cryptoeprint:2021/580}. 
\begin{construction}[DMPF from DPF]\label{constr:DMPF_batch_code}
  Given DPF for any domain of size no larger than $N$ and output group $\GG$, and an $(N,M,t,m,\epsilon)$-PBC with alphabet $\Sigma=\GG$, we can construct a DMPF scheme for $t$-point functions with domain size $N$ and output group $\GG$ as follows: 
  \begin{itemize}
    \item $\Gen(1^\lambda, \hat{f}_{A,B})\rightarrow (k_0,k_1)$: Suppose $A=\{\alpha_1,\cdots,\alpha_t\}$ and $B=\{\beta_1,\cdots,\beta_t\}$. Let $TT\in \GG^N$ be the truth table of $\hat{f}_{A,B}$. Compute $\Encode(TT)\rightarrow (C_1,\cdots,C_m)$ according to the PBC. Then run $\Decode(A, C_1,\cdots, C_m)$ to determine a perfect matching from $A$ to $\{C_1,\cdots,C_m\}$. For $1\le i\le m$, let $f_i:[|C_i|]\rightarrow \GG$ be the following: 
    \begin{itemize}
      \item If $C_i$ is assigned none of $A$ by the perfect matching, then set $f_i$ to be the all-zero function. 
      \item If exactly one $\alpha_j$ of $A$ is assigned to the $l$th position of $C_i$, then set $f_i$ to be the point function that outputs $\beta_j$ on $l$ and 0 elsewhere. 
    \end{itemize}
    For $1\le i\le m$, invoke DPF$.\Gen(1^\lambda, f_i)\rightarrow (k_0^i,k_1^i)$. Set $(k_0,k_1)=(\{k_0^i\}_{i\in [m]}, \{k_1^i\}_{i\in [m]})$. If $\Decode$ fails then run $\Encode$ and $\Decode$ again with fresh randomness. 
    \item $\Eval_b(k_b,x)\rightarrow y_b$: Follow $\Encode(TT)$ to determine the positions $l_{j_1},l_{j_2},\cdots, l_{j_s}$ such that the $x$th entry of $TT$ is sent to the $l_{j_i}$-th position of $C_{j_i}$. Compute $y_b=\sum_{i=1}^s$DPF$.\Eval_b(k_b^{j_i},l_i)$. 
  \end{itemize}
The scheme is correct with overwhelming probability and has distinguish advantage $<2\epsilon$. 
\end{construction}
Note that if one use batch code instead of PBC then the DMPF scheme perfectly correct and secure. When instantiating PBC from $w$-way cuckoo hashing, the \emph{key generation time} is roughly the time needed for computing cuckoo hashing algorithm plus the total time of all DPF$.\Gen(1^\lambda, f_i)$. The \emph{evaluation time} is roughly the total time of all DPF$.\Eval_b(k_b^{j_i},l_i)$. Similarly, the \emph{full-domain evaluation time} is roughly the total time of all DPF$.\FullEval_b(k_b^{j})$ for $j=1,\dots,m$. 

\subsection{OKVS-based DMPF}
Displayed in \cref{fig:DMPF_OKVS}. 
TBD: explain
\newpage
\begin{figure}
  \caption{The parameter $l$ and methods' setting that turns the paradigm of DMPF in~\cref{fig:DMPF_paradigm} into the OKVS-based DMPF. }
  \label{fig:DMPF_OKVS}
  \fbox{\parbox{\linewidth}{
  \begin{algorithmic}
    \State Set $l\leftarrow 1$. 
    \State For $1\le i\le n$, let $\OKVS_i$ be an OKVS scheme (\cref{def:OKVS}) with key space $\mathcal{K} = \{0,1\}^{i-1}$, value space $\mathcal{V} = \{0,1\}^{\lambda+2}$ and input length $t$. 
    \State let $\OKVS_{\sf convert}$ be an OKVS scheme with key space $\mathcal{K} = \{0,1\}^n$, value space $\mathcal{V} = \GG$ and input length $t$. 
    \item[]
    \Procedure{Initialize}{$\{\seed_b^{(0)},\sign_b^{(0)}\}_{b=0,1}$}
    \State For $b=0,1$, let $\seed_b^{(0)} = [r_b\xleftarrow{\$}\{0,1\}^\lambda]$ and $\sign_b^{(0)} = [b]$. 
    \EndProcedure
    \item[]
    \Procedure{GenCW}{$i,A,\{\seed_b^{(i-1)},\sign_b^{(i-1)}\}_{b=0,1}$}
    \State Let $\{A^{(i)}\}_{0\le i\le n}$ be defined as in~\cref{fig:DMPF_paradigm}. 
    \State Sample a list $V$ of $t$ random strings from $\{0,1\}^{\lambda+2}$.  
    \For{$k = 1$ to $|A^{(i-1)}|$}
      \State Parse $G(\seed_b^{(i-1)}[k]) = \seed_b^0\|\sign_b^0\|\seed_b^1\|\sign_b^1$, for $b=0,1$, $\seed_b^0,\seed_b^1\in\{0,1\}^\lambda$ and $\sign_b^0,\sign_b^1\in\{0,1\}$. 
      \State Compute $\Delta\seed^c = \seed_0^c\oplus\seed_1^c$ and $\Delta \sign^c = \sign_0^c\oplus\sign_1^c$ for $c=0,1$. 
      \State Denote ${\sf path}\leftarrow A^{(i-1)}[k]$. 
      \If{both ${\sf path}\|z$ for $z=0,1$ are in $A^{(i)}$}
        \State $V[k]\gets r\|\Delta\sign^0\oplus 1\|\Delta\sign^1\oplus 1$, where $r\xleftarrow{\$}\{0,1\}^\lambda$. 
      \Else
        \State Let $z$ be such that ${\sf path}\|z\in A^{(i)}$. 
        \State $V[k]\gets \Delta \seed^1\|\Delta\sign^0\oplus (1-z)\|\Delta\sign^1\oplus z$.        
      \EndIf
    \EndFor
    \State \Return $\OKVS_i.\Encode(\{A^{(i-1)}[k], V[k]\}_{1\le k\le |A^{(i-1)}|})$. 
    \EndProcedure
    \item[]
    \Procedure{GenConvCW}{$A,B,\{\seed_b^{(n)},\sign_b^{(n)}\}$}
      \State Sample a list $V$ of $t$ random $\GG$-elements. 
      \For{$k = 1$ to $|A|$}
        \State $\Delta g\gets G_{\sf convert}(\seed_0^{(n)}[k]) - G_{\sf convert}(\seed_1^{(n)}[k])$. 
        \State$V[k]\gets (-1)^{\sign_0^{(n)}[k][k]}(\Delta g-B[k])$.
      \EndFor 
      \State \Return $\OKVS_{\sf convert}(\{A[k], V[k]\}_{1\le k\le t})$. 
    \EndProcedure
    \item[]
    \Procedure{Correct}{$\bar{x}, \sign,CW$}
      \State\Return $C_{\seed}\|C_{\sign^0}\|C_{\sign^1}\gets\sign\cdot\OKVS_i.\Decode(CW, \bar{x})$, where $C_{\sign^0}$ and $C_{\sign^1}$ are bits. 
    \EndProcedure
    \item[]
    \Procedure{ConvCorrect}{$x,\sign,CW$}
      \State \Return $\sign\cdot\OKVS_{\sf convert}.\Decode(CW,x)$. 
    \EndProcedure
  \end{algorithmic}}}
\end{figure}
\Yaxin{One point: the $\row$ matrix of the current layer contains the $\row$ matrix of the previous layers, which might be useful for speedup. }

\subsection{Comparison}
%Comparison table dependent to PRG \& $\FF$-MUL(list the formulas?)\\
%analyze tradeoff\\
%distributed gen advantage
\Cref{tab:formulas_DMPF_comparison} displays the keysize, running time of $\Gen$,$\Eval$ and $\FullEval$ for different DMPF schemes, computed in terms of costs of abstract tools such as PRG, batch code and OKVS. \Yaxin{We can plug in the actual costs of these tools after carrying out a complete experiment. }
	\begin{table*}
    \caption{Keysize and running time comparison for different DMPF constructions for domain size $N$, $t$ accepting points and computational security parameter $\lambda$. We leave this table with the abstraction of (probabilistic) batch code in the second column and the abstraction of OKVS in the last column, and plug in concrete instantiations later. $m$ in the second column stands for the number of buckets used in batch code, and $w$ stands for the number of buckets that each input coordinate is mapped to (we only consider regular degree because this is the case in most instantiations). }
    \label{tab:formulas_DMPF_comparison}
		\scalebox{0.86}{
			\begin{tabular}{ccccc}
				\toprule 
        %Header
				 &Sum of $t$ DPFs & Batch code DMPF\cite{cryptoeprint:2019/273,cryptoeprint:2019/1084,cryptoeprint:2021/580,cryptoeprint:2017/1142} & Big-state DMPF & OKVS-based DMPF\\

        \midrule

				keysize & $t(\lambda+2)\log N$ & $m\lambda\log(wN/m)$ & $t(\lambda+2t)\log N$ &\makecell{ $\log N\times$OKVS code size}\\

				\begin{tabular}{cc}
					$Gen()$ & \makecell{Dominating operations \\\hdashline Cheap operations}
				\end{tabular} &\makecell{$2t\log N\times $ PRG\\\hdashline$O(t\lambda\log N$)} &\makecell{$2m\log(wN/m)\times $PRG\\Finding a matching of $t$ inputs to $m$ buckets\\\hdashline$O(m\lambda\log(wN/m))$} &\makecell{$2t\log N\times$PRG \\\hdashline$O(t(\lambda+t)\log N)$} &\makecell{$2t\log N\times$PRG, \\$\log N\times $OKVS  Encoding\\\hdashline$O(t\lambda\log N)$} \\

				\begin{tabular}{cc}
					$Eval()$ & \makecell{Dominating operations \\\hdashline Cheap  operations}
				\end{tabular} &\makecell{$t\log N\times $PRG\\\hdashline$O(t\lambda\log N)$} &\makecell{$w\log(wN/m)\times $PRG\\Finding all buckets an input is mapped to\\\hdashline$O(w\lambda\log(wN/m))$} & \makecell{$\log N\times$PRG\\\hdashline$O((\lambda+t)\log N)$} &\makecell{$\log N\times$PRG, \\$\log N\times$OKVS Decoding\\\hdashline$O(\lambda\log N)$} \\

				\begin{tabular}{cc}
					$FullEval()$ & \makecell{Dominating operations \\\hdashline Cheap  operations}
				\end{tabular} &\makecell{$tN\times$PRG\\\hdashline$O(t\lambda N)$} &\makecell{$wN\times$PRG\\Finding the input sequence in every bucket\\\hdashline $O(w\lambda N)$} & \makecell{$N\times$PRG\\\hdashline$O((\lambda+t)N)$}&\makecell{$N\times$PRG, \\ $N\times$ OKVS Decoding\\\hdashline$O(\lambda N)$} \\

        \bottomrule
			\end{tabular}	
		}
	\end{table*}

  \Yaxin{Take PCG as a potential application. We care about $\FullEval$ time which is related to PCG seed expanding time. In this aspect, the batch code DMPF consumes $d\times$PRGs than big-state DMPF and OKVS-based DMPF, while big-state DMPF's $\FullEval$ time scales with $t$ and OKVS-based DMPF in addition consumes large field multiplications (in OKVS decoding, and maybe more than this). Therefore we expect different DMPF schemes to be the top choice in different choices of $t$ and depending on the computing time of PRG and large field multiplication. }

  
\subsection{Distributed Key Generation}