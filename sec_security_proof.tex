\section{Security Proofs of DMPF Schemes}\label{sec:security_proof}
In this section we show that the big-state and OKVS-based DMPF schemes are computationally secure. We prove by first showing that the DMPF template is computationally secure when the methods $\sf  GenCW$ and $\sf GenConvCW$ satisfy some security requirements called hiding, and then arguing that the realizations of methods in both the big-state DMPF and the OKVS-based DMPF satisfy hiding. 

\subsection{The DMPF template is secure under special conditions}\label{sec:template_security_proof}
We first define the security notion called hiding for $\Gen\CW$ and $\Gen\Conv\CW$, and show that the DMPF template in \Cref{fig:DMPF_template} is secure when $\Gen\CW$ and $\Gen\Conv\CW$ are hiding. For simplicity, assume we're working with multi-point functions with exactly $t$ accepting inputs. If there are less than $t$ accepting inputs, simply pad arbitrary distinct inputs to $A$ and pad $B$ with $0$. 

We say that a realization of $\Gen\CW$ is hiding if it gives no information about the multi-point function $f_{A,B}$. It is defined using an indistinguishability-based definition as follows: 
\begin{definition}[Computationally hiding of $\Gen\CW$]\label{def:GenCW_hiding}
    $\Gen\CW$ is computationally hiding if $\forall i\in[n]$, $\forall A\not=A'$,
    \[
        \begin{split}
            &\{CW\gets \Gen\CW(i,A,V^{(i-1)})\}_{V^{(i-1)}\gets(\{0,1\}^{\lambda+2l})^{\otimes |A^{(i-1)}|}}\\\approx_c&\{CW\gets \Gen\CW(i,A',V'^{(i-1)})\}_{V'^{(i-1)}\gets(\{0,1\}^{\lambda+2l})^{\otimes |A'^{(i-1)}|}}
        \end{split}
    \]
    where $A^{(i-1)}$ denotes the set of length-$(i-1)$ prefixes of $A$. 
\end{definition}
Similarly we define a realization of $\Gen\Conv\CW$ to be hiding if it gives no information about $f_{A,B}$: 
\begin{definition}[Computationally hiding of $\Gen\Conv\CW$]\label{def:GenConvCW_hiding}
    $\,$\\$\Gen\Conv\CW$ is computationally hiding if $\forall i\in[n]$, $\forall (A,B)\not=(A',B')$, 
    \[
        \begin{split}
            &\{CW\gets \Gen\Conv\CW(A,B,\Delta g, \sign_0^{(n)}, \sign_1^{(n)})\}_{\substack{
                \Delta g\gets \GG^t\\\sign_0\gets (\{0,1\}^l)^{\otimes t}\\\sign_1\gets (\{0,1\}^l)^{\otimes t}}}\\\approx_c&\{CW\gets \Gen\Conv\CW(A',B',\Delta g, \sign_0^{(n)}, \sign_1^{(n)})\}_{\substack{
                    \Delta g\gets \GG^t\\\sign_0\gets (\{0,1\}^l)^{\otimes t}\\\sign_1\gets (\{0,1\}^l)^{\otimes t}}}
        \end{split}
    \]
\end{definition}
Next we show that if $\Gen\CW$ and $\Gen\Conv\CW$ in the DMPF template \Cref*{fig:DMPF_template} are computationally hiding, then the DMPF is computationally secure. 
\begin{lemma}\label{lem:template_secure}
    Suppose ${\sf Initialize}$ set $\seed_b^{(0)}$ to be random strings for $b=0,1$. Suppose  $\Gen\CW$ and $\Gen\Conv\CW$ in \Cref{fig:DMPF_template} are computationally hiding with distinguishing advantage $\epsilon_{\Gen\CW}$ and $\epsilon_{\Gen\Conv\CW}$ respectively. Let $\epsilon_G$ and $\epsilon_{G_\conv}$ denote the distinguishing advantage of the PRG $G$ and $G_\conv$. Then the DMPF scheme is $\epsilon$-secure against any n.u.p.p.t. adversary, where $\epsilon = 2tn\epsilon_{G}+2t \epsilon_{G_\conv}+n\epsilon_{\Gen\CW} +\epsilon_{\Gen\Conv\CW}$. 
\end{lemma}
\begin{proof}
    Formally, we show that $\forall b\in\{0,1\}$, $\forall (A,B)\not=(A',B')$, 
    \[
        \begin{split}
            \{k_b\gets \Gen(1^\lambda, \hat{f}_{A,B})\}\approx_c\{k_b'\gets\Gen(1^\lambda, \hat{f}_{A',B'})\}
        \end{split}
    \]
    with distinguishing advantage at most $\epsilon = 2tn\epsilon_{G}+2t \epsilon_{G_\conv}+n\epsilon_{\Gen\CW} +\epsilon_{\Gen\Conv\CW}$. This suffices to prove the simulation-based security of DMPF, as one can simply let ${\sf Sim}(1^\lambda, N,\hat{\GG},t)$ for corrupted party $b$ output $k_b$ output by $\Gen(1^\lambda, \hat{f}_{[t],0^t}$. 
    
    Fix an arbitrary $b$ and $(A,B), (A',B')$ such that $(A,B)\not=(A',B')$. By symmetry we may assume $b=0$. 
    Parse $k_0 = (\seed_0^{(0)}, \sign_0^{(0)}, $\linebreak$CW^{(1)},\dots,CW^{(n+1)})$ and $k_b' = (\seed_0'^{(0)}, \sign_0'^{(0)}, CW'^{(1)},\dots,$\linebreak$CW'^{(n+1)})$. We prove the distributions of $k_0$ and $k_0'$ are indistinguishable by a sequence of standard hybrid arguments. 


    Note that $\{\seed_0^{(0)}, \sign_0^{(0)}\} = \{\seed_0'^{(0)}, \sign_0'^{(0)}\}$ since the \linebreak$\sf Initialize$ processes are the same. For $i\in[n]$, $\Gen\CW(1^\lambda, \hat{f}_{A,B})$ having the $\seed_b^{(i-1)}$ and $\sign_b^{(i-1)}$ strings of the previous layer for $b=0,1$, computes as the following: 
    \begin{enumerate}
        \item First determine the offset value list $V$ by $i, A$ and a substring of $G(\seed_0^{(i-1)})\oplus G(\seed_1^{(i-1)})$. We denote the indices of this substring as $T$. 
        \item Then invoke $\Gen\CW(i,A,V)$ to get $CW^{(i)}$. 
        \item For $b=0,1$, compute the lists $\seed_b^{(i)}$ and $\sign_b^{(i)}$ by adding a correction (determined by $i,A,CW^{(i)}, \sign^{(i-1)}$) to a substring of $G(\seed_b^{(i-1)})$. Denote the indices of this substring as $S$ (it is the same set of indices for $b=0,1$). 
    \end{enumerate}
    We note several crucial points from the above process: First, the computation only dependends on $i,A,\seed_b^{(i-1)}$ and $\sign_b^{(i-1)}$ for $b=0,1$. Second, since $CW^{(i)}$ is computed by $\Gen\CW(i,A,V)$ and according to the construction of $V$, the correction in the last step is only dependent to $i, A, \sign^{(i-1)}$ and a substring of $G(\seed_0^{(i-1)})\oplus G(\seed_1^{(i-1)})$ indexed by $T$. Next, note that the set of $\seed$ indices in $S$ does note intersect with $T$. Therefore, when $G$ is a PRG we have the joint distribution of $V$ and $\seed_b^{(i)}$ is pseudorandom. 
    
    Now we construct the following hybrid distributions for each layer $i\in[n]$: 
    
    \begin{itemize}
        \item$\Hyb_i^0 = \{\seed_0'^{(0)},\sign_0'^{(0)},(CW'^{(k)})_{1\le k<i},(CW^{(k)})_{i\le k\le n+1}\}$ where $\seed_0'^{(0)},\sign_0'^{(0)},(CW'^{(k)})_{1\le k\le i-1}$ are generated by $\Gen(1^\lambda, \hat{f}_{A',B'})$, $\seed_1^{(i-1)}$ is set to truly random with the same length as $A^{(i-1)}$, $\sign_1^{(i-1)}$ is set to satisfy an arbitrary desired correlation with $\sign_0'^{(i-1)}$, and $(CW^{(k)})_{i\le k\le n+1}$ are  generated by $\Gen(1^\lambda, \hat{f}_{A,B})$ with the previous state being\linebreak $\seed_0'^{(0)},\sign_0'^{(0)},(CW'^{(k)})_{1\le k\le i-1},\seed_1^{(i-1)}, \sign_1^{(i-1)}$. 
        \item $\Hyb_i^1 = \{\seed_0'^{(0)},\sign_0'^{(0)},(CW'^{(k)})_{1\le k<i},(CW^{(k)})_{i\le k\le n+1}\}$ where $\seed_0'^{(0)},\sign_0'^{(0)},(CW'^{(k)})_{1\le k\le i-1}$ are generated by $\Gen(1^\lambda, \hat{f}_{A',B'})$. To generate the remaining components,\linebreak $\Gen(1^\lambda, \hat{f}_{A,B})$ sets $\sign_1^{(i-1)}$ to satisfy the arbitrary desired correlation with $\sign_0'^{(i-1)}$, and substitutes the use of \linebreak$G(\seed_1^{(i-1)})$ with a truly random string. This makes  $V^{(i-1)}$ a length-$|A^{(i-1)}|$ list of truly random $(\lambda+2l)$-bit strings, and $\seed_1^{(i)}$ will be computed by adding a correction to a truly random string $s$. Then $\Gen(1^\lambda, \hat{f}_{A,B})$ with the previous state being $\seed_0'^{(0)}$, $\sign_0'^{(0)}$, $(CW'^{(k)})_{1\le k\le i-1}$, $ V^{(i-1)}$, $ \sign_1^{(i-1)}$, $ s$. 
        \item $\Hyb_i^2 = \{\seed_0'^{(0)},\sign_0'^{(0)},(CW'^{(k)})_{1\le k\le i},(CW^{(k)})_{i< k\le n+1}\}$ where $\seed_0'^{(0)},\sign_0'^{(0)},(CW'^{(k)})_{1\le k\le i-1}$ are generated by $\Gen(1^\lambda, \hat{f}_{A',B'})$. To generate the remaining components, set $\sign_1^{(i-1)}$ to satisfy the arbitrary desired correlation with $\sign_0'^{(i-1)}$, sample a length-$|A'^{(i-1)}|$ list $V'^{(i-1)}$ of truly random $(\lambda+2l)$-bit strings, and generate $CW'^{(i)}\gets \Gen\CW(i,$\linebreak$A',V'^{(i-1)})$. Then invoke $\Correct$ to compute $\seed_0^{(i)}$ and $\sign_0^{(i)}$ of length $|A^{(i)}$ in $\Gen(1^\lambda, \hat{f}_{A,B})$. Set 
        $\seed_1^{(i)}$ to be a length-$|A^{(i)}|$ list of truly random strings, $\sign_1^{(i)}$ to satisfy the desired correlation with $\sign_0^{(i)}$, and generate \linebreak$(CW^{(k)})_{i< k\le n+1}$ by $\Gen(1^\lambda, \hat{f}_{A,B})$ with the previous state being $\seed_0'^{(0)},\sign_0'^{(0)},(CW'^{(k)})_{1\le k\le i}, V^{(i-1)}$, $\seed_1^{(i-1)}$, \linebreak$ \sign_1^{(i-1)}$. 
    \end{itemize}
    
    For the convert layer, we construct the following hybrid distributions: 
    \begin{itemize}
        \item $\Hyb_{n+1}^0 = \{\seed_0'^{(0)},\sign_0'^{(0)},(CW'^{(k)})_{1\le k\le n},CW^{(n+1)}\}$ \linebreak 
        where $\seed_0'^{(0)},\sign_0'^{(0)},(CW'^{(k)})_{1\le k\le n}$ are generated by $\Gen(1^\lambda, \hat{f}_{A',B'})$. $\seed_0^{(n)}$ and $\sign_0^{(n)}$ are computed by $\Gen(1^\lambda, $\linebreak
        $\hat{f}_{A,B})$ but with the previous state being $\seed_0'^{(0)},\sign_0'^{(0)},$\linebreak
        $(CW'^{(k)})_{1\le k\le n}$. $\seed_1^{(n)}$ is set to a length-$t$ list with truly random $\lambda$-bit strings, $\sign_1^{(n)}$ is set to satisfy the desired correlation with $\sign_0'^{(i-1)}$, and $CW^{(n+1)}$ is generated by \linebreak
        $\Gen\Conv\CW(A,B,G_{\conv}(\seed_0^{(n)}) - G_\conv(\seed_1^{(n)}),$ $\sign_0^{(n)}$, \linebreak
        $\sign_1^{(n)})$. 
        \item $\Hyb_{n+1}^1 = \{\seed_0'^{(0)},\sign_0'^{(0)},(CW'^{(k)})_{1\le k\le n},CW^{(n+1)}\}$\linebreak 
        where $\seed_0'^{(0)},\sign_0'^{(0)},(CW'^{(k)})_{1\le k\le n}$ are generated by\linebreak 
        $\Gen(1^\lambda, \hat{f}_{A',B'})$. $\seed_0^{(n)}$ and $\sign_0^{(n)}$ are computed by\linebreak 
        $\Gen(1^\lambda, \hat{f}_{A,B})$ but with the previous state being $\seed_0'^{(0)}$, $\sign_0'^{(0)}$, $(CW'^{(k)})_{1\le k\le n}$. Sample a length-$t$ list $\Delta g$ of random $\GG$ elements, set $\sign_1^{(n)}$ to satisfy the desired correlation with $\sign_0'^{(i-1)}$, and generate $CW^{(n+1)}$ by $\Gen\Conv\CW(A,B,$\linebreak
        $\Delta g, \sign_0^{(n)},\sign_1^{(n)})$. 
        \item$\Hyb_{n+1}^2 = \{\seed_0'^{(0)},\sign_0'^{(0)},(CW'^{(k)})_{1\le k\le n+1}\}$ where \linebreak$\seed_0'^{(0)},\sign_0'^{(0)},(CW'^{(k)})_{1\le k\le n}$ are generated by $\Gen(1^\lambda,$\linebreak$ \hat{f}_{A',B'})$. $\seed_0^{(n)}$ and $\sign_0^{(n)}$ are computed by $\Gen(1^\lambda, \hat{f}_{A,B})$ but with the previous state being $\seed_0'^{(0)}$, $\sign_0'^{(0)}$, \linebreak$(CW'^{(k)})_{1\le k\le n}$. Sample a length-$t$ list $\Delta g$ of random $\GG$ elements, set $\sign_1^{(n)}$ to satisfy the desired correlation with $\sign_0'^{(i-1)}$, and generate $CW'^{(n+1)}$ by $\Gen\Conv\CW(A',B',$\linebreak$\Delta g, \sign_0^{(n)},\sign_1^{(n)})$. 
        \item$\Hyb_{n+1}^3 = \{\seed_0'^{(0)},\sign_0'^{(0)},(CW'^{(k)})_{1\le k\le n+1}\}$ generated by $\Gen(1^\lambda, \hat{f}_{A',B'})$. 
    \end{itemize}

    Next we argue that $\Hyb_1^0\approx_c\Hyb_{n+1}^3$. Since $\Hyb_1^0$ is the distribution of $k_0$ and $\Hyb_{n+1}^3$ is the distribution of $k_0'$, this proves the distributions of $k_0$ and $k_0'$ are computationally indistinguishable. 
    
    For all $i\in[n]$, $\Hyb_i^0\approx_c\Hyb_i^1$ with distinguishing advantage at most $t\epsilon_G$, since the only difference between the two distributions is the substitution of $G(\seed_1^{(i-1)})$ with truly random strings, which contains $|A^{(i-1)}|<t$ invocations of $G$. $\Hyb_i^1\approx_c\Hyb_i^2$ with distinguishing advantage at most $\epsilon_{\Gen\CW}$ by computational hiding of $\Gen\CW$, since the only difference between the two distributions is that $\Hyb_i^2$ replaces the output of $\Gen\CW(i,A,V^{(i-1)})$ by $\Gen\CW(i,A',V'^{(i-1)})$ where both $V^{(i-1)}$ and $V'^{(i-1)}$ are truly random. In the end $\Hyb_i^2\approx \Hyb_{i+1}^0$ with distinguishing advantage at most $t\epsilon_G$, since the only difference between the two distributions is the substitution of truly random strings with  $G(\seed_1'^{(i-1)})$ (happened in $\Gen(1^\lambda, \hat{f}_{A',B'})$, which is the inverse substitution happened when switching $\Hyb_i^0$ to $\Hyb_i^1$), which contains $|A'^{(i-1)}|<t$ invocations of $G$. In conclusion, $\Hyb_1^0\approx_c \Hyb_{n+1}^0$ with distinguishing advantage at most $2tn\epsilon_{G}+n\epsilon_{\Gen\CW}$. 

    Through a similar argument, $\Hyb_{n+1}^0\approx_c\Hyb_{n+1}^1$ , $\Hyb_{n+1}^1\approx_c\Hyb_{n+1}^2$, and $\Hyb_{n+1}^2  \approx_c\Hyb_{n+1}^3$, with distinguishing advantage at most $t\epsilon_{G_\conv}$, $\epsilon_{\Gen\Conv\CW}$, and $t\epsilon_{G_\conv}$ respectively. 

    Henceforth, $\Hyb_1^0\approx_c\Hyb_{n+1}^3$ with distinguishing advantage at most $\epsilon = 2tn\epsilon_{G}+2t \epsilon_{G_\conv}+n\epsilon_{\Gen\CW} +\epsilon_{\Gen\Conv\CW}$.  
\end{proof}

\subsection{The big-state DMPF scheme is secure}\label{sec:big-state_security_proof}
In this section we show that the big-state realization of DMPF in \Cref{fig:DMPF_big-state} satisfies the conditions: (1) $\sf Initialize$ set $\seed_0^{(0)}$ and $\seed_1^{(1)}$ to random strings. (2) $\Gen\CW$ is computationally hiding. (3) $\Gen\Conv\CW$ is computationally hiding. Combined with \Cref{lem:template_secure}, it can be concluded that the big-state DMPF scheme is computationally secure. 

Condition (1) is direct from the construction. 

For condition (2), note that when $V$ is a truly random list, \linebreak$\Gen\CW(i,A,V)$ in \Cref{fig:DMPF_big-state} computes each entry of $CW$ by adding up a random string with one of $0^{\lambda+2t}$, $0^\lambda\|e_d\|0^t$, $0^\lambda\|0^t, e_d$ and $0^\lambda\|e_d\|e_{d+1}$ for some $d$, determined by $i,A$. Therefore, the resulting $CW$ is a length-$t$ list of random strings, independent of $i,A$. Hence $\epsilon_{\Gen\CW} = 0$. 

For condition (3), when $\Delta_g$ is a truly random list, $\Gen\Conv\CW(A,$\linebreak$B,\Delta g,\sign_0^{(n)},\sign_1^{(n)})$ computes $CW[k] = (-1)^{\sign_0^{(n)}[k][k]}(\Delta g[k] $\linebreak
$- B[k])$, which is a random group element. Therefore the resulting $CW$ is a length-$t$ list of random group elements, independent of $A,B$. Hence $\epsilon_{\Gen\Conv\CW} = 0$. 

The above arguments, combined with \Cref{lem:template_secure}, concludes that the big-state DMPF is computationally secure. 

\begin{theorem}[Big-state DMPF is secure]\label{thm:big-state_secure}
    Let $\epsilon_G$ and $\epsilon_{G_\conv}$ denote the distinguishing advantage of the PRG $G$ and $G_\conv$ respectively. Then the big-state DMPF scheme in \Cref{fig:DMPF_big-state} is $\epsilon$-secure against any n.u.p.p.t. adversary, where $\epsilon = 2tn\epsilon_G+2t\epsilon_{G_\conv}$. 
\end{theorem}

\subsection{The OKVS-based DMPF scheme is secure}\label{sec:OKVS_DMPF_security_proof}
In this section we argue that the OKVS-based realization of DMPF in \Cref{fig:DMPF_OKVS} satisfies the conditions: (1) $\sf Initialize$ set $\seed_0^{(0)}$ and $\seed_1^{(1)}$ to random strings. (2) $\Gen\CW$ is computationally hiding. (3) $\Gen\Conv\CW$ is computationally hiding. Combined with \Cref{lem:template_secure}, it can be concluded that the OKVS-based DMPF scheme is computationally secure. 

Condition (1) is direct from the construction. 

For condition (2), note that when $V$ is a truly random list, \linebreak$\Gen\CW(i,A,V)$ in \Cref{fig:DMPF_OKVS} pads $V$ to be of length $t$, and updates each entry of $V$ by adding up a random string with one of $0^{\lambda+2}$, $0^\lambda\|10$, $0^\lambda\|01$ and $0^\lambda\|11$ for some $d$, determined by $i,A$. Therefore, the resulting $V$ is a length-$t$ list of random strings, independent of $i,A$. By the obliviousness of the OKVS scheme, $CW$ as an encoding of keys from $A$ and values from $V$ should be computationally indistinguishable from an encoding of keys from $A'$ and values from $V$. Hence $\epsilon_{\Gen\CW} = \epsilon_{\OKVS_i}$, the security of the corresponding $\OKVS_i$ scheme. 

For condition (3), when $\Delta_g$ is a truly random list, $\Gen\Conv\CW(A,$\linebreak$B,\Delta g,\sign_0^{(n)},\sign_1^{(n)})$ computes a value list $V$ of group elements by $V[k] = (-1)^{\sign_0^{(n)}[k][k]}(\Delta g[k] - B[k])$, which is a random group element. Therefore $V$ is a length-$t$ list of random group elements independent of $A,B$. Hence $\epsilon_{\Gen\Conv\CW} = \epsilon_{\OKVS_\conv}$, the security of the corresponding $\OKVS_\conv$ scheme. 

The above arguments, combined with \Cref{lem:template_secure}, concludes that the OKVS-based DMPF is computationally secure. 

\begin{theorem}[OKVS-based DMPF is secure]\label{thm:OKVS_DMPF_secure}
    Let $\epsilon_G$ and $\epsilon_{G_\conv}$ denote the distinguishing advantage of the PRG $G$ and $G_\conv$ respectively. Suppose in the OKVS-based DMPF scheme in \Cref{fig:DMPF_OKVS}, for $1\le i\le n$ the $\OKVS_i$ scheme is $\epsilon_{\OKVS_i}$-oblivious, and the $\OKVS_{\conv}$ scheme is $\epsilon_{\OKVS_\conv}$-oblivious. Then the big-state DMPF scheme in \Cref{fig:DMPF_big-state} is $\epsilon$-secure against any n.u.p.p.t. adversary, where $\epsilon = 2tn\epsilon_G+2t\epsilon_{G_\conv} +\sum_{i=1}^n\epsilon_{\OKVS_i}+\epsilon_{\OKVS_\conv}$. 
\end{theorem}

%remember to refer back in text. 
