\newcommand{\Enote}[1]{\color{purple}Enote: #1\color{black}}
\newcommand{\Yaxin}[1]{\color{purple}Yaxin: #1\color{black}}

\newcommand{\Gen}{{\sf Gen}}
\newcommand{\Eval}{{\sf Eval}}
\newcommand{\FullEval}{{\sf FullEval}}
\newcommand{\Conv}{{\sf Conv}}
\newcommand{\conv}{{\sf conv}}
\newcommand{\CW}{{\sf CW}}
\newcommand{\Correct}{{\sf Correct}}
\newcommand{\Encode}{{\sf Encode}}
\newcommand{\Decode}{{\sf Decode}}
\newcommand{\Position}{{\sf Position}}
\newcommand{\Schedule}{{\sf Schedule}}
\newcommand{\row}{{\sf row}}
\newcommand{\seed}{{\sf seed}}
\newcommand{\sign}{{\sf sign}}
\newcommand{\correct}{{\sf correct}}
\newcommand{\map}{{\sf map}}
\newcommand{\stat}{{\sf stat}}
\newcommand{\poly}{{\sf poly}}
\newcommand{\Adv}{{\sf Adv}}
\newcommand{\MULT}{{\sf MULT}}
\newcommand{\Hyb}{{\sf Hyb}}

\newcommand{\DMPF}{{\sf DMPF}}
\newcommand{\DPF}{{\sf DPF}}
\newcommand{\OKVS}{{\sf OKVS}}
\newcommand{\LPN}{{\sf LPN}}
\newcommand{\CBC}{{\sf CBC}}
\newcommand{\PBC}{{\sf PBC}}
%rule: specific parametrized objects in \sf, other in text terminologies in \rm. 
    
\newcommand{\GG}{\mathbb{G}}
\newcommand{\NN}{\mathbb{N}}
\newcommand{\ZZ}{\mathbb{Z}}
\newcommand{\FF}{\mathbb{F}}

\newcommand{\cA}{\mathcal{A}}
\newcommand{\cHW}{\mathcal{HW}}
\newcommand{\cRHW}{\mathcal{RHW}}

\newcommand{\red}[1]{\textcolor{red}{#1}}
\newcommand{\ipd}[2]{\langle #1, #2 \rangle}

\newtheorem{theorem}{Theorem}
\newtheorem{lemma}[theorem]{Lemma}
\newtheorem{claim}[theorem]{Claim}
\newtheorem{fact}[theorem]{Fact}
\newtheorem{definition}[theorem]{Definition}
\newtheorem{example}[theorem]{Example}
\newtheorem{remark}[theorem]{Remark}
\newtheorem{construction}{Construction}
\newtheorem{corollary}[theorem]{Corollary}
\newtheorem{proposition}[theorem]{Proposition}

\makeatletter
\newenvironment{breakablefigure}
  {
   \begin{center}
     \refstepcounter{figure}% New algorithm
     %\hrule height.8pt depth0pt \kern2pt% \@fs@pre for \@fs@ruled
     \renewcommand{\caption}[2][\relax]{% Make a new \caption
       {\raggedright\textbf{\figurename~\thefigure} ##2\par}%
       \ifx\relax##1\relax % #1 is \relax
         \addcontentsline{loa}{figure}{\protect\numberline{\thefigure}##2}%
       \else % #1 is not \relax
         \addcontentsline{loa}{figure}{\protect\numberline{\thefigure}##1}%
       \fi
       %\kern2pt\hrule\kern2pt
     }
  }{
     %\kern2pt\hrule\relax% \@fs@post for \@fs@ruled
    \end{center}
  }
\makeatother