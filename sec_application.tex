\section{Applications}\label{sec:applications}
In this section, we outline important applications of DMPF. In particular, we will describe prior works using different operations of DMPF to construct pseudorandom correlation generator (PCG) and unbalanced private set intersection with weighted weighted cardinality (PSI-WCA). 
For convenience of discussion, we use $\DMPF_{t, N, \GG}$ to denote a DMPF scheme for $t$-point functions with domain $[N]$ and output group $\GG$, which can be instantiated by all of the DMPF constructions we mentioned earlier. 

\paragraph{PCG for OLE from QA-SD.}

Prior works \cite{cryptoeprint:2022/1035,foleage} introduce and construct a pseudorandom correlation generator (PCG) for OLE correlations over various fields from $\LPN$-like hardness assumptions, using DMPF as a primitive. A PCG for OLE  consists of $(\PCG.\Gen, \PCG.\Expand)$ where $\PCG.\Gen$ generates a pair of short seeds and $\PCG.\Expand$ expands both seeds to a pair of strings satisfying the OLE correlation. 

We focus on the construction of \cite{foleage} which constructs PCG for $N$ random OLE correlations over $\mathbb{F}_4$ under the Quasi-Abelian Syndrome Decoding (QA-SD) assumption, parameterized by the total degree $N$, an LPN-like compression factor $c$ and a weight parameter $t$ such that $c\cdot t\approx O(\lambda)$. The ${\sf PCG.Gen}$ algorithm is comprised of $c^2 t\times \DMPF_{t, N/t, \FF_4}.\Gen$ and some additional operations.
The ${\sf PCG.Expand}$ algorithm contains $c^2t\times \DMPF_{t, N/t, \FF_4}.\FullEval$ along with many multi-variate polynomial multiplications. 


In~\cite{foleage} each $\DMPF_{t, N/t, \FF_4}$ is instantiated using the na\"ive DMPF, with the claim that step (1) of evaluating these na\"ive DMPF's is the bottleneck of their implementation. They also mentioned potential optimization by switching from na\"ive DMPF to PBC-based DMPF for a better asymptotic cost. 
By plugging in our new DMPF constructions as the underlying DMPF scheme, we obtain new realizations of the PCG construction. It turns out by plugging in our big-state DMPF, we get a significantly faster $\PCG.\Expand$, with a reasonable $\PCG.\Gen$ time and succinct $\PCG$ seed size (see~\cref{sec:impl_pcg}). 

\paragraph{Unbalanced PSI-WCA.}
DMPF is also useful when constructing a non-colluding 2-server-1-client unbalanced private set intersection with weighted cardinality (PSI-WCA) protocol, as proposed in \cite{cryptoeprint:2020/1599}, which can be utilized to privacy preserving applications such as contact tracing \cite{eames2003contact}. In an unbalanced PSI-WCA protocol, there are two servers each holding the same large set of keywords $X\subseteq [N]$ of size $n_X$, and a client holding a small set of keywords $Y\subseteq[N]$ of size $n_Y$. The client leans the weighted cardinality of $X\cap Y$, without the protocol revealing any information about $Y$ to the servers (except an upperbound on $n_Y$) or any additional information about $X$ to the client. 

\cite{cryptoeprint:2020/1599} designs an unbalanced PSI-WCA using DMPF in a straightforward way:
\begin{itemize}
  \item[(1)] The client invokes $\DMPF_{n_Y,N,\GG}.\Gen(1^\lambda, \hat{f}_{Y,W(Y)})$ to get a pair of keys $ (k_0,k_1)$, where $W(Y)$ is the set of weights (from group $\GG$) of elements in $Y$. Then the client send $k_0$ to server 0 and $k_1$ to server 1. 
  \item[(2)] Server $b$ computes $\sum_{x\in X}\DMPF_{n_Y,N,\GG}.\Eval_b(1^\lambda, k_b,x)$ and sends the result ${\sf sum}_b$ to the client. 
  \item[(3)] The client computes $W(X\cap Y)={\sf sum}_0+{\sf sum}_1$. 
\end{itemize}
We plug our new DMPF constructions in the above protocol, to obtain a significantly better server-respond time than the original realization in \cite{cryptoeprint:2020/1599} which instantiated $\DMPF_{n_Y,N,\GG}$ by na\"ive DMPF. 

\paragraph{Complementary perspectives of the two applications.}
We emphasize that the two applications we discussed have complementary requirements for their underlying DMPF. The PCG application typically demands an efficient $\DMPF.\FullEval$ for fast seed expansion, whereas the unbalanced PSI-WCA application requires efficient $\DMPF.\Gen$ and $\DMPF.\Eval$, along with a short key size. Consequently, different DMPF constructions exhibit varying performance across these applications, as we will demonstrate in the next section.


