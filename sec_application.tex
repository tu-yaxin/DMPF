\section{Applications}\label{sec:applications}
In this section we compare and discuss about the efficiency of different DMPF schemes in concrete application scenarios, namely when used for constructing pseudorandom correlation generator (PCG) and unbalanced private set intersection protocol (unbalanced PSI).  For convenience of discussion, we use $\DMPF_{t, N, \GG}$ to denote a DMPF scheme for $t$-point functions with domain $[N]$ and output group $\GG$. 

To give a rough impression, we list the number of accepting points $t$, the domain size $N$ and the output group $\GG$ of DMPF usually used in generating the PCG or unbalanced PSI protocol in \cref{tab:app_parameters}. 

\begin{table*}
  \caption{Parameters of DMPF in concrete applications. }
  \label{tab:app_parameters}
	\begin{tabular}{ccc}
    \toprule
		Concrete application &\makecell{Cost in terms of DMPF\\per correlation/execution}& Typical DMPF parameters \\
    \midrule
		PCG for OLE from Ring-LPN &\makecell{seedsize $\propto$ DMPF.$keysize$\\expand time $\propto$ DMPF.$FullEval()$} & \makecell{Number of accepting points: $5^2, 16^2, 76^2$\\Domain size: $2^{20}$\\Output group: $\ZZ_p$ where $\log p =128$}  \\
    \cline{1-3}
		PSI-WCA & \makecell{communication $\propto$ DMPF.$keysize$\\client computation $\propto$ DMPF.$Gen()$\\server computation $\propto$ DMPF.$Eval()$} & \makecell{Number of accepting points: any\\Domain size: $2^{128}$\\Output group: any}\\
    \bottomrule
	\end{tabular}
\end{table*}

\subsection{PCG for OLE from Ring-$\LPN$}

In this section we discuss the efficiency of different DMPF schemes in the PCG application. We begin by briefly introducing the protocol of PCG for OLE from Ring-$\LPN$ assumption, proposed in \cite{cryptoeprint:2022/1035}. 

\paragraph{The PCG protocol for OLE correlation.}The hardness assumption we will make use of is a variant of Ring-$\LPN$, called module-$\LPN$ assumption. 
\begin{definition}[Module-$\LPN$]\label{def:module-LPN}
    Let $c\ge 2$ be an integer, $R = \ZZ_p[X]/F(X)$ for a prime $p$ and a deg-$N$ polynomial $F(X)\in \ZZ_p[X]$, and $\mathcal{HW}_{R,t}$ be the uniform distribution over \`weight-$t$\' polynomials in $R$ whose degree is less than $N$ and has at most $t$ nonzero coefficients. 
    %(We write $\mathcal{HW}_t$ when $R$ is clear from the context. )
     For $R=R(\lambda)$, $t=t(\lambda)$ and $m=m(\lambda)$, we say that the module-$\LPN$ problem $R^c$-$\LPN$ is hard if for every nonuniform polynomial-time probabilistic distinguisher $\cA$, it holds that 
    \begin{align*}
        |&\Pr[\cA(\{\vec{a}^{(i)}, \ipd{\vec{a}^{(i)}}{\vec{s}}+\vec{e}^{(i)}\}_{i\in [m]})] \\
        -&\Pr[\cA(\{\vec{a}^{(i)},\vec{u}^{(i)}\}_{i\in [m]})]| \le \mathsf{negl}(\lambda)
    \end{align*}
    where the probabilities are taken over the randomness of $\cA$, random samples $\vec{a}^{(1)},\cdots, \vec{a}^{(m)}\leftarrow R^{c-1}$, $\vec{u}^{(1)},\cdots, \vec{u}^{(m)}\leftarrow R$, $\vec{s}\leftarrow \cHW_{R,t}^{c-1}$, and $\vec{e}^{(1)},\cdots, \vec{e}^{(m)}\leftarrow \cHW_{R,t}$. 

    When we only consider $m=1$, each $R^c$-$\LPN$ instance $ \ipd{\vec{a}}{\vec{s}}+\vec{e}$ can be restated as $\ipd{\vec{a'}}{\vec{e'}}$ where $\vec{a'}=1||\vec{a}$ and $\vec{e'}\leftarrow \cHW_{R,t}^c$. 
\end{definition}

The PCG protocol in \cite{cryptoeprint:2022/1035} generates seed for the OLE correlation $(x_0,x_1,z_0,z_1)\in R^4$ such that $x_0+x_1 = z_0\cdot z_1$, where $R = \ZZ_p[X]/F(X)$ for a prime $p$ and a deg-$N$ polynomial $F(X)\in \ZZ_p[X]$. The idea is to first set $z_b = \ipd{\vec{a}}{\vec{e_b}}$ (an $R^c$-$\LPN$ instance with public $\vec{a}$ and $\vec{e_b}\leftarrow \cHW_{R,t}^c$). Basing on the fact that $\ipd{\vec{a}}{ \vec{e_0}}\cdot \ipd{\vec{a}}{\vec{e_1}} = \ipd{\vec{a}\otimes \vec{a}}{\vec{e_0}\otimes \vec{e_1}}$, the next step is to additively share the tensor product $\vec{e_0}\otimes \vec{e_1}$ and each party can compute an additive share of $z_0\cdot z_1$. Note that the tensor product $\vec{e_0}\otimes\vec{e_1}$ consists of $c^2$ entries, each being an deg-$2N$ polynomial with at most ${t}^2$ nonzero coefficients. Therefore it can be shared by invoking $\DMPF_{{t}^2, 2N, \mathbb{Z}_p}$ for $c^2$ times.

One can compute the seed size and expanding time of this PCG protocol as follows: 
\begin{itemize}
    \item The seed size is $ct(\log N+\log p)$ bits for specifying $\vec{e_b}$ plus the $c^2\times $keysize of $\DMPF$. 
    \item The expanding time is $c^2$ multiplications in the deg-$2N$ polynomial ring \Yaxin{or $2c^2$ multiplications since $\vec a\otimes\vec a$ need also be computed? } plus $c^2\times$full-domain evaluation time of $\DMPF$. 
\end{itemize}

\begin{remark}[From OLE over polynomial ring $R$ to OLE over $\ZZ_p$]\label{rem:use_reducible_ring}
    Note that the above PCG protocol generates seed for OLE correlation over deg-$N$ polynomial ring $R$. One can immediately convert an OLE correlation over ring $R$ to \textbf{$N$ OLE correlations over $\ZZ_p$} if the polynomial $F(X)$ splits into $N$ distinct linear factors modulo $p$\cite{cryptoeprint:2022/1035}. Therefore we mostly consider reducible $F$ and more concretely $F$ a two-power cyclotomic due to its useful properties. 
\end{remark}

\paragraph{Amortized expanding time for each OLE correlation over $\ZZ_p$}The amortized expanding time for each OLE correlation over $\ZZ_p$ is computed by 
\[
  T_\text{Amortized} = c^2\cdot(2\bar{T}^\MULT_{N} + \bar{T}^\FullEval_{t,N})
\]
where 
\[
  \bar{T}^\MULT_{N} := \frac{\text{deg}<N\text{ polynomial multiplication}}{N}
\]
is the amortized cost for computing one deg$<N$ polynomial multiplication, and 
\[
  \bar{T}^\FullEval_{t,N}:=\frac{\DMPF_{t^2, 2N, \ZZ_p}.\FullEval}{N}
\]
is the amortized cost for computing a share of an entry of $\vec{e_0}\otimes\vec{e_1}$. This cost differs under different $\DMPF_{t^2, 2N, \ZZ_p}$ instantiations: \Yaxin{$T_{G_{\sf conv}}$ is ignored for now. }
\begin{itemize}
  \item Sum of $\DPF$s: $2t^2\times T_G$.
  %\item Sum of DPFs ($\cRHW_t$): $2t\times T_G$.
  \item CBC-based $\DMPF$ ($\cHW_t$): $2w\times T_G$ + $2w\times T_{\rm hash}$\footnote{The hash function's domain and range are related to $t$. }.
  \item Big-state $\DMPF$: $2T_{G^*}$, where $G^*$ maps $\lambda$-bit to $(2\lambda+2t)$-bit. 
  \item $\OKVS$-based $\DMPF$: $2\left(T_G + \OKVS.\Decode\right)$\footnote{The $\OKVS$ scheme encodes at most $t$ key-value pairs. $\OKVS.\Decode$ usually takes a small number of field addition or field multiplication in $\FF_{2^\lambda}$, depending on the implementation. }.
\end{itemize}



\paragraph{Using the regular noise distribution to split $\DMPF_{t^2, 2N, \ZZ_p}$}

A previous optimization in \cite{cryptoeprint:2022/1035}, aiming to share the entries of $\vec{e_0}\otimes \vec{e_1}$ through $\DPF$s but with less overhead (in contrast to the $2t^2\times T_G$ cost before), is to substitute $\cHW_{R,t}$ with the distribution of random regular weight-$t$ polynomials denoted as $\cRHW_{R,t}$. Each regular weight-$t$ polynomial $e$ contains exactly one nonzero coefficient $e_j$ in the range of degree $[j\cdot N/t, (j+1)\cdot N/t-1]$ for $j=0,\cdots t-1$. When multiplying two regular weight-$t$ polynomials $e$ and $f$, $e_i\cdot f_j$ contributes to a coefficient in the range of degree $[(i+j)\cdot N/t, (i+j+2)\cdot N/t-2]$. Therefore the deg-$2N$ polynomial $e\cdot f$ can be shared by invoking $\{\DMPF^{(k)} = \DMPF_{\min(k,2t-k), 2N/t, \ZZ_p}\}_{1\le k\le 2t-1}$, where $\DMPF^{(k)}$'s domain corresponds to the coefficients in the range of degree $[(k-1)N/t, (k+1)N/t-2]$ in the resulting polynomial $e\cdot f$. Then each $\DMPF^{(k)}$ in the set $\{\DMPF^{(k)}\}_{1\le k\le 2t-1}$ can be instantiated by one of the mentioned schemes: sum of $\DPF$s (used in \cite{cryptoeprint:2022/1035}), CBC-based, big-state, or $\OKVS$-based $\DMPF$. We note that the efficiency (in terms of $\FullEval$ time) of all these instantiations are more or less related to the number of nonzero inputs in the targeted $\DMPF$, therefore using $\cRHW_{R,t}$ instead to $\cHW_{R,t}$ reduces the number of nonzero inputs from $t^2$ to at most $t$ may be benefitial in some occasions.

An alternative way to split $\DMPF_{t^2, 2N, \ZZ_p}$, basing on the previous observation, is to share the coefficients of $e\cdot f$ by invoking $\{\DMPF'^{(k)} = \DMPF_{2\min(k+1, 2t-k)-1, N/t, \ZZ_p}\}_{0\le k\le 2t-1}$, where $\DMPF'^{(k)}$'s domain corresponds to the coefficients in the range of degree $[kN/t, (k+1)N/t -1]$. Since the number of nonzero coefficients in $[kN/t, (k+1)N/t -1]$ is upperbounded by the sum of number of nonzero coefficients in $[(k-1)N/t, (k+1)N/t -2]$ and in $[kN/t, (k+2)N/t -2]$, $\DMPF'^{(k)}$ has at most $2\min(k+1, 2t-k)-1$ nonzero inputs. The advantage of using $\{\DMPF'^{(k)}\}_{0\le k\le 2t-1}$ is that the previous concatenation through $\{\DMPF^{(k)}\}_{1\le k\le 2t-1}$ creates overlapping ranges, which doubles the number of PRG invocations when realizing by CBC-based, big-state, and $\OKVS$-based $\DMPF$. By using $\{\DMPF'^{(k)}\}_{0\le k\le 2t-1}$, the ranges are not overlapping and therefore maintains the minimal PRG invocations, while also preserving a relatively small number of nonzero inputs (at most $2t-1$) in each $\DMPF'^{(k)}$. 

Previously, \cite{cryptoeprint:2022/1035} uses sum of DPFs to instantiate $\{\DMPF^{(k)}\}_{1\le k\le 2t-1}$ in the first optimized design with regular noise distribution. It indicates using batch code to achieve DMPF as another optimization but not in the clear. We'll analyze the cost of this PCG protocol under the following settings:
\begin{itemize}
    \item[(1)]with noise distribution $\cHW_{R,t}$ and each multiplication of sparse polynomials is shared by $\DMPF_{t^2, 2N, \mathbb{Z}_p}$
    \item[(2)]ith noise distribution $\cRHW_{R,t}$ and each multiplication of regular sparse polynomials is shared by $$\{\DMPF^{(k)} = \DMPF_{\min(k,2t-k), 2N/t, \ZZ_p}\}_{1\le k\le 2t-1}$$
    \item[(3)]ith noise distribution $\cRHW_{R,t}$ and each multiplication of sparse polynomials is shared by $$\{\DMPF'^{(k)} = \DMPF_{2\min(k+1, 2t-k)-1, N/t, \ZZ_p}\}_{0\le k\le 2t-1}$$
\end{itemize}
\Yaxin{Dec 27: It is also mentioned using more advanced noise distributions to avoid overlapping ranges. For now it remains to give the concrete costs for (1) $\DMPF_{t^2, 2N, \mathbb{Z}_p}$; (2)$\DMPF_{1/2/\dots/t, 2N/t, \ZZ_p}$; (3)$\DMPF_{1/2/\dots/(2t-1), N/t, \ZZ_p}$ under sum of DPFs/CBC-based/big-state/OKVS-based instantiations. }

We'll instantiate $\DMPF$ in different ways as listed in \cref{tab:formulas_DMPF_comparison}. The costs of PCG protocols under different settings are listed in \cref{tab:PCG_plug_in_formula}. 
\begin{table*}
    \renewcommand\arraystretch{1.5}
    \begin{threeparttable}
    \caption{Seed size and expanding time of PCG protocols for the same $(\lambda,N, c, t)$ with different choices of noise distributions in module-LPN assumption, and with different DMPF instantiations. We use \cref{con:OKVS_sparse_matrix} as an instantiation of OKVS. The seed size is represented by total DMPF share size and the expanding time is represented by total $\DMPF.\FullEval$ time. The PRG evaluations in the first $\log (2N)$ layers and in the convert layer are both regarded as the same PRG. $e = m/t$ in the second row represents the expansion parameter for PBC where $m$ is the number of buckets, and $e'$ in the last row represents the expansion parameter (the inverse of rate) for OKVS. }
	\label{tab:PCG_plug_in_formula}
		\begin{tabular}{cccc}
            \toprule
			$\DMPF$ instantiation & Noise type & Total share size & Total $\FullEval$ time \textcolor{red}{(only listed PRG and $\OKVS$)} \\
            \midrule

            \multirow{2}{*}{Sum of DPFs} & regular & $c^2t^2\lambda\log(2N/t)+c^2t^2\log p$ & $4c^2tN\times$PRG \\
            %\cline{2-4}
             & nonregular & $c^2t^2\lambda\log(2N)+c^2t^2\log p$ & $4c^2t^2N\times$PRG\\
             \cline{1-4}
            \multirow{2}{*}{Batch-code DMPF} & regular & $ec^2t^2\lambda\log(\frac{wN}{et})+ec^2t^2\log p$ & $8c^2wN\times$PRG \\
            %\cline{2-4}
            & nonregular & $ec^2t^2\lambda\log(\frac{2wN}{et^2})+ec^2t^2\log p$ & $4c^2wN\times$PRG\\
            \cline{1-4}

            \multirow{2}{*}{Big-state DMPF} & regular & $c^2t^2(\lambda+\frac{4}{3}t)\log (2N)+c^2t^2\log p$ & $8c^2N\times$PRG$^*$\tnote{1} \\
            %\cline{2-4}
            & nonregular & $c^2t^2(\lambda+2t)\log (2N)+c^2t^2\log p$ & $4c^2N\times$PRG$^*$ \\
            \cline{1-4}

            \multirow{2}{*}{OKVS-based DMPF} & regular & $e'c^2t^2\lambda\log(2N/t)+e'c^2t^2\log p$ & $8c^2N\times$PRG+$8c^2N\times\OKVS.\Decode$  \\
            %\cline{2-4}
            & nonregular & $e'c^2t^2\lambda\log(2N)+e'c^2t^2\log p$ & $4c^2N\times$PRG+$4c^2N\times\OKVS.\Decode$\\
            \bottomrule
		\end{tabular}
    \begin{tablenotes}
      \item [1] The PRG used in big-state DMPF maps from $\{0,1\}^\lambda$ to $\{0,1\}^{2\lambda+2t^2}$ whose computation time should grow with $t^2$. 
    \end{tablenotes}
  \end{threeparttable}
\end{table*}

\Yaxin{One caveat: can CBC-based / OKVS-based DMPF fit into the regular design, while it requires shares of $1,2,3$-point functions?}

%From \cref{tab:PCG_plug_in_formula} we can see that if we allow small blowup of the seed size of PCG, then we can gain much faster seed expansion by using nonregular noise distribution and either CBC-based DMPF or big-state DMPF or OKVS-based DMPF. 

\iffalse
\paragraph{Comparing the entropy of regular and nonregular noise.}View the $R^c$-$\LPN$ problem as a restricted case of primal $\LPN_{(c-1)N, N,\mathcal{D}}$ with public matrix being a circular matrix of size $(c-1)N\times N$, and the noise distribution over $R$ (or equivalently written as a vector in $\ZZ_p^N$) being either $\cHW_{R, t}$ or $\cRHW_{R,t}$. Moreover, let's consider using the entropy of the noise distribution as the security parameter (which means that we consider the adversary's attack being randomly guessing the secret). We use $(N_n, c_n, t_n)$ to denote the parameters for $\cHW^{\otimes c_n}_{R_n, t_n}$ and $(N_r, c_r, t_r)$ to denote the parameters for $\cRHW^{\otimes c_r}_{R_r, t_r}$. We'll compute the relation of two sets of parameters when they reach the same amount of entropy, under the assumption that $N\gg t$ (in PCG $N = 2^{20}$ and $t\le 76$). By Stirling formula, the entropy of $\cHW^{\otimes c_n}_{R_n, t_n}$ is 
\[
  \begin{split}
    \ln 2 \cdot E_n &\approx c_n\cdot \left(N_n\ln N_n - (N_n-t_n)\ln(N_n-t_n) - t_n\ln t_n\right) \\&\approx c_n t_n\left(\ln N_n - \ln t_n + 1\right) \quad(\text{when }N_n\gg t_n)
  \end{split}\]
and the entropy of $\cRHW^{\otimes c_r}_{R_r, t_r}$ is 
\[
    \ln 2\cdot E_r \approx c_rt_r\left(\ln N_r - \ln t_r\right) \quad(\text{when }N_r\gg t_r)
\]
To force $E_n = E_r$, let's consider fixing two varibles and change one: 
\begin{enumerate}
  \item Fix $(c_n, t_n) = (c_r, t_r)$, then $N_r = \mathrm{e}\cdot N_n$. 
  \item Fix $(N_n, c_n) = (N_r, c_r)$, then $t_n<t_r<t_n+1$. 
  \item Fix $(N_n, t_n) = (N_r, t_r)$, then $\frac{c_r}{c_n} = 1 + \frac{1}{N_n - t_n}\approx 1.05$. 
\end{enumerate}
All of these three changes of variables don't seem to lead to noticeable changes in the amortized expanding time for each OLE correlation. 
\fi

\paragraph{Setting parameters $(N,c,t)$ against best attacks}Next we plug in concrete parameters and evaluate the performance of different $\DMPF$ schemes under different PCG parameter settings. 

The parameters $(N, c, t)$ should be set in a way that the corresponding $R^c$-$\LPN$ problem is secure with computational security parameter $\lambda$. In \cite{cryptoeprint:2022/1035} the parameters are taken to be $(\lambda, N, c, t) \in\{ (128, 2^{20}, 8, 5), (128, 2^{20}, 4, 16), (128, 2^{20}, 2, 76)\}$ against several attacks, with the best among which predicted to be the SD or ISD family. \Yaxin{Dec 28: Using the new lowerbound in \cite{cryptoeprint:2022/712} (if I understood it correctly), the setting of parameters (fixing $\lambda=128$ and $N = 2^{20}$) becomes $(\lambda, N, c, t) \in\{ (128, 2^{20}, 8, 5), (128, 2^{20}, 4, 14), (128, 2^{20}, 2, 70)\}$. In fact, $N$ is independent to other parameters because $R$ is a reducible polynomial ring. }The three parameters are comparable in efficiency in \cite{cryptoeprint:2022/1035} because they used the second $\DMPF$ instantiation (concatenation of DPFs under $\cRHW_t$ distribution) whose cost scales with $c^2t$. By using the new $\DMPF$ instantiations we may see significant difference among the three tuples of parameters. 

\textbf{How to compute $(N,c,t)$: }We set the parameters $(\lambda, c, N, t)$ such that the best attack requires at least $2^\lambda$ arithmetic operations over field $\FF_p$ of size approximately $2^{128}$. 

An $R^c$-$\LPN$ instance $a\cdot e$ can be viewed as a (dual-)$\LPN_{cN,N,\cHW_{N,t}^{\otimes c}}$ instance $\{H, H\cdot \vec{e}\}$, where $H\in\ZZ_p^{N\times cN}$ is a concatenation of $c$ circular matrices representing multiplication with ${a}$, and $\vec{e}\in \ZZ_p^{cN}$ with distribution $\cHW_{N,t}^{\otimes c}$ represents the concatenation of coefficients of $e$. The bit security of the $R^c$-$\LPN$ problem is equivalent to the bit security of the described (dual-)$\LPN$ problem. As in \cite{cryptoeprint:2022/1035}, we consider the bit security of the described (dual-)$\LPN$ problem to be the same as the bit security of the standard (dual-)$\LPN$ problem, whose error distribution is $\cHW_{cN, ct}$, the random weight-$ct$ noises. 

According \cite{cryptoeprint:2022/1035}, for $R$ from an irreducible $F$, we lowerbound the number of arithmetic operations by $N\cdot (c\cdot \frac{N}{N-1})^{ct}\approx N\cdot c^{ct}$. \Yaxin{Dec 28: \cite{cryptoeprint:2022/712} seems to indicate a better lowerbound of $N^{2.8}\cdot c^{ct}$. } For $R$ from a reducible $F$, the number of arithmetic operations is lowerbounded by $2^i\cdot c^{w_i}$ \Yaxin{Dec 28: or $2^{2.8 i}\cdot c^{w_i}$ by \cite{cryptoeprint:2022/712}}, where $w_i$ is the expected number of noisy coordinates modulo an 1-sparse deg-$2^i$ polynomial and $i:=\text{ the smallest integer such that }w_i<2^i$. Then $t$ is computed by the equation 
\[w_i = c\cdot 2^i\left(1-(1-2^{-i})^t\right)\]
\Yaxin{Dec 29: In \cite{cryptoeprint:2022/1035} there are two formulas calculating $w_i$. One is as above, and the other is $$w_i=ct-2^ic+((2^i-1)c+ct)\cdot \left(1-2^{-i}\right)^{t-1}$$The second one does not make sense to me but is used to compute the concrete results. Nevertheless the two formulas computes similar results so I used the first one which makes more sense to me. }
  %\begin{split}
    %i:=&\text{ the smallest integer such that }\lambda - i<2^i\\ 
    %\mathop{\arg\min}\limits_{1\le j\le \log N}\left(2^j\cdot c^{W_j}\right)\\ 
  %\end{split}



%\Yaxin{Previous calculation as a reference: } choosing the big-state DMPF for $t<8$ and the OKVS-DMPF for $t\ge 8$ gives at least $\times 2$ acceleration on expand time over other choices with sacrifice on the keysize. There is a tradeoff between the CBC-based and OKVS-DMPF in that the OKVS-DMPF always provides a $\sim\times 2$ acceleration on expand time, but a loss in seed size that when $t$ is large it may blow up the seed size to $\sim \times 2$ that of the CBC-based-DMPF. 

\subsection{Unbalanced PSI-WCA}
A private set intersection (PSI) protocol allows two parties with input $X$, $Y$ being two sets to learn about their intersection $X\cap Y$ without revealing additional information of $X$ or $Y$. We denote by PSI-WCA (weighted cardinality) a variant of PSI that computes the weighted cardinality of elements in $X\cap Y$ where the weights are determined by a pre-fixed function $w(\cdot)$. 

We will be interested in \emph{unbalanced} PSI-WCA where $|X|\gg |Y|$ and the output should be received by the party holding $Y$. In this problem we call the party holding $X$ as the server, and the party holding $Y$ as the client. If further the big set $X$ is held by two non-colluding servers, then such an unbalanced PSI-WCA protocol can be constructed from DMPF, as suggested in \cite{cryptoeprint:2020/1599}: 
\begin{itemize}
  \item The client invokes $\DMPF.\Gen(1^\lambda, \hat{f}_{Y,w(Y)})\rightarrow (k_0,k_1)$, where $w(Y)$ is the set of weights of elements in $Y$. Then the client send $k_0$ to server 0 and $k_1$ to server 1. 
  \item Server $b$ computes $s_b=\sum_{x\in X}\DMPF.\Eval_b(1^\lambda, k_b,x)$ and send it back to the client. 
  \item The client computes $s_0+s_1$, which will be the weighted cardinality of $X\cap Y$. 
\end{itemize}
One caveat is that this protocol reveals information about $Y$ that is leaked by $\DMPF$. Plugging in any $\DMPF$ instantiations we have mentioned, the size of $|Y|$ will be leaked to the servers. 

The cost of our unbalanced PSI-WCA can be computed as follows: 
\begin{itemize}
  \item The communication cost equals the keysize of $\DMPF$. 
  \item The client computation time equals the key generation time of $\DMPF$. 
  \item The server computation time equals $|X|\times$the evaluation time of $\DMPF$. 
\end{itemize}

We'll instantiate $\DMPF$ in different ways as listed in \cref{tab:formulas_DMPF_comparison}. As suggested in \cite{cryptoeprint:2020/1599}, we take an infeasibly large domain for the sets $X$ and $Y$ to locate, whose size is $N = 2^{128}$. The set sizes $|X|$ and $|Y|$ can vary depending on application scenarios. Since $|Y|$ is the crucial factor that distinguishes different $\DMPF$ instantiations, we will only consider the change of $|Y|$. The costs of PSI-WCA protocols under different settings of $|Y|$ are listed in \cref{tab:PSI_plug_in_formula}. 


\begin{table*}
  \renewcommand\arraystretch{1.5}
  \begin{threeparttable}
  \caption{Communication cost, client and server computation time of the PSI-WCA protocol for domain size $N = 2^{128}$, weight group $\GG$, and  different choices of client's set size $|Y|$. We use \cref{con:OKVS_sparse_matrix} as an instantiation of OKVS. The PRG evaluations in the first $\log N$ layers and in the convert layer are both regarded as the same PRG. $e$ in the second row represents the expansion parameter for PBC, and $e'$ in the last row represents the expansion parameter for OKVS. }
\label{tab:PSI_plug_in_formula}
  \begin{tabular}{cccc}
          \toprule
    $\DMPF$ instantiation & Communication cost & Client computation time & Server computation time \\
          \midrule

          Sum of DPFs & $|Y|\lambda\log N+|Y|\log|\GG|$ & $2|Y|\log N\times $PRG & $|X|\cdot |Y|\log N\times $PRG\\

          Batch-code DMPF & $e|Y|\lambda\log(\frac{wN}{e|Y|})+e|Y|\log|\GG|$ & $2e|Y|\log(\frac{wN}{e|Y|})\times$PRG & $w|X|\log(\frac{wN}{e|Y|})\times$PRG\\

          Big-state DMPF & $|Y|(\lambda+2|Y|)\log N+|Y|\log|\GG|$ & $2|Y|\log N\times $PRG$^*$\tnote{1} & $|X|\log N \times$PRG$^*$\\

          OKVS-based DMPF& $e'|Y|\lambda\log N +e'|Y|\log|\GG|$ & $2|Y|\log N\times$PRG+$\log N\times \OKVS.\Encode$ & $|X|(\log N\times$PRG+$\log N\times \OKVS.\Decode)$\\
          \bottomrule
  \end{tabular}
  \begin{tablenotes}
    \item [1] The PRG used in big-state DMPF maps from $\{0,1\}^\lambda$ to $\{0,1\}^{2\lambda+2|Y|}$ whose computation time should grow with $|Y|$.
  \end{tablenotes}
\end{threeparttable}
\end{table*}



%\Yaxin{Previous calculation as a reference: }A short conclusion is using big-state DMPF for $t<64$ and the OKVS-DMPF for $t\ge 64$ gives at $\sim \times 2$ faster Eval() time and faster Gen() time compared to the naive and CBC-based construction. The keysize ($\propto$ communication complexity) of our choice is usually smaller than the CBC-based DMPF and slightly larger than the naive construction. 
\subsection{Security analysis}
\subsection{Heavy-hitters}
private heavy-hitter\\
or parallel ORAM?