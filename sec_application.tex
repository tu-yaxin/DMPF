\section{Applications}\label{sec:applications}
In this section we evaluate different DMPF schemes in concrete application scenarios: pseudorandom correlation generator (PCG) and unbalanced private set intersection protocol (unbalanced PSI). For convenience of discussion, we use $\DMPF_{t, N, \GG}$ to denote a DMPF scheme for $t$-point functions with domain $[N]$ and output group $\GG$. 

\subsection{PCG for OLE from Ring-$\LPN$}
The protocol in \cite{cryptoeprint:2022/1035} for PCG for OLE correlation basing on Ring-$\LPN$ assumption uses DMPF schemes to succinctly share sparse vectors which are interpreted as coefficients of low-weight polynomials. We focus on the following setting of this protocol, mentioned in \Cref{sec:PCG_OLE_protocol}, which achieves good computational efficiency and security simultaneously: 
\begin{enumerate}
    \item The hardness assumption is $R^c$-$\LPN$ assumption where $R = \ZZ_p[X]/F(X)$ for a prime $p$ and a deg-$N$ polynomial $F(X)\in\ZZ_p[X]$, while the noise distribution is the regular weight-$t$ distribution $\cRHW_{R,t}$. 
    \item The polynomial ring $R = \ZZ_p[X]/F(X)$ where $F(X) = X^N+1$ is cyclotomic with $N$ being a power of two. 
    \item The underlying field's order $p$ is prime such that that $F(X)$ splits completely into linear factors modulo $p$, in order to convert an OLE correlations over $R$ to $N$ OLE correlations over $\ZZ_p$. 
\end{enumerate}
\begin{remark}
    In fact, using either the big-state DMPF, the PBC-based DMPF, or the OKVS-based DMPF schemes to share the low-weight polynomials does not need the restriction of the noise distribution to be regular. \red{However our experiments show that restricting to the regular noise distribution still leads to faster $\Expand$ time, so we reserve this setting. }\Yaxin{Tentatively. Double check with final implementation. }
\end{remark}

Leveraging the regular noise distribution and the cyclotomic polynomial $F$, the secret sharing task for generating one OLE correlation can be achieved using $c^2t\times \DMPF_{t,2N/t, \ZZ_p}$. Then the seed size, $\Gen$ time and $\Expand$ time of this PCG protocol are: 
\begin{itemize}
    \item The seed size is $ct(\log N+\log p)$ bits for specifying $\vec{e_b}$ plus the $c^2t\times $ keysize of $\DMPF_{t,2N/t, \ZZ_p}$. 
    \item The $\Gen$ time is $c^2$ multiplications in the ring $R$ plus $c^2t\times$ $\DMPF_{t,2N/t, \ZZ_p}.\Gen$. 
    \item The $\Expand$ time is $c^2$ multiplications in the ring $R$ plus $c^2t\times$ $\DMPF_{t,2N/t, \ZZ_p}.\FullEval$. 
\end{itemize}

We set the parameters $(\lambda,p,N,c,t)$ where $\lambda$ is the computational security parameter following the security analysis for the protocol: 
\[
\lambda = 128,\,\red{\log p \approx 128},\, \red{N = 2^{20}}, \begin{array}{cc}
    c=2, & t=66 \\
    c=4, & t=14\\
    c=8, & t=5
\end{array}
\]
\Yaxin{These should be modified to match with the implementation. }

\red{Insert experiment results. }
\begin{table*}
\renewcommand\arraystretch{1.25}
%\scalebox{1}{
\begin{threeparttable}
\caption{TBD}
\label{tab:application_PCG}
    \begin{tabular}{cccccccc}
    \toprule 
    $\lambda$ & $N$ & $\log p$& $c$ & $t$ & DMPF scheme & Seed size & $\Expand$ time (ns)\\
    
    %&Sum of DPFs & PBC-based DMPF & Big-state DMPF & OKVS-based DMPF\\

    \midrule
    \multirow{4}{*}{128} & \multirow{4}{*}{$2^{20}$}&
    \multirow{4}{*}{128}&
    \multirow{4}{*}{2}& \multirow{4}{*}{66}&
    Sum of DPFs & & \\\cline{6-8}
    & & & & &Big-state DMPF & & \\\cline{6-8}
    & & & & &PBC-based DMPF& & \\\cline{6-8}
    & & & & &OKVS-based DMPF & & \\
     
     \cline{1-8}

     \multirow{4}{*}{128} & \multirow{4}{*}{$2^{20}$}& 
     \multirow{4}{*}{128}&
     \multirow{4}{*}{4}& \multirow{4}{*}{14}
    &Sum of DPFs & & \\\cline{6-8}
    & & & & &Big-state DMPF & & \\\cline{6-8}
    & & & & &PBC-based DMPF& & \\\cline{6-8}
    & & & & &OKVS-based DMPF & & \\
     
     \cline{1-8}

     \multirow{4}{*}{128} & \multirow{4}{*}{$2^{20}$}& 
     \multirow{4}{*}{128}&
     \multirow{4}{*}{8}& \multirow{4}{*}{5}
    &Sum of DPFs & & \\\cline{6-8}
    & & & & &Big-state DMPF & & \\\cline{6-8}
    & & & & &PBC-based DMPF& & \\\cline{6-8}
    & & & & &OKVS-based DMPF & & \\

    \bottomrule
\end{tabular}	
%\begin{tablenotes}
%\end{tablenotes}
\end{threeparttable}
%}
\end{table*}

%From \cref{tab:PCG_plug_in_formula} we can see that if we allow small blowup of the seed size of PCG, then we can gain much faster seed expansion by using nonregular noise distribution and either CBC-based DMPF or big-state DMPF or OKVS-based DMPF. 

\subsection{Unbalanced PSI-WCA}
\cite{cryptoeprint:2020/1599} suggests using DMPF to construct an non-colluding 2-server-1-client unbalanced PSI-WCA (weighted cardinality) protocol, as described in \Cref{sec:PSI_prelim}. Denote the size of the servers' set $X$ and the client's set $Y$ as $n_X, n_Y$ respectively. 
The cost of our unbalanced PSI-WCA can be computed as follows: 
\begin{itemize}
  \item The communication cost is the keysize of $\DMPF_{n_Y,N,\GG}$. 
  \item The client's computation time equals to the running time of $\DMPF_{n_Y,N,\GG}.\Gen$. 
  \item The server's computation time equals to $n_X\times$ $\DMPF_{n_Y,N,\GG}.\Eval$. 
\end{itemize}

We follow the typical setting of parameters $N = 2^{128}$, \red{$n_X = 2^{20}$, $\GG\cong \ZZ_{2^{16}}$}, and let $n_Y$ vary. The performance of different DMPF schemes are listed in \cref{tab:PSI_plug_in_formula}. 


\begin{table*}
  \renewcommand\arraystretch{1.5}
  \begin{threeparttable}
  \caption{Communication cost, client and server computation time of the PSI-WCA protocol for domain size $N = 2^{128}$, weight group $\GG$, and  different choices of client's set size $|Y|$. We use \cref{con:OKVS_sparse_matrix} as an instantiation of OKVS. The PRG evaluations in the first $\log N$ layers and in the convert layer are both regarded as the same PRG. $e$ in the second row represents the expansion parameter for PBC, and $e'$ in the last row represents the expansion parameter for OKVS. }
\label{tab:PSI_plug_in_formula}
  \begin{tabular}{cccc}
          \toprule
    $\DMPF$ instantiation & Communication cost & Client computation time & Server computation time \\
          \midrule

          Sum of DPFs & $|Y|\lambda\log N+|Y|\log|\GG|$ & $2|Y|\log N\times $PRG & $|X|\cdot |Y|\log N\times $PRG\\

          Batch-code DMPF & $e|Y|\lambda\log(\frac{wN}{e|Y|})+e|Y|\log|\GG|$ & $2e|Y|\log(\frac{wN}{e|Y|})\times$PRG & $w|X|\log(\frac{wN}{e|Y|})\times$PRG\\

          Big-state DMPF & $|Y|(\lambda+2|Y|)\log N+|Y|\log|\GG|$ & $2|Y|\log N\times $PRG$^*$\tnote{1} & $|X|\log N \times$PRG$^*$\\

          OKVS-based DMPF& $e'|Y|\lambda\log N +e'|Y|\log|\GG|$ & $2|Y|\log N\times$PRG+$\log N\times \OKVS.\Encode$ & $|X|(\log N\times$PRG+$\log N\times \OKVS.\Decode)$\\
          \bottomrule
  \end{tabular}
  \begin{tablenotes}
    \item [1] The PRG used in big-state DMPF maps from $\{0,1\}^\lambda$ to $\{0,1\}^{2\lambda+2|Y|}$ whose computation time should grow with $|Y|$.
  \end{tablenotes}
\end{threeparttable}
\end{table*}

%\subsection{Heavy-hitters}private heavy-hitter or parallel ORAM?